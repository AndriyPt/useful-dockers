Львівський національний університет ім. І.Я. Франка

УДК 517.958:539.377

Спеціальність 01.05.02

Математичне моделювання та обчислювальні методи

Застосування методів граничних елементів до розв'язування задач
термопружності у композитах

Здобувач наукового ступеня

кандидат фіз.-мат. наук

Петльований А.Т.

Науковий керівник

кандидат фіз.-мат. наук

Гудзь Р.В.

Львів -- 2006

\hypertarget{ux437ux43cux456ux441ux442}{%
\section[Зміст]{\texorpdfstring{\protect\hypertarget{anchor}{}{}Зміст}{Зміст}}\label{ux437ux43cux456ux441ux442}}

\hypertarget{ux432ux441ux442ux443ux43f}{%
\section[Вступ]{\texorpdfstring{\protect\hypertarget{anchor-1}{}{}Вступ}{Вступ}}\label{ux432ux441ux442ux443ux43f}}

\emph{\textbf{Актуальність теми.}} Більшість елементів сучасного
приводу, конструкцій чи побутові речі виготовлені із гетерогенного
матеріалу, зокрема із композиційного матеріалу або композиту. Тому
виникає потреба у розробці методів, які б дозволяли розраховувати
напружено-деформований стан, а також температуру.

Широко вживаними протягом кількох десятиріч методом розв'язування задач
такого класу є проекційно-сітковий метод або метод скінчених елементів
(МСЕ). Завдяки таким властивостям як розрідженість і симетричність
матриць для класу задач у локально-неоднорідних тілах, різні варіанти
МСЕ набули популярності.

Із МСЕ цілком успішно може конкурувати, а іноді бути ефективнішим,
велике сімейство методів граничних елементів
(МГЕ){[}\protect\hyperlink{anchor-2}{2}{]}{[}\protect\hyperlink{anchor-3}{4}{]}{[}\protect\hyperlink{anchor-4}{6}{]}{[}\protect\hyperlink{anchor-5}{7}{]}{[}\protect\hyperlink{anchor-6}{9}{]}{[}\protect\hyperlink{anchor-7}{12}{]}{[}\protect\hyperlink{anchor-8}{16}{]},
які базуються на ідеях розроблених
С.Г.Міхліном{[}\protect\hyperlink{anchor-9}{19}{]},
Н.І.Мусхелішвілі{[}\protect\hyperlink{anchor-10}{20}{]},
В.Д.Купрадзе{[}\protect\hyperlink{anchor-11}{17}{]} та інші. Ці
обставині особливо сприяють такі переваги, як зменшення на одиницю
геометричної розмірності задачі, простота розв'язування для безмежних
областей, можливість обчислювати у лінійних задачах значення шуканих
величин у довільній точці без нової апроксимації, простота
дискретизації.

У свою чергу серед класу МГЕ можна виділити такі методи як комплексний
метод граничних елементів (КМГЕ) і метод при граничних елементів (МПГЕ).
На відміну від КМГЕ, який є видозміненим варіантом класичного МГЕ, МПГЕ
базується на більш радикальніших ідеях. У порівнянні із МГЕ, МПГЕ
простіший для реалізації, проте у більшості випадків дає точніші
результати при тих самих затратах обчислювальних ресурсів.

Наукове завдання дисертаційної роботи -- розробка ефективного методу для
розв'язування задач термопружності у композитах є актуальною проблемою.

\emph{\textbf{Зв'язок роботи з науковими програмами, планами та
темами.}} Робота виконувалась на кафедрі програмування львівського
національного університету імені Івана Франка, а також у інституті
прикладних проблем механіки і математики імені Ярослава Підстригача НАН
України, відділ термомеханіки в рамках держбюджетної теми ``Розробка
аналітико-чисельних методів дослідження напруженого стану неоднорідних
тіл з тепловими та залишковими деформаціями і дефектами структури''.
(!!! Номер теми !!!)

\emph{\textbf{Мета роботи та задачі дослідження.}} Метою роботи є:

\begin{itemize}
\tightlist
\item
  побудова нових методів розв'язування задач термопружності у композитах
\item
  дослідження ефективності нових методів
\item
  порівняння швидкості збіжності і використання ресурсів із відомими
  методами
\end{itemize}

\emph{Об'єктом} дослідження є композити.

\emph{Предметом }дослідження є нові методи розв'язування задач
термопружності для композитів.

\emph{Методи дослідження. }Для розв'язування поставлених задач
використовується метод поєднання граничних і скінчених елементів.

\emph{\textbf{Наукова новизна одержаних результатів.}} Наступні
результати отримані у роботі є новими:

\begin{enumerate}
\def\labelenumi{\arabic{enumi}.}
\item
  побудований метод на основі КМГЕ і МСЕ для розв'язування задачі
  теплопровідності у тілах із локальними неоднорідностями
\item
  для тіл із локальними неоднорідностями побудований метод на основі
  МПГЕ і МСЕ для розв'язування задач:

  \begin{enumerate}
  \def\labelenumii{\alph{enumii}.}
  \tightlist
  \item
    пружності
  \item
    термопружності
  \end{enumerate}
\end{enumerate}

\emph{\textbf{Практичне значення отриманих результаті.}}\textbf{
}Отримані результати дозволяють ефективніше розв'язувати актуальні
проблеми, які ставить перед нами народне господарство.

\emph{\textbf{Апробація результатів дисертації.}} Основні результати
виконаних досліджень доповідалися та обговорювалися на:

\begin{enumerate}
\def\labelenumi{\arabic{enumi}.}
\tightlist
\item
  Восьмій Всеукраїнській науковій конференції (25-27 вересня 2001 р., м.
  Львів) ``Сучасні проблеми прикладної математики та інформатики'';
\item
  Дев'ятій Всеукраїнській науковій конференції (24-26 вересня 2002 р.,
  м. Львів) ``Сучасні проблеми прикладної математики та інформатики'';
\item
  Десятій Всеукраїнській науковій конференції (23-25 вересня 2003 р., м.
  Львів) ``Сучасні проблеми прикладної математики та інформатики'';
\item
  Одинадцятій Всеукраїнській науковій конференції (!!! 2004р., м. Львів)
  ``Сучасні проблеми прикладної математики та інформатики'';
\item
  Дванадцятій Всеукраїнській науковій конференції (4-6 жовтня 2005р., м.
  Львів) ``Сучасні проблеми прикладної математики та інформатики'';
\item
  Науковому семінарі ІППММ НАНУ імені Я.С.Підстригача, присвяченої
  пам'яті Є.Г.Грицька (22 квітня 2003 р., м. Львів);
\item
  Шостій Міжнародній науковій конференції (26-29 травня 2003 р., м.
  Львів) ``Математичні проблеми механіки неоднорідних структур'';
\item
  Семінар інституту прикладних проблем механіки та математики ім. Я.
  Підстригала
\end{enumerate}

У повному обсязі робота доповідалася на семінарі кафедри програмування
львівського національного університету імені Івана Франка.

\emph{\textbf{Публікації.}} Результати досліджень висвітлені в 4 статтях
у журналах з переліку фахових видань ВАК України, в тому числі у
доповідях НАН України.

\emph{\textbf{Особистий внесок здобувача.}} Всі результати дисертаційної
роботи отримано здобувачем самостійно.

\emph{\textbf{Структура та обсяг роботи. }}Дисертація складається із
вступу і шести розділів основної частини, висновків, списку використаної
літератури із 37 найменувань.

У першому розділі описано підходи до розв'язування задач термопружності
і підхід, який був вибраний у даній роботи. Проаналізовано переваги і
недоліки кожного із запропонованих методів.

Другий розділ містить опис двох підходів для розв'язування задачі
теплопровідності. Перший підхід базується на ідеї предиктор-коректор, а
другий -- на рівняннях нев'язки у точках областей локальної
неоднорідності коефіцієнта теплопровідності.

Метод адитивного розчеплення оператора застосований для задачі пружності
для локально-неоднорідного тіла запропонований у третьому розділі. Даний
підхід був апробований для двох методів МГЕ і МПГЕ. Наведено порівняння
чисельних результатів для цих двох методів, а також порівняння методів
по інших критеріях, таких як простота реалізації, точність, діагональне
переважання матриць.

Четвертий розділ містить порівняльний аналіз МГЕ і МПГЕ при
розв'язуванні задачі пружності у кусково-однорідному тілі. Наводяться
рекомендації щодо застосування даних підходів для розв'язування задач
такого класу.

У п'ятому розділі наведено підхід на основі МГЕ і МПГЕ для розв'язування
задачі термопружності у тілі із локально-неоднорідними властивостями
фізичних параметрів.

Програмне середовище Aspirant's Calculations Envinronment (ACE) було
створене для простого і зручного програмування запропонованих методів.
Структура даного середовища описана у шостому розділі. Також дано
рекомендації відносно використання даного середовища для програмування
подібних методів, яке вимагає невеликих зусиль спрямованих тільки на
побудову алгоритму, але не на організацію взаємодії між об'єктами.

\emph{\textbf{Усталені домовленості. }}У цій праці кома перед індексом
означає похідну по змінній, яка відповідає цьому індексу, тобто
\textsubscript{ }\[{{f_{,j} = \partial}{f/\partial}x_{j},}{}\]
\[{{f_{,\mathit{\text{ij}}} = \partial^{2}}{f/\partial}x_{i}\partial x_{j}}{}\],
при цьому викори­стовується німе сумування за індек­сами, які
повторюються.

Розділ 1

\hypertarget{ux43cux435ux442ux43eux434ux438ux43aux430-ux43fux43eux454ux434ux43dux430ux43dux43dux44f-ux433ux440ux430ux43dux438ux447ux43dux438ux445-ux456-ux441ux43aux456ux43dux447ux435ux43dux438ux445-ux435ux43bux435ux43cux435ux43dux442ux456ux432}{%
\section[Методика поєднання граничних і скінчених елементів
]{\texorpdfstring{\protect\hypertarget{anchor-12}{}{}Методика поєднання
граничних і скінчених елементів
}{Методика поєднання граничних і скінчених елементів }}\label{ux43cux435ux442ux43eux434ux438ux43aux430-ux43fux43eux454ux434ux43dux430ux43dux43dux44f-ux433ux440ux430ux43dux438ux447ux43dux438ux445-ux456-ux441ux43aux456ux43dux447ux435ux43dux438ux445-ux435ux43bux435ux43cux435ux43dux442ux456ux432}}

\hypertarget{ux43eux433ux43bux44fux434-ux43cux435ux442ux43eux434ux456ux432-ux440ux43eux437ux432ux44fux437ux443ux432ux430ux43dux43dux44f-ux437ux430ux434ux430ux447-ux442ux435ux440ux43cux43eux43fux440ux443ux436ux43dux43eux441ux442ux456-ux443-ux43aux43eux43cux43fux43eux437ux438ux442ux430ux445}{%
\subsection[1.1. Огляд методів розв'язування задач термопружності у
композитах]{\texorpdfstring{\protect\hypertarget{anchor-13}{}{}1.1.
Огляд методів розв'язування задач термопружності у
композитах}{1.1. Огляд методів розв'язування задач термопружності у композитах}}\label{ux43eux433ux43bux44fux434-ux43cux435ux442ux43eux434ux456ux432-ux440ux43eux437ux432ux44fux437ux443ux432ux430ux43dux43dux44f-ux437ux430ux434ux430ux447-ux442ux435ux440ux43cux43eux43fux440ux443ux436ux43dux43eux441ux442ux456-ux443-ux43aux43eux43cux43fux43eux437ux438ux442ux430ux445}}

\protect\hypertarget{anchor-14}{}{}Існують різні підходи, які найбільш
часто використовують при дослідженні напружено-деформованого стану тіл.
Коротко зупинимось на деяких з них. З допомогою методу збурення форми
границі {[}\protect\hyperlink{anchor-15}{27}{]} на основі загальних
розв'язків задач теорії пружності для сфери, циліндра і еліпсоїда
обертання одержані наближені розв'язки задач для ізотропного середовища
з включеннями, гранична поверхня яких близька до вказаних. Головною
перевагою цього методу є те, що розв'язок задачі одержується в
замкнутому вигляді, а істотним недоліком те, що для побудови розв'язку
форма поверхні кусково-однорідного тіла і включень має вирішальне
значення. Тому клас задач, які розв'язуються цим методом, обмежений
тілами, граничні поверхні та поверхні розділу середовищ яких є
поверхнями другого порядку та близькими до них, зокрема, в
{[}\protect\hyperlink{anchor-16}{28}{]} розглянута для тривимірного
ізотропного пружного тіла з еліпсоїдальним включенням під дією
одновісного розтягу пружна задача, яка зведена до розв'язування системи
трьох інтегро-диференційних рівнянь відносно стрибків напружень і
переміщень на поверхнях включення, знайдено точний розв'язок цих рівнянь
і виписано формули для наближеного обчислення напружень у матриці і
включенні.

Однак при розв'язуванні багатьох практично важливих задач теорії
пружності потрібно розглядати тіла, поверхні контакту в яких мають
найрізноманітнішу форму. В зв'язку з цим широкого застосування набув
метод R-функцій {[}\protect\hyperlink{anchor-17}{29}{]}, який дає
можливість точно задовольняти граничні умови та умови спряження на
поверхнях контакту, але вимагає розвинутого спеціального математичного
та програмного забезпечення.

У багатьох роботах автори використовують деякі допущення про характер
контакту, зокрема в {[}\protect\hyperlink{anchor-18}{30}{]} розглянуто
задачі про контактну взаємодію двох скінченних стрингерів (накладок) з
різних пружних матеріалів із лінійно-деформованими основами (плоска
деформація) у вигляді пружної нескінченної смуги й анізотропної
півплощини в допущенні, що один із стрингерів перебуває в ідеальному
контакті з основою, а інший контактує з нею через тонкий шар клею.
Задачі зводяться до систем інтегро-алгебричних рівнянь другого роду. У
роботі {[}\protect\hyperlink{anchor-19}{31}{]} за допомогою аналітичних
комплекс­них потенціалів побудовано алгоритм розв'язування плоскої
задачі теорії пружності для трьохкомпонентної композиції, в якій границя
розділу матриці і включення моделюється пружним прошарком, а також дано
кількісну оцінку залежності узагальнених коефіцієнтів концентрації
напружень біля вершини криволі­нійного вирізу і абсолютно жосткого
включення від типу навантаження, пружних характеристик і ширини пружного
прошарку.

Оскільки аналітичний розв'язок багатьох практично важливих задач
(наприклад, для тіл з вирізами і неоднорідними включеннями) часто
утруднений, широкого використання набули числові методи, зокрема,
найбільш повно вивчені методи скінченних різниць
{[}\protect\hyperlink{anchor-20}{32}{]} та скінченних елементів. Вони
застосовуються здебільшого до плоских задач для розрахунку
напружено-деформованого стану пластин, оболонок, осесиметричних та
кусково-однорідних тіл.

Все більшої популярності при розв'язуванні задач набувають МГЕ
{[}\protect\hyperlink{anchor-2}{2}{]}, які ґрунтуються на ідеях,
розроблених С.Г.Міхліним, Н.І.Мусхелішвілі, В.Д.Купрадзе. Перевагами
методу граничних елементів (МГЕ), зокрема, є зниження на одиницю
розмірності задачі внаслідок дискретизації тільки граничної поверхні,
інтегральне зображення розв'язку є кращим, ніж числове диференціювання в
МСЕ, значно простіше досліджуються задачі для нескінчених областей.
Недоліком МГЕ є неможливість розгляду неперервно-неоднорідних тіл, а
також ускладнення алгоритму внаслідок обчислення сингулярних інтегралів.
Порівняння переваг різних методів привело до ідеї доцільності їх
поєднання і до створення на їх основі ефективних числово-аналітичних
методик, які дозволяють оптимізувати процес розв'язування {[}6{]}. У
роботі {[}\protect\hyperlink{anchor-21}{33}{]} побудовано об'єднаний
підхід МГЕ та МСЕ для розв'язання двовимірної статичної задачі теорії
пружності, що дозволяє звести вихідну задачу до послідовності розв'язків
підзадач у підобластях кожним із методів окремо, та розглянуто задачу
про плоску деформацію жорстко защемленого на одній із поверхонь шару під
рівномірно розподіленим навантеженням.

Одним з ефективних методів розрахунку напружень в кусково-однорідних
тілах є підхід, який ґрунтується на математичній постановці відповідних
узагальнених задач спряження, коли на основі частково вироджених
диференціальних рівнянь квазістатичної задачі термопружності для тіла з
чужорідними включеннями отримуються граничні інтегральні рівняння, єдині
для всієї розглядуваної області, які розв'язуються за допомогою прямого
МГЕ, при цьому враховуються умови ідеального та неідеального контакту
{[}\protect\hyperlink{anchor-22}{34},
\protect\hyperlink{anchor-23}{35}{]}.

МПГЕ {[}\protect\hyperlink{anchor-24}{14}{]}, який виник під впливом
методів граничних інтегральних рівнянь, проекційно-сіткових, занурення,
граничних і скінченних елементів та теорії синтезу математичних
алгоритмів, можна розглядати як один з варіантів методу джерела і
віднести його до непрямих методів досліджень, оскільки введені для
одержання розв'язку задачі невідомі не є фізичними змінними. У роботі
{[}\protect\hyperlink{anchor-25}{36},
\protect\hyperlink{anchor-26}{37}{]} зроблено порівняльний аналіз
теоретичних та обчислювальних особливостей МГЕ та МПГЕ для другої
основної задачі пружності та досліджено точність отриманих числових
результатів.

У даній роботі на основі підходів, запропонованих в
{[}\protect\hyperlink{anchor-2}{2}, \protect\hyperlink{anchor-19}{31},
\protect\hyperlink{anchor-24}{14}{]} для розв'язу­вання змішаної
статичної задачі теорії пружності неоднорідних тіл, змодельовано
напружено-деформований стан об'єктів та порів­няно точ­ність отриманих
МПГЕ і МГЕ розв'язків.

\hypertarget{ux441ux443ux442ux44c-ux43cux435ux442ux43eux434ux443-ux43fux43eux454ux434ux43dux430ux43dux43dux44f-ux433ux440ux430ux43dux438ux447ux43dux438ux445-ux456-ux43fux440ux438ux433ux440ux430ux43dux438ux447ux43dux438ux445-ux435ux43bux435ux43cux435ux43dux442ux456ux432}{%
\subsection[1.2. Суть методу поєднання граничних і приграничних
елементів]{\texorpdfstring{\protect\hypertarget{anchor-27}{}{}1.2. Суть
методу поєднання граничних і приграничних
елементів}{1.2. Суть методу поєднання граничних і приграничних елементів}}\label{ux441ux443ux442ux44c-ux43cux435ux442ux43eux434ux443-ux43fux43eux454ux434ux43dux430ux43dux43dux44f-ux433ux440ux430ux43dux438ux447ux43dux438ux445-ux456-ux43fux440ux438ux433ux440ux430ux43dux438ux447ux43dux438ux445-ux435ux43bux435ux43cux435ux43dux442ux456ux432}}

Одним із недоліків МГЕ є неможливість його застосування у тілах із
локальними неоднорідностями. Для розв'язання цієї проблеми був
запропонований підхід {[}\protect\hyperlink{anchor-5}{7}{]}. Основна
ідея даного підходу полягає у розчеплення оператора задачі на дві
частини, одна із них представляє оператор для однорідного тіла, для якої
МГЕ застосовний, а друга частина містить невідому функцій і
переміщується у праву частину рівняння. Ми отримуємо рівняння, яке за
своїм виглядом нагадує рівняння Пуасона, проте права частина містить
функцію залежну від шуканої функції. Для розв'язування даної проблеми
застосовуються різні підходи, а саме: а) метод пре диктор-коректор б)
метод нев'язок

У наступних розділах подані використання зазначених методик для задач
теплопровідності, пружності і термопружності у тілах із локальними
неоднорідностями.

Розділ 2

\hypertarget{ux437ux430ux434ux430ux447ux430-ux442ux435ux43fux43bux43eux43fux440ux43eux432ux456ux434ux43dux43eux441ux442ux456-ux434ux43bux44f-ux43fux43bux43eux441ux43aux43eux433ux43e-ux442ux456ux43bux430-ux456ux437-ux43bux43eux43aux430ux43bux44cux43dux43e-ux43dux435ux43eux434ux43dux43eux440ux456ux434ux43dux438ux43c-ux432ux43aux43bux44eux447ux435ux43dux43dux44fux43c}{%
\section[Задача теплопровідності для плоского тіла із
локально-неоднорідним
включенням]{\texorpdfstring{\protect\hypertarget{anchor-28}{}{}Задача
теплопровідності для плоского тіла із локально-неоднорідним
включенням}{Задача теплопровідності для плоского тіла із локально-неоднорідним включенням}}\label{ux437ux430ux434ux430ux447ux430-ux442ux435ux43fux43bux43eux43fux440ux43eux432ux456ux434ux43dux43eux441ux442ux456-ux434ux43bux44f-ux43fux43bux43eux441ux43aux43eux433ux43e-ux442ux456ux43bux430-ux456ux437-ux43bux43eux43aux430ux43bux44cux43dux43e-ux43dux435ux43eux434ux43dux43eux440ux456ux434ux43dux438ux43c-ux432ux43aux43bux44eux447ux435ux43dux43dux44fux43c}}

\hypertarget{ux43fux43eux441ux442ux430ux43dux43eux432ux43aux430-ux437ux430ux434ux430ux447ux456}{%
\subsection[2.1. Постановка
задачі]{\texorpdfstring{\protect\hypertarget{anchor-29}{}{}2.1.
Постановка
задачі}{2.1. Постановка задачі}}\label{ux43fux43eux441ux442ux430ux43dux43eux432ux43aux430-ux437ux430ux434ux430ux447ux456}}

Нехай характеристика середовища \[\] стосовно певного фізичного процесу,
модельованого потенціалом безвихрового векторного поля,
(теплопровідності, дифузії, електромагнітостатики, пружного руху
твердого тіла, течією ґрунтових вод тощо) є неперервною функцією
декартових координат
\includegraphics[width=0.5717in,height=0.2354in]{./ObjectReplacements/Obj103},
сталою скрізь у плоскій однозв'язній області \[\Omega{}\] з простим
замкнутим краєм \[{\Gamma = {\Gamma_{1} \cup \Gamma_{2}}}{}\], за
винятком скінченої сукупності локальних неоднорідних включень
\[{\Omega_{k}\subset\Omega}{}\] (\[\], \[\],
\[{k,{l = \overline{1,K}}}{}\]).

Розглянемо змішану крайову задачу для основного рівняння стаціонарного
поля в локально-неоднорідному середовищі

\[\] \[\](2.1)

\[\] \[{z\in\Gamma_{1},}{}\]\[\] \[{z\in\Gamma_{2},}{}\](2.2)

де

\[\] \[\],(2.3)

\includegraphics[width=0.9571in,height=0.2071in]{./ObjectReplacements/Obj118};\includegraphics[width=0.3783in,height=0.2354in]{./ObjectReplacements/Obj119}
-- якісна характеристика провідності середовища в
\[\Omega_{k}{}\]\includegraphics[width=0.5854in,height=0.2929in]{./ObjectReplacements/Obj121},
причому \[{\lambda_{k}{\left( z \right) \geq 0}}{}\],
\[{\underset{\Omega_{k}}{\text{max}}\lambda_{k}\left( z \right){= 1}}{}\],
\[{\lambda_{k}\left( z \right){_{z\in\partial\Omega_{k}} = 0}}{}\];
\[{\chi_{k}\left( z \right)}{}\] -- характеристична функція області
\[\Omega_{k}{}\].

Враховуючи (2.3), задача (2.1)(2.2) набуде вигляду:

\includegraphics[width=1.25in,height=0.4in]{./ObjectReplacements/Obj127}
в
\includegraphics[width=0.1646in,height=0.15in]{./ObjectReplacements/Obj128},(2.4)

\includegraphics[width=0.3646in,height=0.1929in]{./ObjectReplacements/Obj129}
на
\includegraphics[width=0.1646in,height=0.2071in]{./ObjectReplacements/Obj130};\includegraphics[width=0.5283in,height=0.3854in]{./ObjectReplacements/Obj131}
на
\includegraphics[width=0.1783in,height=0.2071in]{./ObjectReplacements/Obj132},(2.5)

де
\includegraphics[width=1.0146in,height=0.4283in]{./ObjectReplacements/Obj133},
\includegraphics[width=0.15in,height=0.1646in]{./ObjectReplacements/Obj134}
- оператор Гамільтона.

\hypertarget{ux43cux435ux442ux43eux434-ux43fux440ux435ux434ux438ux43aux442ux43eux440-ux43aux43eux440ux435ux43aux442ux43eux440}{%
\subsection[2.2. Метод
предиктор-коректор]{\texorpdfstring{\protect\hypertarget{anchor-30}{}{}2.2.
Метод
предиктор-коректор}{2.2. Метод предиктор-коректор}}\label{ux43cux435ux442ux43eux434-ux43fux440ux435ux434ux438ux43aux442ux43eux440-ux43aux43eux440ux435ux43aux442ux43eux440}}

\hypertarget{ux441ux443ux442ux44c-ux43cux435ux442ux43eux434ux443}{%
\subsubsection[2.2.1. Суть
методу]{\texorpdfstring{\protect\hypertarget{anchor-31}{}{}2.2.1. Суть
методу}{2.2.1. Суть методу}}\label{ux441ux443ux442ux44c-ux43cux435ux442ux43eux434ux443}}

Розглянемо послідовність плоских змішаних крайових задач для основного
рівняння стаціонарного поля

\[{\mathit{\text{div}}{\left( {\Lambda_{n}\left( z \right)\mathit{\text{grad}}\phi_{n}\left( z \right)} \right) = 0},}{}\]
\[{z\in\Omega}{}\];(2.6)

\[{\phi_{n}{\left( z \right) = \phi_{s}}\left( z \right),}{}\]
\[{z\in\Gamma_{1}}{}\];(2.7)

\[{{\frac{\partial\phi_{n}\left( z \right)}{\partial\overrightarrow{n}} = {- \psi_{s}}}\left( z \right),}{}\]
\[{z\in\Gamma_{2}}{}\](2.8)

за умови, що характеристика середовища
\[{\Lambda_{n}(z)({n = \overline{0,N}})}{}\] (коефіцієнт
теплопровідності, коефіцієнт дифузії тощо) для кожної з цих задач є
неперервною функцією декартових координат
\[{z = \left( {x,y} \right)}{}\], сталою скрізь в однозв'язній області
\[\Omega{}\] з простим замкнутим краєм
\[{\Gamma = {\Gamma_{1} \cup \Gamma_{2}}}{}\], за винятком скінченої
сукупності неоднорідних включень \[{\Omega_{k}\subset\Omega}{}\]
(\[{\Omega_{k}\underset{k \neq l}{\mathit{intersect}}{\Omega_{l} = \varnothing}}{}\],
\[{\partial\Omega_{k}\mathit{intersect}{\Gamma =}}{}\],
\[{k,{l = \overline{1,K}}}{}\]), геометрія та конфігурація розташування
яких фіксовані для всіх \[n{}\]одночасно. А саме:

\[{\Lambda_{n}{\left( z \right) = {1 + {\sum\limits_{k = 1}^{K}{a_{\mathit{\text{nk}}}\lambda_{k}\left( z \right)\chi_{k}\left( z \right)}}}},}{}\]
\[{z\in\Omega,}{}\](2.9)

де \[{\lambda_{k}\left( z \right)}{}\] -- якісна характеристика
провідності середовища в області \[\Omega_{k}{}\]
(\[{k = \overline{1,K}}{}\]), причому
\[{\lambda_{k}{\left( z \right) \geq 0}}{}\],
\[{\underset{\Omega_{k}}{\text{max}}\lambda_{k}\left( z \right){= 1}}{}\],
\[{\lambda_{k}\left( z \right){_{z\in\partial\Omega_{k}} = 0}}{}\];
\[{\chi_{k}\left( z \right)}{}\] -- характеристична функція області
\[\Omega_{k}{}\]; \[\left\{ a_{\mathit{\text{nk}}} \right\}{}\] --
монотонна послідовність (\[{n = \overline{0,N}}{}\]) для кожного
конкретного \[k{}\], така, що \[{a_{0k} = 0}{}\],
\[{a_{\mathit{\text{nk}}} > {- 1}}{}\].

Застосуємо до оператора основного рівняння стаціонарного поля (2.6),
враховуючи (2.9), операції методу адитивного розщеплення операторів.
Внаслідок цього (2.6) набуде вигляду

\[{\nabla^{2}{\phi_{n} = P_{n}}\left\lbrack \phi_{n} \right\rbrack}{}\]
в \[\Omega{}\],(2.10)

де
\[{{\nabla = \frac{\partial}{\partial x}}{\overrightarrow{i} + \frac{\partial}{\partial y}}\overrightarrow{j}}{}\]
-- оператор Гамільтона;
\[{P_{n}{\left\lbrack {} \right\rbrack = {- \Lambda_{n}^{- 1}}}{\sum\limits_{k = 1}^{K}{a_{\mathit{\text{nk}}}\nabla\lambda_{k}\nabla\left\lbrack {} \right\rbrack}}\chi_{k}}{}\]
-- оператор, який враховує додатковий вплив неоднорідностей.

Отже, маємо послідовність змішаних задач (2.10), (2.7), (2.8) для
визначення розв'язків \[{\phi_{n}\left( z \right)}{}\]
(\[{n = \overline{0,N}}{}\]).

\hypertarget{ux43fux43eux431ux443ux434ux43eux432ux430-ux43fux440ux43eux446ux435ux434ux443ux440ux438-ux442ux438ux43fux443-ux43fux440ux43eux433ux43dux43eux437ux43aux43eux440ux435ux43aux442ux443ux432ux430ux43dux43dux44f}{%
\subsubsection[2.2.2. Побудова процедури типу
прогноз--коректування]{\texorpdfstring{\protect\hypertarget{anchor-32}{}{}\protect\hypertarget{anchor-33}{}{}2.2.2.
Побудова процедури типу
прогноз--коректування}{2.2.2. Побудова процедури типу прогноз--коректування}}\label{ux43fux43eux431ux443ux434ux43eux432ux430-ux43fux440ux43eux446ux435ux434ux443ux440ux438-ux442ux438ux43fux443-ux43fux440ux43eux433ux43dux43eux437ux43aux43eux440ux435ux43aux442ux443ux432ux430ux43dux43dux44f}}

Для вибраної конкретної\textbf{ }геометрії неоднорідних включень та
конкретної конфігурації їхнього розташування в \[\Omega{}\] побудуємо
єдину процедуру, яка дає змогу розв'язувати відразу цілу низку задач
типу (2.10), (2.7), (2.8), що відрізняються характеристиками середовища
\[\Lambda_{n}{}\](\[{n = \overline{0,N}}{}\]). У цьому разі внаслідок
послідовного застосування аналогічно до прийому попереднього моделювання
шуканого розв'язку у правій частині (2.10) перейдемо до задач відносно
\[{{\phi_{n}^{} = \phi_{n}^{}}\left( z \right)}{}\].

Ітерації пропонованої процедури для визначення послідовності функцій
\[\phi_{i}^{}{}\]задамо такими формулами:

\[{\nabla^{2}{\phi_{i}^{{}^{m}} = P_{i}}\left\lbrack {\hat{\phi}}_{i^{m - 1}} \right\rbrack}{}\]
в \[\Omega{}\];(2.11)

\[{\phi_{i}^{{}^{m}} = \phi_{s}}{}\] на \[\Gamma_{1}{}\];(2.12)

\[{\frac{\partial\phi_{i}^{{}^{m}}}{\partial\overrightarrow{n}} = {- \psi_{s}}}{}\]
на \[\Gamma_{2}{}\];(2.13)

\[{i = \overline{0,N}}{}\]; \[{m = \overline{1,M_{i}}}{}\],

де \[{\hat{\phi}}_{i}^{m - 1}{}\]-- апроксимація чергової ітерації
розв'язку \[\phi_{i}^{{}^{m - 1}}{}\]задачі при характеристиці
середовища \[\Lambda_{i}{}\], причому
\[{{\hat{\phi}}_{i + 1}^{0} = {\hat{\phi}}_{i}^{M_{i}}}{}\];
\[\]\[{\hat{\phi}}_{i}^{M_{i}}{}\]-- апроксимація \emph{відкоректованого
}\[M_{i}{}\] ітераціями розв'язку задачі при характеристиці середовища
\[\Lambda_{i}{}\]; \[{\hat{\phi}}_{i + 1}^{0}{}\] -- апроксимація
початкового наближення \emph{прогнозованого }розв'язку при
\[\Lambda_{i + 1}{}\]; \[{\hat{\phi}}_{0}^{0}{}\] -- апроксимація
розв'язку наступної задачі для однорідної області

\[{\nabla^{2}{\phi_{0} = 0}}{}\] в \[\Omega{}\];

\[{\phi_{0} = \phi_{s}}{}\] на \[\Gamma_{1}{}\];
\[{\frac{\partial\phi_{0}}{\partial\overrightarrow{n}} = {- \psi_{s}}}{}\]
на \[\Gamma_{2}{}\];

Розв'язок кожної із задач (2.11) -- (2.13) зобразимо у вигляді

\[{\phi_{i}^{{}^{m}}{\left( z \right) = \varphi_{i}^{m}}{\left( z \right) - \Phi_{i}^{m}}\left( z \right)}{}\],

де

\[{\Phi_{i}^{m}{\left( z \right) = \frac{1}{2\pi}}{\sum\limits_{k = 1}^{K}{a_{\mathit{\text{ik}}}{\int\limits_{\Omega_{k}}{\text{ln}\left( {\left( {x - x^{}} \right)^{2} + \left( {y - y^{}} \right)^{2}} \right)}}}^{\frac{- 1}{2}}}{\Lambda_{i}^{- 1}\left( z^{} \right)}{\nabla\lambda_{k}\left( z^{} \right)}{\nabla{\hat{\phi}}_{i}^{m - 1}\left( z^{} \right)}{d\Omega_{k}\left( z^{} \right)}}{}\]--
частковий розв'язок рівняння (2.11);
\[{\varphi_{i}^{m}\left( z \right)}{}\] -- розв'язок наступної змішаної
задачі для рівняння Лапласа

\[{\nabla^{2}{\varphi_{i}^{m} = 0}}{}\] в \[\Omega{}\];

\[{\varphi_{i}^{m} = {\phi_{s} + \Phi_{i}^{m}}}{}\] на \[\Gamma_{1}{}\];
\[{\frac{\partial\psi_{i}^{m}}{\partial\overrightarrow{s}} = {{- \psi_{s}} + \frac{\partial\Phi_{i}^{m}}{\partial\overrightarrow{n}}}}{}\]
на \[\Gamma_{2}{}\],

який легко можна знайти, використовуючи інтегральну формулу Коші для
функції
\[{\omega_{i}^{m}{\left( z \right) = \varphi_{i}^{m}}{\left( z \right) + {i \cdot \psi_{i}^{m}}}\left( z \right)}{}\]
комплексної змінної \[z{}\]( \[{\varphi_{i}^{m}\left( z \right)}{}\] --
потенціал; \[{\psi_{i}^{m}\left( z \right)}{}\] -- функція потоку;
\[\overrightarrow{s}{}\] -- одиничний вектор, дотичний до
\[\Gamma_{2}{}\] ) згідно з прямим чи непрямим формулюванням алгоритму
КМГЕ.

\hypertarget{ux447ux438ux441ux43bux43eux432ux456-ux440ux435ux437ux443ux43bux44cux442ux430ux442ux438}{%
\subsubsection[2.2.3. Числові
результати]{\texorpdfstring{\protect\hypertarget{anchor-34}{}{}2.2.3.
Числові
результати}{2.2.3. Числові результати}}\label{ux447ux438ux441ux43bux43eux432ux456-ux440ux435ux437ux443ux43bux44cux442ux430ux442ux438}}

Об'єктом числових досліджень було вибрано послідовність змішаних
крайових задач теплопровідності типу (2.6) -- (2.9) в області
\[{\Omega = {\left( {x^{-};x^{+}} \right) \times \left( {y^{-};y^{+}} \right)}}{}\]
з краєм
\includegraphics[width=0.7646in,height=0.2217in]{./ObjectReplacements/Obj217},
де\[\]\textbf{; }\[\]\textbf{;}

\[\]\textbf{; }\[\]\textbf{ , }яка містить область локальної
неоднорідності \[\].

Граничні умови, схарактеризовані так: \[\]\textbf{; }\[\]\textbf{.}

Якісну характеристику теплопровідності середовища було задано у вигляді

\[\]\textbf{;
}\includegraphics[width=0.5429in,height=0.2217in]{./ObjectReplacements/Obj226}\textbf{.}

Усі графіки, показані на рис. 2.1--2.3, стосуються таких конкретних
значень параметрів: \[\]; \[\]; \[\]; \[\]; \[\]

\includegraphics[width=5.111in,height=3.528in]{Pictures/100000000000020200000163358C6F063C52CE42.png}\includegraphics[width=5.111in,height=3.25in]{Pictures/10000000000001EB0000013832389E2C121C29E6.png}

Рис. 2.1

На рис. 2.1 наочно простежується залежність температурного поля (рис.
2.1,а) від властивостей середовища (рис. 2.1,б). Видно, що зі
збільшенням відстані до області локальної неоднорідності
(\includegraphics[width=0.5283in,height=0.2217in]{./ObjectReplacements/Obj232})
зменшується її вплив на температурне поле, тобто розв'язок задачі щораз
менше відрізняється від розв'язку аналогічної задачі для випадку
однорідного середовища.

\includegraphics[width=5.111in,height=2.889in]{Pictures/100000000000031C000001C112ECC8EDD043C886.png}

Рис. 2.2

\includegraphics[width=5.111in,height=2.889in]{Pictures/100000000000031C000001C1A825D2F7BF570EC8.png}

Рис. 2.3

Зображені на рис. 2.2, рис. 2.3 графіки демонструють зміну розв'язків
крайових задач (2.6)--(2.9) уздовж осі
\includegraphics[width=0.4in,height=0.2217in]{./ObjectReplacements/Obj233},
що проходить через центр області локальної неоднорідності, де
найяскравіше виявляються закономірності впливу теплофізичних
властивостей середовища на розв'язок.

Крива 1 на рис. 2.2 -- розв'язок задачі без неоднорідності; криві 2--5
-- розв'язки задач за наявної неоднорідності радіуса
\includegraphics[width=0.5283in,height=0.2217in]{./ObjectReplacements/Obj234}
і максимальних відхиленнях у ній значень коефіцієнта теплопровідності
від значення у решті області, відповідно, у 3, 5, 7, 9 разів. Криві 1--5
добре демонструють збіжність послідовності розв'язків крайових задач
(2.6)--(2.9)
(\includegraphics[width=0.4854in,height=0.2354in]{./ObjectReplacements/Obj235}),
а також щоразу яскравіший прояв із зростанням провідності ефекту
``плато''.

Графіки на рис. 2.3 одержані при
\includegraphics[width=0.3646in,height=0.1783in]{./ObjectReplacements/Obj236}
і значеннях
\includegraphics[width=1.0429in,height=0.2217in]{./ObjectReplacements/Obj237}
(відповідно, криві 2--4). Крива 1 відповідає розв'язку задачі для
однорідного середовища. Можемо спостерігати розширення сфери впливу
неоднорідності на розв'язок із збільшенням розмірів неоднорідності.

\hypertarget{ux43cux435ux442ux43eux434-ux43fux43eux454ux434ux43dux430ux43dux43dux44f-ux433ux440ux430ux43dux438ux447ux43dux438ux445-ux435ux43bux435ux43cux435ux43dux442ux456ux432-ux456ux437-ux441ux43aux456ux43dux447ux435ux43dux438ux43cux438-ux435ux43bux435ux43cux435ux43dux442ux430ux43cux438}{%
\subsection[2.3. Метод поєднання граничних елементів із скінченими
елементами]{\texorpdfstring{\protect\hypertarget{anchor-35}{}{}2.3.
Метод поєднання граничних елементів із скінченими
елементами}{2.3. Метод поєднання граничних елементів із скінченими елементами}}\label{ux43cux435ux442ux43eux434-ux43fux43eux454ux434ux43dux430ux43dux43dux44f-ux433ux440ux430ux43dux438ux447ux43dux438ux445-ux435ux43bux435ux43cux435ux43dux442ux456ux432-ux456ux437-ux441ux43aux456ux43dux447ux435ux43dux438ux43cux438-ux435ux43bux435ux43cux435ux43dux442ux430ux43cux438}}

\hypertarget{ux441ux443ux442ux44c-ux43cux435ux442ux43eux434ux443-1}{%
\subsubsection[2.3.1. Суть
методу]{\texorpdfstring{\protect\hypertarget{anchor-36}{}{}2.3.1. Суть
методу}{2.3.1. Суть методу}}\label{ux441ux443ux442ux44c-ux43cux435ux442ux43eux434ux443-1}}

Припустимо, що нам відома апроксимація
\includegraphics[width=0.1283in,height=0.2071in]{./ObjectReplacements/Obj238}
функції розв'язку
\includegraphics[width=0.1283in,height=0.1646in]{./ObjectReplacements/Obj239}
задачі (2.4),(2.5) в області кожної з локальних неоднорідностей
\includegraphics[width=0.2071in,height=0.2071in]{./ObjectReplacements/Obj240}.
Використаємо
\includegraphics[width=0.1283in,height=0.2071in]{./ObjectReplacements/Obj241}
у правій частині (2.4). Потрактуємо розв'язок
\includegraphics[width=0.1354in,height=0.1646in]{./ObjectReplacements/Obj242}
одержаної крайової задачі

\includegraphics[width=1.2929in,height=0.4in]{./ObjectReplacements/Obj243}
в
\includegraphics[width=0.1646in,height=0.15in]{./ObjectReplacements/Obj244};(2.14)

\includegraphics[width=0.3783in,height=0.1929in]{./ObjectReplacements/Obj245}
на
\includegraphics[width=0.1646in,height=0.2071in]{./ObjectReplacements/Obj246};
\includegraphics[width=0.5429in,height=0.3854in]{./ObjectReplacements/Obj247}
на
\includegraphics[width=0.1783in,height=0.2071in]{./ObjectReplacements/Obj248},(2.15)

як аналог розв'язку вихідної задачі.

Далі розв'язок задачі (2.14),(2.15) зобразимо у вигляді

\includegraphics[width=1.5in,height=0.4in]{./ObjectReplacements/Obj249},(2.16)

де

\includegraphics[width=2.1783in,height=0.4429in]{./ObjectReplacements/Obj250},(2.17)

а
\includegraphics[width=0.3217in,height=0.2354in]{./ObjectReplacements/Obj251}
- розв'язок задачі для рівняння Лапласа з видозміненими крайовими
умовами

\includegraphics[width=0.4854in,height=0.1929in]{./ObjectReplacements/Obj252}
в
\includegraphics[width=0.1646in,height=0.15in]{./ObjectReplacements/Obj253};(2.18)

\includegraphics[width=1.1146in,height=0.35in]{./ObjectReplacements/Obj254}
на
\includegraphics[width=0.1646in,height=0.2071in]{./ObjectReplacements/Obj255};
\includegraphics[width=1.2929in,height=0.4in]{./ObjectReplacements/Obj256}
на
\includegraphics[width=0.1783in,height=0.2071in]{./ObjectReplacements/Obj257},(2.19)

де
\includegraphics[width=1.4146in,height=0.4283in]{./ObjectReplacements/Obj258}.

Шуканий розв'язок задачі (2.18), (2.19) проінтерпретуємо наступним чином
\includegraphics[width=0.5429in,height=0.1646in]{./ObjectReplacements/Obj259},
де
\includegraphics[width=1.4in,height=0.2354in]{./ObjectReplacements/Obj260}
- аналітична в
\includegraphics[width=0.3783in,height=0.1929in]{./ObjectReplacements/Obj261}
функція комплексної змінної
\includegraphics[width=0.5717in,height=0.1929in]{./ObjectReplacements/Obj262},
причому потенціал
\includegraphics[width=0.1283in,height=0.1354in]{./ObjectReplacements/Obj263}
та функція потоку
\includegraphics[width=0.1146in,height=0.1354in]{./ObjectReplacements/Obj264}
є гармонійними в
\includegraphics[width=0.3783in,height=0.1929in]{./ObjectReplacements/Obj265}.

Беручи до уваги, що
\includegraphics[width=0.5146in,height=0.3854in]{./ObjectReplacements/Obj266},
де
\includegraphics[width=0.1283in,height=0.1783in]{./ObjectReplacements/Obj267}
- одиничний додатньоорієнтований (проти руху стрілок годинника) вектор,
дотичний до
\includegraphics[width=0.1354in,height=0.15in]{./ObjectReplacements/Obj268},
перейдемо до розв'язування такої крайової задачі:

\includegraphics[width=0.6783in,height=0.2354in]{./ObjectReplacements/Obj269}
в
\includegraphics[width=0.1646in,height=0.15in]{./ObjectReplacements/Obj270},(2.20)

\includegraphics[width=1.5in,height=0.35in]{./ObjectReplacements/Obj271}
на
\includegraphics[width=0.1646in,height=0.2071in]{./ObjectReplacements/Obj272};(2.21)

\includegraphics[width=1.6646in,height=0.4429in]{./ObjectReplacements/Obj273}
на
\includegraphics[width=0.1783in,height=0.2071in]{./ObjectReplacements/Obj274}.(2.22)

Для довільної точки
\includegraphics[width=0.1283in,height=0.1283in]{./ObjectReplacements/Obj275}
заданої однозв'язної області
\includegraphics[width=0.1646in,height=0.15in]{./ObjectReplacements/Obj276}
з простим замкнутим додатньоорієнтованим краєм
\includegraphics[width=0.1354in,height=0.15in]{./ObjectReplacements/Obj277}
справедлива інтегральна формула Коші

\includegraphics[width=1.3783in,height=0.4283in]{./ObjectReplacements/Obj278},(2.23)

яка слугуватиме основою наступних викладок.

\hypertarget{ux434ux438ux441ux43aux440ux435ux442ux438ux437ux430ux446ux456ux44f-ux433ux435ux43eux43cux435ux442ux440ux456ux457-ux442ux430-ux432ux438ux437ux43dux430ux447ux435ux43dux43dux44f-ux430ux43fux440ux43eux43aux441ux438ux43cux430ux446ux456ux439}{%
\subsubsection[2.3.2. Дискретизація геометрії та визначення
апроксимацій]{\texorpdfstring{\protect\hypertarget{anchor-37}{}{}\protect\hypertarget{anchor-38}{}{}2.3.2.
Дискретизація геометрії та визначення
апроксимацій}{2.3.2. Дискретизація геометрії та визначення апроксимацій}}\label{ux434ux438ux441ux43aux440ux435ux442ux438ux437ux430ux446ux456ux44f-ux433ux435ux43eux43cux435ux442ux440ux456ux457-ux442ux430-ux432ux438ux437ux43dux430ux447ux435ux43dux43dux44f-ux430ux43fux440ux43eux43aux441ux438ux43cux430ux446ux456ux439}}

Дискретизуємо
\includegraphics[width=0.1354in,height=0.15in]{./ObjectReplacements/Obj279}послідовністю
\includegraphics[width=0.1646in,height=0.1646in]{./ObjectReplacements/Obj280}
граничних елементів
\includegraphics[width=0.1783in,height=0.2071in]{./ObjectReplacements/Obj281}
(таких, що
\includegraphics[width=0.35in,height=0.4in]{./ObjectReplacements/Obj282}
апроксимує
\includegraphics[width=0.1646in,height=0.2071in]{./ObjectReplacements/Obj283},
а
\includegraphics[width=0.4283in,height=0.4in]{./ObjectReplacements/Obj284}
-
\includegraphics[width=0.1783in,height=0.2071in]{./ObjectReplacements/Obj285}),
моделюючи геометрію кожного з елементів за допомогою вектора
\includegraphics[width=0.1354in,height=0.2071in]{./ObjectReplacements/Obj286}
базових інтерполюючих функцій, пов'язаних із локальною нормалізованою
координатою
\includegraphics[width=0.1283in,height=0.1646in]{./ObjectReplacements/Obj287}.
Вздовж кожного з елементів апроксимуємо
\includegraphics[width=0.15in,height=0.1354in]{./ObjectReplacements/Obj288}
інтерполяційним поліномом

\includegraphics[width=1.0571in,height=0.2354in]{./ObjectReplacements/Obj289},(2.24)

де
\includegraphics[width=0.1783in,height=0.2071in]{./ObjectReplacements/Obj290}
- вектор невідомих вузлових значень
функції\includegraphics[width=0.15in,height=0.1354in]{./ObjectReplacements/Obj291}.

Кожну з областей
\includegraphics[width=0.2071in,height=0.2071in]{./ObjectReplacements/Obj292}
дискретизуємо системою ермітових чотирикутних елементів
\includegraphics[width=0.2646in,height=0.2071in]{./ObjectReplacements/Obj293}\includegraphics[width=0.6929in,height=0.2929in]{./ObjectReplacements/Obj294}
і представимо
\includegraphics[width=0.1283in,height=0.2071in]{./ObjectReplacements/Obj295}
пробною функцією {[}\protect\hyperlink{anchor-39}{24}{]}

\includegraphics[width=1.4854in,height=0.25in]{./ObjectReplacements/Obj296},(2.25)

де
\includegraphics[width=0.2217in,height=0.2354in]{./ObjectReplacements/Obj297}
- вектор невідомих вузлових значень функції
\includegraphics[width=0.1283in,height=0.2071in]{./ObjectReplacements/Obj298},
значень її перших та змішаних похідних на елементі
\includegraphics[width=0.2646in,height=0.2071in]{./ObjectReplacements/Obj299},
\includegraphics[width=0.15in,height=0.2071in]{./ObjectReplacements/Obj300}
- вектор базових функцій у локальній системі координат
\includegraphics[width=0.65in,height=0.2354in]{./ObjectReplacements/Obj301}.

Враховуючи (2.24), дискретний аналог (2.23) набуде вигляду

\includegraphics[width=1.7354in,height=0.4571in]{./ObjectReplacements/Obj302}.(2.26)

Зважаючи на (2.25),
\includegraphics[width=0.1646in,height=0.2071in]{./ObjectReplacements/Obj303}
(подібно й
\includegraphics[width=0.1929in,height=0.2071in]{./ObjectReplacements/Obj304})
зобразимо

\includegraphics[width=1.7783in,height=0.2929in]{./ObjectReplacements/Obj305},

де

\includegraphics[width=3.5283in,height=0.4717in]{./ObjectReplacements/Obj306}.

\hypertarget{ux43fux43eux431ux443ux434ux43eux432ux430-ux441ux438ux441ux442ux435ux43cux438-ux440ux456ux432ux43dux44fux43dux44c-ux434ux43bux44f-ux437ux43dux430ux445ux43eux434ux436ux435ux43dux43dux44f-ux432ux443ux437ux43bux43eux432ux438ux445-ux437ux43dux430ux447ux435ux43dux44c}{%
\subsubsection[2.3.3. Побудова системи рівнянь для знаходження вузлових
значень]{\texorpdfstring{\protect\hypertarget{anchor-40}{}{}\protect\hypertarget{anchor-41}{}{}2.3.3.
Побудова системи рівнянь для знаходження вузлових
значень}{2.3.3. Побудова системи рівнянь для знаходження вузлових значень}}\label{ux43fux43eux431ux443ux434ux43eux432ux430-ux441ux438ux441ux442ux435ux43cux438-ux440ux456ux432ux43dux44fux43dux44c-ux434ux43bux44f-ux437ux43dux430ux445ux43eux434ux436ux435ux43dux43dux44f-ux432ux443ux437ux43bux43eux432ux438ux445-ux437ux43dux430ux447ux435ux43dux44c}}

У підрозділі 2.2 запропонована процедура типу прогноз-коректування до
розв'язування зразу цілої серії задач класу (2.1), (2.2) для конкретно
вибраної кількості, геометрії та конфігурації розташування неоднорідних
включень, але різних функцій, що описують провідність середовища. Зараз
же представимо принципово інший підхід, а саме: систему рівнянь для
визначення вузлових значень побудуємо за методом зважених нев'язок,
уведених як для крайових умов, так і для областей локальних включень. У
матричній формі система матиме, зокрема, якщо скористатися непрямим
формулюванням методу граничних елементів, такий загальний вигляд:

\includegraphics[width=2.3217in,height=1.1in]{./ObjectReplacements/Obj307}\includegraphics[width=0.3354in,height=1.5429in]{./ObjectReplacements/Obj308}=\includegraphics[width=0.2646in,height=1.0717in]{./ObjectReplacements/Obj309},(2.27)

де
\includegraphics[width=0.1283in,height=0.1783in]{./ObjectReplacements/Obj310},
\includegraphics[width=0.15in,height=0.2354in]{./ObjectReplacements/Obj311},
...,
\includegraphics[width=0.1783in,height=0.2354in]{./ObjectReplacements/Obj312},
\includegraphics[width=0.1146in,height=0.1783in]{./ObjectReplacements/Obj313}
- вектори невідомих вузлових значень відповідно потенціалу на
дискретному аналозі
\includegraphics[width=0.1783in,height=0.2071in]{./ObjectReplacements/Obj314},
дискретних аналогах областей
\includegraphics[width=0.1929in,height=0.2071in]{./ObjectReplacements/Obj315},
...,
\includegraphics[width=0.2354in,height=0.2071in]{./ObjectReplacements/Obj316},
а також невідомі вузлові значення функції потоку на дискретному аналозі
\includegraphics[width=0.1646in,height=0.2071in]{./ObjectReplacements/Obj317}
краю області\[\Omega{}\].

Блоки \[V_{j}^{i}{}\] \[\left( {{j = u},v} \right){}\], \[G_{k}{}\]
\[\left( {k = \overline{1,K}} \right){}\] - суми вкладів відповідно
дискретного аналога потенціалу і потоку на \[\]\[\Gamma_{i}{}\]
\[\left( {i = 1,2} \right){}\], а також в \[\Omega_{k}{}\], для уявної
частини \[\hat{w}{}\] на \[\Gamma_{1}{}\]. Блоки \[U_{j}^{i}{}\]
\[\left( {{j = u},v} \right){}\], \[T_{k}{}\]
\[\left( {k = \overline{1,K}} \right){}\] - суми вкладів відповідно
аналога потенціалу і потоку на \[\]\[\Gamma_{i}{}\]
\[\left( {i = 1,2} \right){}\], а також в \[\Omega_{k}{}\], для дійсної
частини \[\hat{w}{}\] на \[\Gamma_{2}{}\]. Блоки
\[{\overset{\sim}{U}}_{\mathit{\text{jl}}}^{i}{}\]
\[\left( {{j = u},v} \right){}\],
\[{\overset{\sim}{T}}_{\mathit{\text{lk}}}{}\]
\[\left( {k,{l = \overline{1,K}}} \right){}\] - суми вкладів відповідно
дискретного аналога потенціалу і потоку на \[\]\[\Gamma_{i}{}\]
\[\left( {i = 1,2} \right){}\], а також в \[\Omega_{k}{}\], для уявної
частини \[\hat{w}{}\] на \[\Omega_{k}{}\].

У довільній точці
\includegraphics[width=0.1283in,height=0.1283in]{./ObjectReplacements/Obj349}
дискретного аналога області
\includegraphics[width=0.1646in,height=0.15in]{./ObjectReplacements/Obj350}
значення потенціалу та функції потоку знаходимо за допомогою (2.26),
використовуючи розв'язок (2.27).

\hypertarget{ux447ux438ux441ux43bux43eux432ux456-ux440ux435ux437ux443ux43bux44cux442ux430ux442ux438-1}{%
\subsubsection[2.3.4. Числові
результати]{\texorpdfstring{\protect\hypertarget{anchor-42}{}{}2.3.4.
Числові
результати}{2.3.4. Числові результати}}\label{ux447ux438ux441ux43bux43eux432ux456-ux440ux435ux437ux443ux43bux44cux442ux430ux442ux438-1}}

Числові дослідження викладеного підходу проводились для ряду змішаних
крайових задач теорії потенціалу типу (2.1)-(2.3). Наведемо результати,
одержані, зокрема, для задачі теплопровідності в області
\includegraphics[width=0.1646in,height=0.15in]{./ObjectReplacements/Obj351}
з краєм
\includegraphics[width=0.6929in,height=0.2071in]{./ObjectReplacements/Obj352},
де\textbf{:
}\includegraphics[width=4.0283in,height=0.2646in]{./ObjectReplacements/Obj353}\textbf{;
}\[\]\textbf{; }\[\]\textbf{, }\[\].

Граничні умови схарактеризуємо так \[\]\textbf{; }\[\]\textbf{. }

Усі графіки, наведені на рис. 2.4-2.7, стосуються таких конкретних
значень параметрів: \[\]; \[\]; \[\]; \[\].

\includegraphics[width=5.111in,height=3.278in]{Pictures/100000000000026F0000018FE2545062E6ED6A43.png}

Рис. 2.4

\includegraphics[width=3.6457in,height=2.3646in]{Pictures/100000000000032200000209705079E12375BE8C.jpg}Рисунок
2.4 наочно демонструє збурення температурного поля наявністю системи з
п'яти локальних включень, максимум коефіцієнта теплопровідності
матеріалу яких на порядок перевищує значення в однорідній частині.

Наведені на рис. 2.5 криві 1-3, побудовані для значень
\[\]\includegraphics[width=0.8854in,height=0.2071in]{./ObjectReplacements/Obj364}
відповідно, якщо в області -- єдине локальне включення з геометричними
характеристиками
\includegraphics[width=0.4146in,height=0.2217in]{./ObjectReplacements/Obj365};
Рис. 2.5

\includegraphics[width=0.5283in,height=0.2217in]{./ObjectReplacements/Obj366};
\includegraphics[width=0.4571in,height=0.2217in]{./ObjectReplacements/Obj367};
\includegraphics[width=0.5283in,height=0.2217in]{./ObjectReplacements/Obj368}.
Графіки демонструють зміну температурного поля вздовж осі
\includegraphics[width=0.3646in,height=0.2071in]{./ObjectReplacements/Obj369},
що проходить через центр області неоднорідності, де найяскравіше
проявляються закономірності впливу теплофізичних властивостей середовища
на розв'язок. Існує точка в зоні включення, розв'язок у якій не залежить
від коефіцієнта провідності в цій області

\includegraphics[width=5.111in,height=3.3055in]{Pictures/10000000000004000000029795911D6C1FD05AFF.jpg}

Рис. 2.6

Ізотерми 1-9, зображені на рис. 2.6, 2.7, відповідають значенням
температури 0,1; ...; 0,9. На рисунках вказано область неоднорідного
включення. Результати одержані при дискретизації кожного ребра
шестикутника за допомогою п'яти елементів, а області неоднорідного
включення -- шістнадцяти.

\includegraphics[width=5.139in,height=3.3335in]{Pictures/100000000000040000000297BBC8A69C9C99964F.jpg}

Рис. 2.7

Легко зауважити, що при \[{\lambda_{\text{max}}\rightarrow\infty}{}\]
розв'язок прямує до розв'язку задачі для однорідної області з отвором в
області включення, на краю якого задана стала температура; а при
\[{\lambda_{\text{max}}{\rightarrow + 0}}{}\] - з отвором, край якого
тепло ізольований.

\hypertarget{ux43fux456ux434ux441ux443ux43cux43eux43a}{%
\subsection[2.4.
Підсумок]{\texorpdfstring{\protect\hypertarget{anchor-43}{}{}2.4.
Підсумок}{2.4. Підсумок}}\label{ux43fux456ux434ux441ux443ux43cux43eux43a}}

До розв'язування послідовності задач для основного рівняння
стаціонарного поля в локально-неоднорідній області застосовано
комбіновану методику поєднання процедури типу прогноз--коректування з
комплексним методом граничних елементів. Виконані числові дослідження
температурного поля для широкого діапазону зміни значень коефіцієнта
теплопровідності в області з локальними неоднорідними включеннями різних
розмірів. З'ясовано, що використання методики ефективне як при
невеликих, так і при значних відхиленнях значень коефіцієнта в області
неоднорідності від значень у решті тіла.

Поширено область застосування КМГЕ на змішані плоскі задачі теорії
потенціалу для однозв'язних областей із сукупністю локальних
неоднорідних включень. Система рівнянь побудована за методом зважених
залишків, уведених як для крайових умов, так і для областей локальних
включень. Проведено широкий спектр числових досліджень стосовно
співвідношень значень коефіцієнта теплопровідності в областях
неоднорідних включень і поза ними. Запропоновано прийом моделювання з
використанням КМГЕ розв'язків задачі для багатозв'язної однорідної\emph{
}області з теплоізольованими та ізотермічними отворами за допомогою
розв'язків для локально-неоднорідної\emph{ }області.

Розділ 3

\hypertarget{ux437ux430ux434ux430ux447ux430-ux43fux440ux443ux436ux43dux43eux441ux442ux456-ux434ux43bux44f-ux43fux43bux43eux441ux43aux43eux433ux43e-ux442ux456ux43bux430-ux456ux437-ux43bux43eux43aux430ux43bux44cux43dux43e-ux43dux435ux43eux434ux43dux43eux440ux456ux434ux43dux438ux43c-ux432ux43aux43bux44eux447ux435ux43dux43dux44fux43c}{%
\section[Задача пружності для плоского тіла із локально-неоднорідним
включенням]{\texorpdfstring{\protect\hypertarget{anchor-44}{}{}Задача
пружності для плоского тіла із локально-неоднорідним
включенням}{Задача пружності для плоского тіла із локально-неоднорідним включенням}}\label{ux437ux430ux434ux430ux447ux430-ux43fux440ux443ux436ux43dux43eux441ux442ux456-ux434ux43bux44f-ux43fux43bux43eux441ux43aux43eux433ux43e-ux442ux456ux43bux430-ux456ux437-ux43bux43eux43aux430ux43bux44cux43dux43e-ux43dux435ux43eux434ux43dux43eux440ux456ux434ux43dux438ux43c-ux432ux43aux43bux44eux447ux435ux43dux43dux44fux43c}}

\hypertarget{ux43fux43eux441ux442ux430ux43dux43eux432ux43aux430-ux437ux430ux434ux430ux447ux456-1}{%
\subsection[3.1. Постановка
задачі]{\texorpdfstring{\protect\hypertarget{anchor-45}{}{}3.1.
Постановка
задачі}{3.1. Постановка задачі}}\label{ux43fux43eux441ux442ux430ux43dux43eux432ux43aux430-ux437ux430ux434ux430ux447ux456-1}}

Розглянемо тіло, що займає область
\includegraphics[width=0.1929in,height=0.1783in]{./ObjectReplacements/Obj372}
із граничною поверхнею
\includegraphics[width=0.1646in,height=0.1783in]{./ObjectReplacements/Obj373}.
Нехай коефіцієнти Ляме
\includegraphics[width=0.4429in,height=0.2929in]{./ObjectReplacements/Obj374},
\includegraphics[width=0.4429in,height=0.2929in]{./ObjectReplacements/Obj375}
матеріалу тіла описуються неперервними функціями, які набувають сталих
значень скрізь у
\includegraphics[width=0.1929in,height=0.1783in]{./ObjectReplacements/Obj376},
за винятком локальних областей неоднорідності
\[{\Omega_{m}\subset\Omega}{}\] (\[{m = \overline{1,M}}{}\]), причому
\[{{\Omega_{m} \cap \Omega_{l}} = \varnothing}{}\](\[{m \neq l}{}\]), де
вони залежать від декартових координат
\includegraphics[width=1.1354in,height=0.2929in]{./ObjectReplacements/Obj381},
тобто

\[{\lambda(X{) = {\lambda_{0} + {\sum\limits_{m = 1}^{M}{\lambda_{m}(X{) \cdot \chi_{m}}(X)}}}}}{}\],
\[{\mu(X{) = {\mu_{0} + {\sum\limits_{m = 1}^{M}{\mu_{m}(X{) \cdot \chi_{m}}(X)}}}}}{}\].

Тут \[{\chi_{m}(X)}{}\] --- характеристична функція області
\[\Omega_{m}{}\];
\includegraphics[width=0.2217in,height=0.2646in]{./ObjectReplacements/Obj386},
\includegraphics[width=0.2217in,height=0.2646in]{./ObjectReplacements/Obj387}
--- коефіцієнти Ляме в
\[{\Omega\underset{m = 1}{\overset{M}{}}\Omega_{m}}{}\];
\[{{\lambda_{0} + \lambda_{m}}(X)}{}\] і \[{{\mu_{0} + \mu_{m}}(X)}{}\]
--- коефіцієнти Ляме в області Ω\textsubscript{m}, причому
\[{\lambda_{m}(X){|_{\partial\Omega_{m}} = 0}}{}\],
\[{\mu_{m}(X){|_{\partial\Omega_{m}} = 0}}{}\], де
\[{\partial\Omega_{m}}{}\] --- край області Ω\textsubscript{m}.

Скориставшись у рівняннях рівноваги законом Гука

\[{\sigma_{\mathit{\text{ij}}}(X{) = \lambda_{0}}\delta_{\mathit{\text{ij}}}\varepsilon_{\mathit{\text{kk}}}(X{) + 2}\mu_{0}\varepsilon_{\mathit{\text{ij}}}(X{) + {{\sum\limits_{m = 1}^{M}{\lbrack\lambda_{m}\delta_{\mathit{\text{ij}}}\varepsilon_{\mathit{\text{kk}}}(X{) + 2}\mu_{m}\varepsilon_{\mathit{\text{ij}}}(X)\rbrack}} \cdot \chi_{m}}}(X)}{}\]

та співвідношеннями Коші, одержимо систему диференційних рівнянь у
переміщеннях

\[{\mu_{0}{u_{i,\mathit{\text{jj}}} + (}{\lambda_{0} + \mu_{0}}){u_{j,\mathit{\text{ji}}} = {- f_{\mathit{\text{ij}},j}}}}{}\],(3.1)

де
\[{{f_{\mathit{\text{ij}}} = f_{\mathit{\text{ij}}}}(\varepsilon{) = {\sum\limits_{m = 1}^{M}{\lbrack\lambda_{m}\delta_{\mathit{\text{ij}}}{\varepsilon_{\mathit{\text{kk}}} + 2}\mu_{m}\varepsilon_{\mathit{\text{ij}}}{\rbrack \cdot \chi_{m}}}}}}{}\]\textbf{;
}\includegraphics[width=0.2217in,height=0.2783in]{./ObjectReplacements/Obj397},
\includegraphics[width=0.1929in,height=0.2783in]{./ObjectReplacements/Obj398}
--- компоненти відповідно тензора напружень та тензора деформацій;
\includegraphics[width=0.1783in,height=0.2646in]{./ObjectReplacements/Obj399}---
компоненти вектора переміщень;
\includegraphics[width=0.2217in,height=0.2783in]{./ObjectReplacements/Obj400}---
символ Кронекера.

Вважатимемо, що на \[{P_{1}\subset\Gamma}{}\] задані переміщення, тобто

\[{u_{i}{|_{P_{1}} = u_{\mathit{\text{pi}}}}}{}\], (3.2)

а на P\textsubscript{2 }такій, що \[{{P_{1} \cup P_{2}} = \Gamma}{}\] і
\[{{P_{1} \cap P_{2}} = \varnothing}{}\], прикладені поверхневі зусилля

\[{t_{i}{|_{P_{2}} = t_{\mathit{\text{pi}}}}}{}\], (3.3)

де
\includegraphics[width=0.65in,height=0.2783in]{./ObjectReplacements/Obj406},
\includegraphics[width=0.1929in,height=0.2783in]{./ObjectReplacements/Obj407}
--- компоненти одиничного вектора зовнішньої нормалі до граничної
поверхні.

Отже, для визначення переміщень
\includegraphics[width=0.1646in,height=0.2354in]{./ObjectReplacements/Obj408}
маємо змішану крайову задачу (3.1)-(3.3).

Апроксимувавши компоненти невідомого тензора деформацій
\includegraphics[width=0.2071in,height=0.2783in]{./ObjectReplacements/Obj409}
в областях Ω\textsubscript{m} функціями
\[\varepsilon_{\mathit{\text{ij}}}^{m}{}\] і використавши їх у
\includegraphics[width=0.3071in,height=0.2783in]{./ObjectReplacements/Obj411},
перейдемо від крайової задачі відносно
\includegraphics[width=0.1354in,height=0.1354in]{./ObjectReplacements/Obj412}
до крайової задачі відносно
\includegraphics[width=0.2071in,height=0.25in]{./ObjectReplacements/Obj413}

\[{\mu_{0}{u_{i,\mathit{\text{jj}}}^{} + (}{\lambda_{0} + \mu_{0}}){u_{j,\mathit{\text{ji}}}^{} = {- f_{\mathit{\text{ij}},j}}}(\varepsilon)}{}\],(3.4)

\[{u_{i}^{}{|_{P_{1}} = u_{\mathit{\text{pi}}}}}{}\],
\[{t_{i}^{}{|_{P_{2}} = t_{\mathit{\text{pi}}}}}{}\].(3.5)

\hypertarget{ux456ux43dux442ux435ux433ux440ux430ux43bux44cux43dux456-ux437ux43eux431ux440ux430ux436ux435ux43dux43dux44f-ux440ux43eux437ux432ux44fux437ux43aux456ux432}{%
\subsection[3.2. Інтегральні зображення
розв'язків]{\texorpdfstring{\protect\hypertarget{anchor-46}{}{}3.2.
Інтегральні зображення
розв'язків}{3.2. Інтегральні зображення розв'язків}}\label{ux456ux43dux442ux435ux433ux440ux430ux43bux44cux43dux456-ux437ux43eux431ux440ux430ux436ux435ux43dux43dux44f-ux440ux43eux437ux432ux44fux437ux43aux456ux432}}

\protect\hypertarget{anchor-47}{}{}Для побудови розв'язків задачі
(3.4)-(3.5) застосуємо МГЕ, а також МПГЕ. З цією метою розглянемо таку
область \[{B\subset R^{2}}{}\], що
\[{\Omega\subset B,\quad\partial{\Omega \cap \partial}{B = \varnothing}}{}\].
Позначимо приграничну область \[{G = B}{}\]\textsubscript{. }Розв'язок
рівняння (3.4) зобразимо у вигляді:

\[{u_{i}^{\gamma}(X{) = {\int\limits_{\gamma}{E_{\mathit{\text{ij}}}(X,Y)\phi_{j}^{\gamma}(X)\mathit{d\gamma}(Y{) + {\sum\limits_{m = 1}^{M}{\int\limits_{\Omega_{m}}{E_{\mathit{\text{ij}}}(X,Y)f_{\mathit{\text{jk}},k}(\varepsilon^{\gamma_{m}}(Y))d\Omega_{m}{(Y{) + C_{i}}}}}}}}}}}{}\],(3.6)

де \[{\gamma\in{\{{G,\Gamma}\}}\text{.}}{}\]\textsubscript{
}Застосувавши до (3.6) теорему Гріна інтегрування частинами, одержимо
зображення для компонент вектора переміщень:

\[{u_{i}^{\gamma}(X{) = {\int\limits_{\gamma}{E_{\mathit{\text{ij}}}(X,Y)\phi_{j}^{\gamma}(X)\mathit{d\gamma}(Y{) - {\sum\limits_{m = 1}^{M}{\int\limits_{\Omega_{m}}{E_{\mathit{\text{ij}},k}(X,Y)f_{\mathit{\text{jk}}}(\varepsilon^{\gamma_{m}}(Y))d\Omega_{m}{(Y{) + C_{i}}}}}}}}}}}{}\](3.7)

і на основі нього інтегральні зображення компонент деформацій, напружень
та поверхневих зусиль

\[{\begin{matrix}
{\left\{ {\varepsilon_{\mathit{\text{ij}}}^{\gamma}(X)} \right\}\left\{ {\sigma_{\mathit{\text{ij}}}^{\gamma}(X)} \right\}} \\
\end{matrix}{{\{\}} = {\int\limits_{\gamma}{\begin{matrix}
{\left\{ {B_{\mathit{\text{ijk}}}(X,Y)} \right\}\left\{ {T_{\mathit{\text{ijk}}}(X,Y)} \right\}} \\
\end{matrix}{\{\}}\phi_{k}^{\gamma}(Y)\mathit{d\gamma}(Y{) - {\sum\limits_{m = 1}^{M}{\int\limits_{\Omega_{m}}{\begin{matrix}
{\left\{ {B_{\mathit{\text{ijk}},l}(X,Y)} \right\}\left\{ {T_{\mathit{\text{ijk}},l}(X,Y)} \right\}} \\
\end{matrix}{\{\}}f_{\mathit{\text{kl}}}(\varepsilon^{\gamma_{m}}(Y))d\Omega_{m}{(Y)}}}}}}}}}{}\]

\[{{{+ f_{\mathit{\text{ij}}}}(\varepsilon^{\gamma_{m}}(X))}\begin{matrix}
{\left\{ 0 \right\}\left\{ 1 \right\}} \\
\end{matrix}{\{\}}}{}\],(3.8)

де\[{E_{\mathit{\text{ij}}}(X,Y{) = E_{\mathit{\text{ij}}} = \frac{\lambda_{0} + \mu_{0}}{4\mathit{\text{πμ}_{\mathrm{0}}}({\lambda_{0} + \mu_{0}})}}\left\{ {{- \frac{{\lambda_{0} + 3}\mu_{0}}{\lambda_{0} + \mu_{0}}}\text{ln}{\mathit{r\delta}_{\mathit{\text{ij}}} + r_{,i}}r_{,j}} \right\}}{}\]---фундаментальний
розв'язок системи рівнянь задачі пружності;
\[{{Y = (}Y_{1},Y_{2}),\mspace{9mu} Y_{1},Y_{2}}{}\]--- система
координат, що співпадає зі системою \[{X_{1},X_{2}}{}\] і
використовується для опису точки, в якій прикладені компоненти невідомих
фіктивних поверхневих зусиль \[\phi_{j}^{\Gamma}{}\] чи масових сил
\[\phi_{j}^{G}{}\];
\includegraphics[width=1.6646in,height=0.3071in]{./ObjectReplacements/Obj430};
\includegraphics[width=0.1929in,height=0.2354in]{./ObjectReplacements/Obj431}
--- невідомі сталі,
\[{B_{\mathit{\text{ijk}}}(X,Y{) = B_{\mathit{\text{ijk}}} = \frac{1}{2}}({E_{\mathit{\text{ik}},j} + E_{\mathit{\text{jk}},i}})}{}\],
\[{T_{\mathit{\text{jk}}}(X,Y{) = T_{\mathit{\text{jk}}} = \lambda_{0}}\delta_{\mathit{\text{ij}}}{B_{\mathit{\text{llk}}} + 2}\mu_{0}B_{\mathit{\text{ijk}}}}{}\],
\[{F_{\mathit{\text{ij}}}(X,Y{) = F_{\mathit{\text{ij}}} = T_{\mathit{\text{ijk}}}}n_{k}}{}\].

\hypertarget{ux434ux438ux441ux43aux440ux435ux442ux438ux437ux430ux446ux456ux44f-ux433ux435ux43eux43cux435ux442ux440ux456ux457-ux442ux456ux43bux430}{%
\subsection[3.3. Дискретизація геометрії
тіла]{\texorpdfstring{\protect\hypertarget{anchor-48}{}{}3.3.
Дискретизація геометрії
тіла}{3.3. Дискретизація геометрії тіла}}\label{ux434ux438ux441ux43aux440ux435ux442ux438ux437ux430ux446ux456ux44f-ux433ux435ux43eux43cux435ux442ux440ux456ux457-ux442ux456ux43bux430}}

У приграничній області G і на ділянках краю
\[{P_{1},P_{2}}{}\]\textsubscript{ }уведемо відповідно пригра­нич­нi
\[G_{v}{}\] та граничні
\[{{\Gamma_{v} = \partial}{G_{v} \cap \Gamma}\mspace{9mu}}{}\]\[{({v = 1},\text{.}\text{.}\text{.},V)}{}\]\textsubscript{
}елементи, при­чому
\[{\mathit{\text{mes}}\mspace{9mu}{G_{\nu} = 2}}{}\],
\[{{}_{v = 1}^{V}{G_{v} = G}}{}\],
\[{{{G_{v} \cap G_{w}} = \varnothing},}{}\]
\[{{{\Gamma_{v} \cap \Gamma_{w}} = \varnothing},}{}\] при
\[{v \neq w}{}\], \[{{}_{v = 1}^{V_{1}}\Gamma_{v}{= P_{1}}}{}\],
\[{{}_{v = {V_{1} + 1}}^{V}\Gamma_{v}{= P_{2}}}{}\]. Геометрію
приграничних і граничних елементів моделюємо за допомогою вектора
\includegraphics[width=0.1783in,height=0.1783in]{./ObjectReplacements/Obj446}
інтерполяційних функцій у локальній системі координат. Вздовж кожного з
елементів \[\Gamma_{v}{}\] та по площі елемента
\[G_{v}{}\]\textsubscript{ }здійснимо апроксимацію невідомих функцій
\[\phi_{j}^{\gamma}{}\] за допомогою інтерполяційного полінома
\[{\phi_{j}^{\gamma}(\eta){|_{\gamma_{\nu}} = \psi^{T}}{(\eta{) \cdot \phi_{\mathit{j\nu}}^{\gamma}}}}{}\],
де \[\phi_{\mathit{\text{jv}}}^{\gamma}{}\] --- вектор невідомих
вузлових апроксимацій функції \[\phi_{j}^{\gamma}{}\] на ν -ому
елементі.

Області Ω\textsubscript{m} дискретизуємо системою ермітових скінчених
елементів Ω\textsubscript{ms} (s=\[\overline{1,S_{m}}{}\]), зобразивши
кожну функцію \[\varepsilon_{\mathit{\text{ij}}}^{\mathit{\gamma m}}{}\]
на кожному з них пробною функцією

\[{\varepsilon_{\mathit{\text{ij}}}^{\mathit{\gamma m}}{|_{\Omega_{\mathit{\text{ms}}}} = \xi^{T}}{(\zeta_{1}}{,\zeta_{2}}{) \cdot \varepsilon_{\mathit{\text{ij}}}^{\gamma\mathit{\text{ms}}}}}{}\],

де \[\varepsilon_{\mathit{\text{ij}}}^{\gamma\mathit{\text{ms}}}{}\] ---
вузловий вектор значень компоненти
\[\varepsilon_{\mathit{\text{ij}}}^{\gamma}{}\] тензора деформацій, її
перших та змішаних похідних на \emph{s}-тому елементі,
\includegraphics[width=0.1354in,height=0.2354in]{./ObjectReplacements/Obj458}
--- вектор базових функцій у локальній системі координат.

На основі таких апроксимацій одержимо дискретні аналоги формул (7),(8)

\[{u_{i}^{\gamma}(X{) = E_{\mathit{\text{ij}}}^{\mathit{\text{γν}}}}(X{{) \cdot \phi_{\mathit{j\nu}}^{\gamma}} + E_{i}^{\lambda\mathit{\text{ms}}}}(X{{) \cdot \varepsilon_{\mathit{\text{kk}}}^{\gamma\mathit{\text{ms}}}} + E_{\mathit{\text{ijl}}}^{\mu\mathit{\text{ms}}}}(X{{) \cdot \varepsilon_{\mathit{\text{jl}}}^{\gamma\mathit{\text{ms}}}} + C_{i}}}{}\],(3.9)

\[{\begin{matrix}
{\left\{ \varepsilon_{\mathit{\text{ij}}}^{\gamma} \right\}\left\{ \sigma_{\mathit{\text{ij}}}^{\gamma} \right\}} \\
\end{matrix}{\{\}}(X{) =}\begin{matrix}
{\left\{ B_{\mathit{\text{ijk}}}^{\nu} \right\}\left\{ T_{\mathit{\text{ijk}}}^{\nu} \right\}} \\
\end{matrix}{\{\}}(X{{) \cdot \phi_{\mathit{k\nu}}^{\gamma}} +}\begin{matrix}
{\left\{ B_{\mathit{\text{ij}}}^{\lambda\mathit{\text{ms}}} \right\}\left\{ T_{\mathit{\text{ij}}}^{\lambda\mathit{\text{ms}}} \right\}} \\
\end{matrix}{\{\}}(X{{) \cdot \varepsilon_{\mathit{\text{kk}}}^{\gamma\mathit{\text{ms}}}} +}\begin{matrix}
{\left\{ B_{\mathit{\text{ijkl}}}^{\mu\mathit{\text{ms}}} \right\}\left\{ T_{\mathit{\text{ijkl}}}^{\mu\mathit{\text{ms}}} \right\}} \\
\end{matrix}{\{\}}(X{) \cdot \varepsilon_{\mathit{\text{kl}}}^{\gamma\mathit{\text{ms}}}}}{}\],(3.10)

де

\[{\begin{matrix}
{\left\{ E_{\mathit{\text{ij}}}^{\Gamma_{v}} \right\}\left\{ B_{\mathit{\text{ijk}}}^{\mathit{\Gamma\nu}} \right\}\left\{ T_{\mathit{\text{ijk}}}^{\mathit{\Gamma\nu}} \right\}} \\
\end{matrix}{\{\}}(X{) = {\int\limits_{- 1}^{1}{\begin{matrix}
{\left\{ E_{\mathit{\text{ij}}} \right\}\left\{ B_{\mathit{\text{ijk}}} \right\}\left\{ T_{\mathit{\text{ijk}}} \right\}} \\
\end{matrix}{\{\}}}}}\psi^{T}J_{\mathit{\Gamma\nu}}\mathit{d\eta}}{}\];\[{\begin{matrix}
{\left\{ E_{\mathit{\text{ij}}}^{G_{v}} \right\}\left\{ B_{\mathit{\text{ijk}}}^{\mathit{G\nu}} \right\}\left\{ T_{\mathit{\text{ijk}}}^{\mathit{G\nu}} \right\}} \\
\end{matrix}{\{\}}(X{) = {\int\limits_{- 1}^{1}{{\int\limits_{- 1}^{1}{\begin{matrix}
{\left\{ E_{\mathit{\text{ij}}} \right\}\left\{ B_{\mathit{\text{ijk}}} \right\}\left\{ T_{\mathit{\text{ijk}}} \right\}} \\
\end{matrix}{\{\}}}}\psi^{T}J_{\mathit{G\nu}}\mathit{d\eta}_{1}\mathit{d\eta}_{2}}}}}{}\];

\[{\begin{matrix}
{\left\{ E_{i}^{\lambda\mathit{\text{ms}}} \right\}\left\{ B_{\mathit{\text{ij}}}^{\lambda\mathit{\text{ms}}} \right\}\left\{ T_{\mathit{\text{ij}}}^{\lambda\mathit{\text{ms}}} \right\}} \\
\end{matrix}{\{\}}(X{) = {- {\int\limits_{- 1}^{1}{{\int\limits_{- 1}^{1}{\begin{matrix}
{\left\{ E_{\mathit{\text{ij}},j} \right\}\left\{ B_{\mathit{\text{ijk}},k} \right\}\left\{ T_{\mathit{\text{ijk}},k} \right\}} \\
\end{matrix}{\{\}}\lambda_{\mathit{\text{ms}}}\xi^{T}H_{\mathit{\text{ms}}}\mathit{d\zeta}_{1}{\mathit{d\zeta}_{2} + \lambda_{\mathit{\text{ms}}}}}}\begin{matrix}
{\left\{ 0 \right\}\left\{ 0 \right\}\left\{ \delta_{\mathit{\text{ij}}} \right\}} \\
\end{matrix}{\{\}}\xi^{T}\chi_{\mathit{\text{ms}}}(X)}}}}}{}\],

\[{\begin{matrix}
{\left\{ E_{\mathit{\text{ijl}}}^{\mu\mathit{\text{ms}}} \right\}\left\{ B_{\mathit{\text{ijkl}}}^{\mu\mathit{\text{ms}}} \right\}\left\{ T_{\mathit{\text{ijkl}}}^{\mu\mathit{\text{ms}}} \right\}} \\
\end{matrix}{\{\}}(X{) = {- 2}}{\int\limits_{- 1}^{1}{{\int\limits_{- 1}^{1}{\begin{matrix}
{\left\{ E_{\mathit{\text{ij}},l} \right\}\left\{ B_{\mathit{\text{ijk}},k} \right\}\left\{ T_{\mathit{\text{ijk}},k} \right\}} \\
\end{matrix}{\{\}}\mu_{\mathit{\text{ms}}}\xi^{T}H_{\mathit{\text{ms}}}\mathit{d\zeta}_{1}{\mathit{d\zeta}_{2} + 2}\mu_{\mathit{\text{ms}}}}}\begin{matrix}
{\left\{ 0 \right\}\left\{ 0 \right\}\left\{ 1 \right\}} \\
\end{matrix}{\{\}}\xi^{T}\chi_{\mathit{\text{ms}}}(X)}}}{}\].

\[J_{\Gamma_{v}}{}\],\[J_{G_{v}}{}\],\[H_{\mathit{\text{ms}}}{}\] --
якобіани переходу від змінних \[\eta{}\], \[\eta_{1}{}\],
\[\eta_{2}{}\];\[\zeta_{1}{}\], \[\zeta_{2}{}\] до Х відповідно.
\[\chi_{\mathit{\text{ms}}}{}\]-- характеристична функція елемента
\[\Omega_{\mathit{\text{ms}}}{}\], \[\lambda_{\mathit{\text{ms}}}{}\],
\[\mu_{\mathit{\text{ms}}}{}\] -- значення функцій
\[{\lambda_{m}(X)}{}\], \[{\mu_{m}(X)}{}\] при
\[{X\in\Omega_{\mathit{\text{ms}}}}{}\].

\hypertarget{ux43fux43eux431ux443ux434ux43eux432ux430-ux441ux438ux441ux442ux435ux43cux438-ux440ux456ux432ux43dux44fux43dux44c-ux434ux43bux44f-ux437ux43dux430ux445ux43eux434ux436ux435ux43dux43dux44f-ux432ux443ux437ux43bux43eux432ux438ux445-ux437ux43dux430ux447ux435ux43dux44c-1}{%
\subsection[3.4. Побудова системи рівнянь для знаходження вузлових
значень
]{\texorpdfstring{\protect\hypertarget{anchor-49}{}{}\protect\hypertarget{anchor-50}{}{}3.4.
Побудова системи рівнянь для знаходження вузлових значень
}{3.4. Побудова системи рівнянь для знаходження вузлових значень }}\label{ux43fux43eux431ux443ux434ux43eux432ux430-ux441ux438ux441ux442ux435ux43cux438-ux440ux456ux432ux43dux44fux43dux44c-ux434ux43bux44f-ux437ux43dux430ux445ux43eux434ux436ux435ux43dux43dux44f-ux432ux443ux437ux43bux43eux432ux438ux445-ux437ux43dux430ux447ux435ux43dux44c-1}}

Уведемо функції нев'язок
\[{R_{\mathit{\text{ij}}}^{\gamma\mathit{\text{ms}}}(x{) = \varepsilon_{\mathit{\text{ij}}}^{\gamma}}(x{) - \varepsilon_{\mathit{\text{ij}}}^{\gamma\mathit{\text{ms}}}}(x)}{}\]
на
\[{\underset{s = 1}{\overset{S_{m}}{}}\Omega_{\mathit{\text{ms}}}}{}\]
(\[{m = \overline{1,M}}{}\]),
\[{R_{1i}^{\gamma}(x{) = u_{i}^{\gamma}}(x{) - u_{\mathit{\text{pi}}}}(x)}{}\]
на \[P_{1}{}\] і
\[{R_{2i}^{\gamma}(x{) = t_{i}^{\gamma}}(x{) - t_{\mathit{\text{pi}}}}(x)}{}\]
на \[P_{2}{}\].

Унаслідок граничного переходу та дискретизації виразів для переміщень
(3.9) та зусиль (3.10), одержимо їх дискретні аналоги у точці
\[{X_{0}{\in\underset{\nu = 1}{\overset{V}{\cup}}\Gamma_{\nu}}}{}\]:

\[{u_{i}^{\gamma}(X_{0}{) = E_{\mathit{\text{ij}}}^{\mathit{\text{γν}}}}(X_{0}{{) \cdot \phi_{\mathit{j\nu}}^{\gamma}} + E_{i}^{\lambda\mathit{\text{ms}}}}(X_{0}{{) \cdot \varepsilon_{\mathit{\text{kk}}}^{\gamma\mathit{\text{ms}}}} + E_{\mathit{\text{ijl}}}^{\mu\mathit{\text{ms}}}}(X_{0}{{) \cdot \varepsilon_{\mathit{\text{jl}}}^{\gamma\mathit{\text{ms}}}} + C_{i}}}{}\],
(3.11)

\[{t_{i}^{\Gamma}(X_{0}{) = {\pm \frac{1}{2}}}\delta_{\mathit{\text{ij}}}\psi^{T}(X_{0}{{) \cdot \phi_{\mathit{j\nu}}^{\Gamma}} + F_{\mathit{\text{ij}}}^{\mathit{\Gamma\nu}}}(X_{0}{{) \cdot \phi_{\mathit{j\nu}}^{\Gamma}} + F_{i}^{\lambda\mathit{\text{ms}}}}(X_{0}{{) \cdot \varepsilon_{\mathit{\text{kk}}}^{\Gamma\mathit{\text{ms}}}} + F_{\mathit{\text{ijl}}}^{\mu\mathit{\text{ms}}}}(X_{0}{) \cdot \varepsilon_{\mathit{\text{jl}}}^{\Gamma\mathit{\text{ms}}}}}{}\],\[{t_{i}^{G}(X_{0}{) = F_{\mathit{\text{ij}}}^{\mathit{G\nu}}}(X_{0}{{) \cdot \phi_{\mathit{j\nu}}^{G}} + F_{i}^{\lambda\mathit{\text{ms}}}}(X_{0}{{) \cdot \varepsilon_{\mathit{\text{kk}}}^{\mathit{\text{Gms}}}} + F_{\mathit{\text{ijl}}}^{\mu\mathit{\text{ms}}}}(X_{0}{) \cdot \varepsilon_{\mathit{\text{jl}}}^{\mathit{\text{Gms}}}}}{}\](3.12)

причому точка X\textsubscript{0} така, що в ній існує єдина дотична до
\[{\underset{\nu = 1}{\overset{V}{}}\Gamma_{\nu}}{}\]; інтеграли
\[{F_{\mathit{\text{ij}}}^{\mathit{\Gamma\nu}}(X_{0})}{}\] слід розуміти
у сенсі головного значення Коші, а всі решта --- у сенсі Рімана.

Для визначення невідомих векторів вузлових значень фіктивних поверхневих
зусиль \[\phi_{\mathit{j\nu}}^{\Gamma}{}\] та масових сил
\[\phi_{\mathit{j\nu}}^{G}{}\], векторів вузлових значень компонент
тензора
\[\varepsilon_{\mathit{\text{ij}}}^{\gamma\mathit{\text{ms}}}{}\] в
областях неоднорідностей та вектора
\includegraphics[width=0.6783in,height=0.2929in]{./ObjectReplacements/Obj496}
побудуємо систему лінійних алгебричних рівнянь за методом зважених
нев'язок. Враховуємо для цього і дискретні аналоги

\begin{quote}
\[{{{W^{\mathit{\text{γν}}} \cdot \phi_{\mathit{j\nu}}^{\gamma}} + {W_{j}^{\lambda\mathit{\text{ms}}} \cdot \varepsilon_{\mathit{\text{kk}}}^{\gamma\mathit{\text{ms}}}} + {W_{l}^{\mu\mathit{\text{ms}}} \cdot \varepsilon_{\mathit{\text{jl}}}^{\gamma\mathit{\text{ms}}}}} = 0}{}\]
(3.13)
\end{quote}

умов

\[{\int\limits_{\gamma}{\phi_{j}^{\gamma}(Y)\mathit{d\gamma}(Y{) + {\sum\limits_{m = 1}^{M}{\int\limits_{\Omega_{m}}{f_{\mathit{\text{jk}},k}(\varepsilon^{\gamma_{m}}(Y))d\Omega_{m}{(Y{) = 0}}}}}}}}{}\],

де

\[{W^{\mathit{\Gamma\nu}} = {\int\limits_{- 1}^{1}{\psi^{T}J_{\mathit{\Gamma\nu}}\mathit{d\eta}}}}{}\],
\[{W^{\mathit{G\nu}} = {\int\limits_{- 1}^{1}{\int\limits_{- 1}^{1}{\psi^{T}J_{\mathit{G\nu}}\mathit{d\eta}_{1}\mathit{d\eta}_{2}}}}}{}\],
\[{W_{j}^{\lambda\mathit{\text{ms}}} = {\int\limits_{- 1}^{1}{\int\limits_{- 1}^{1}{\lambda_{\mathit{\text{ms}}}\xi^{T}H_{\mathit{\text{ms}}}\mathit{d\zeta}_{1}\mathit{d\zeta}_{2}}}}}{}\],
\[{{W_{j}^{\mu\mathit{\text{ms}}} = 2}{\int\limits_{- 1}^{1}{\int\limits_{- 1}^{1}{\mu_{\mathit{\text{ms}}}\xi^{T}H_{\mathit{\text{ms}}}\mathit{d\zeta}_{1}\mathit{d\zeta}_{2}}}}}{}\].

У матричній формі система для знаходження невідомих
\[\phi_{\mathit{\text{jv}}}^{\gamma}{}\],
\[\varepsilon_{\mathit{\text{ij}}}^{\gamma\mathit{\text{ms}}}{}\] та
\[C_{j}{}\] набуде такого загального вигляду

\[{\begin{bmatrix}
\begin{matrix}
E^{\gamma} \\
F^{\gamma} \\
B^{\gamma} \\
W^{\gamma} \\
\end{matrix} & {\begin{matrix}
E^{1} \\
F^{1} \\
B^{1} \\
W^{1} \\
\end{matrix}\text{.}\text{.}\text{.}\begin{matrix}
E^{M} \\
F^{M} \\
B^{M} \\
W^{M} \\
\end{matrix}} & \begin{matrix}
I \\
0 \\
0 \\
0 \\
\end{matrix} \\
\end{bmatrix}{\begin{bmatrix}
\phi^{\gamma} \\
\varepsilon^{1} \\
 \vdots \\
\varepsilon^{M} \\
C \\
\end{bmatrix} = \begin{bmatrix}
u^{p} \\
t^{p} \\
0 \\
0 \\
\end{bmatrix}}}{}\],(3.14)

де \[\phi^{\gamma}{}\] --- вектор невідомих вузлових значень фіктивних
поверхневих зусиль або масових сил у всьому дискретному аналозі
\[{\underset{\nu = 1}{\overset{V}{}}\gamma_{v}}{}\] краю тіла або
приграничної до нього області;

\[\varepsilon^{m}{}\] --- вектор невідомих вузлових значень компонент
деформацій та їх похідних у дискретному аналозі
\[{\underset{s = 1}{\overset{S_{m}}{}}\Omega_{\mathit{\text{ms}}}}{}\]
(\[{m = \overline{1,M}}{}\]) області локальної неоднорідності
Ω\textsubscript{m};

u\textsuperscript{p}, t\textsuperscript{p} --- вектори значень заданих
на краю тіла переміщень та поверхневих зусиль.

Елементи блоків глобальної матриці системи є сумами внесків окремих
елементів (3.10)-(3.13). А саме блок \[E^{\gamma}{}\] сформований з
\[E_{\mathit{\text{ij}}}^{\mathit{\text{γν}}}{}\]; блок \[E^{m}{}\] ---
з \[E_{i}^{\lambda\mathit{\text{ms}}}{}\] та
\[E_{\mathit{\text{ijl}}}^{\mu\mathit{\text{ms}}}{}\]; блок
\[F^{\gamma}{}\] --- з
\[F_{\mathit{\text{ik}}}^{\mathit{\text{γν}}}{}\]; блок \[F^{m}{}\] ---
з \[F_{i}^{\lambda\mathit{\text{ms}}}{}\] та
\[F_{\mathit{\text{ijl}}}^{\mu\mathit{\text{ms}}}{}\]; блок
\[B^{\gamma}{}\]--- з
\[B_{\mathit{\text{ijk}}}^{\mathit{\text{γν}}}{}\]; блок \[B^{m}{}\] ---
з \[B_{\mathit{\text{ij}}}^{\lambda\mathit{\text{ms}}}{}\] та
\[B_{\mathit{\text{ijkl}}}^{\mu\mathit{\text{ms}}}{}\]; блок
\[W^{\gamma}{}\]--- з \[W^{\mathit{\text{γν}}}{}\]; блок \[W^{m}{}\] ---
з \[W_{j}^{\lambda\mathit{\text{ms}}}{}\] та
\[W_{l}^{\mu\mathit{\text{ms}}}{}\].

У довільній точці
\includegraphics[width=0.8854in,height=0.2071in]{./ObjectReplacements/Obj532}
переміщення, деформації, напруження та поверх­неві зусилля визначають з
формул (3.9),(3.10), використовуючи розв'язок системи (3.14).

\hypertarget{ux43fux43eux440ux456ux432ux43dux44fux43dux43dux44f-ux447ux438ux441ux43bux43eux432ux438ux445-ux440ux435ux437ux443ux43bux44cux442ux430ux442ux456ux432}{%
\subsection[3.5. Порівняння числових
результатів]{\texorpdfstring{\protect\hypertarget{anchor-51}{}{}3.5.
Порівняння числових
результатів}{3.5. Порівняння числових результатів}}\label{ux43fux43eux440ux456ux432ux43dux44fux43dux43dux44f-ux447ux438ux441ux43bux43eux432ux438ux445-ux440ux435ux437ux443ux43bux44cux442ux430ux442ux456ux432}}

Для порівняння МГЕ і МПГЕ було вибрано просту задачу з теорії пружності:

Тіло \[\Omega{}\] було вибрано у вигляді квадрата, гранична поверхня
якого задана так: \[{\partial\Omega}{}\], де

\[{\partial{\Omega_{1} = {\{{(x_{1},x_{2}):{x_{1} = 0,0 < x_{2} < 1}}\}}}}{}\],
\[{\partial{\Omega_{2} = {\{{(x_{1},x_{2}):{x_{2} = 0,0 \leq x_{1} \leq 1}}\}}}}{}\],

\[{\partial{\Omega_{3} = {\{{(x_{1},x_{2}):{x_{1} = 1,0 < x_{2} < 1}}\}}}}{}\],

\[{\partial{\Omega_{4} = {\{{(x_{1},x_{2}):{x_{2} = 1,0 \leq x_{1} \leq 1}}\}}}}{}\].

Граничні умови були типу Діріхле:

На \[{\partial\Omega_{1}}{}\]і \[{\partial\Omega_{3}}{}\]було задано
\[{u_{\text{10}} = 0}{}\],
\[{{u_{\text{20}} = x_{2}}\text{10}^{- 2}}{}\].

На \[{\partial\Omega_{2}}{}\]і \[{\partial\Omega_{4}}{}\]-
\[{u_{\text{10}} = 0}{}\],
\[{{u_{\text{20}} = x_{2}}\text{10}^{- 2}}{}\].

Цю задачу розв'язували, використовуючи вісім граничних елементів і вісім
приграничних елементів. На кожній ділянці \textsubscript{l }(l=1..4)
вибирали по два граничні елементи однакової довжини, на яких будували
прямокутні приграничні елементи висотою h.

Інтенсивності введених ``фіктивних'' сил у кожному граничному і
приграничному елементі апроксимуються кусково-постійними функціями.
Розв'язки одержані за допомогою МГЕ і МПГЕ порівнювалися з точним
аналітичним розв'язком даної задачі. Графіки похибок переміщень
u\textsubscript{i} (i=1,2) показані на частині межі \textsubscript{2}.
Не доцільно показувати графіки похибок всередині області , оскільки
виконується принцип максимуму і похибки всередині області не перевищують
похибки на межі області.

\includegraphics[width=5.889in,height=4.389in]{Pictures/100000000000035A0000027F723B715214D19193.png}

Рис. 3.1

На рис. 3.1 зображено похибки розв'язку u\textsubscript{1} для МГЕ -- це
крива 4, і для МПГЕ -- це криві 1, 2 і 3, кожна із яких відповідає
різній ширині смуги приграничних елементів, в даному випадку вона являє
собою 0.5, 1, 2 відповідно, при тій самій кількості приграничних
елементів. Спостерігається цікавий ефект, при збільшенні ширини
приграничної зони зростає точність. Отже, такий параметр МПГЕ, як ширина
приграничної зони вже показує свої переваги над МГЕ, в якому такого
параметра немає і ми відповідно не можемо його міняти.

\includegraphics[width=5.889in,height=4.4445in]{Pictures/100000000000034E0000027E1BB74643BAD86187.png}

Рис. 3.2

На рис. 3.2 зображено похибки розв'язку u\textsubscript{2} для МГЕ -- це
крива 4, і для МПГЕ -- це криві 1, 2 і 3, кожна із яких відповідає
конкретній ширині смуги приграничних елементів, а саме 0.5, 1, 2
відповідно.

На рис. 3.1, 3.2 похибки помножені на 10\textsuperscript{4} для того щоб
результати були наочнішими.

Порівняльний аналіз графіків засвідчує, що точність обчислень переміщень
при використанні восьми елементів за допомогою МПГЕ є вищою, ніж за
допомогою розв'язків МГЕ. Такий висновок підтвердився і при використанні
більшої кількості граничних і відповідних їм приграничних елементів.

На основі вище наведеного прикладу було продемонстровану одну з переваг
МПГЕ над МГЕ.

Наведемо результати досліджень, які проведені для тіла, що займає
область

 = \{ (X\textsubscript{1}, X\textsubscript{2}): 0  X\textsubscript{1
} 1, 0  X\textsubscript{2 } 1 \}

і містить в собі наступну неоднорідність

\textsubscript{g} = \{ (X\textsubscript{1}, X\textsubscript{2}): 0.7 
X\textsubscript{1 } 1.3, 0.7  X\textsubscript{2 } 1.3 \}

для уникнення громіздкості запису тут і надалі не будемо приводити
розмірності декартових координат, які, як і решта величин, розглядаються
в системі СІ. Зрозуміло, що в цьому випадку гранична поверхня тіла
описується виразом \[{{Г = \underset{і = 1}{\overset{4}{}}}Г_{і}}{}\],
де

Г\textsubscript{1} = \{ (X\textsubscript{1}, X\textsubscript{2­}): 0 
X\textsubscript{1 } 1, X\textsubscript{2} = 0\}

Г\textsubscript{2} = \{ (X\textsubscript{1}, X\textsubscript{2­}):
X\textsubscript{1 }= 1, 0  X\textsubscript{2 } 1\}

Г\textsubscript{3} = \{ (X\textsubscript{1}, X\textsubscript{2­}): 0 
X\textsubscript{1 } 1, X\textsubscript{2} = 1\}

Г\textsubscript{4} = \{ (X\textsubscript{1}, X\textsubscript{2­}):
X\textsubscript{1 }= 0, 0  X\textsubscript{2 } 1\}

На
\includegraphics[width=0.7929in,height=0.2646in]{./ObjectReplacements/Obj548}
переміщення:

На Г\textsubscript{1} і Г\textsubscript{3} було задано
\[{u_{\text{10}} = 0}{}\],
\[{{u_{\text{20}} = x_{2}}\text{10}^{- 2}}{}\].

На Г\textsubscript{2} і Г\textsubscript{4 } --\[{u_{\text{10}} = 0}{}\],
\[{{u_{\text{20}} = x_{2}}\text{10}^{- 2}}{}\].

Коефіцієнт Пуасона був сталий  = 0.22; коефіцієнт Ляме задавався
формулою:  = 8200 + 1000  (0.7-X\textsubscript{2}) 
(X\textsubscript{2}-1.3).

\includegraphics[width=6.222in,height=4.0693in]{Pictures/10000000000004000000029DB842D6D5524D8E2B.png}

Рис. 3.3

На рис. 3.3 зображення значення u\textsubscript{2} вектора переміщень.
Всередині видно збурення, яке зумовлене тим, що там міститься включення.

\hypertarget{ux43fux456ux434ux441ux443ux43cux43eux43a-1}{%
\subsection[3.6.
Підсумок]{\texorpdfstring{\protect\hypertarget{anchor-52}{}{}3.6.
Підсумок}{3.6. Підсумок}}\label{ux43fux456ux434ux441ux443ux43cux43eux43a-1}}

У результаті порівняння МГЕ і МПГЕ отримали наступні висновки:

\begin{enumerate}
\def\labelenumi{\arabic{enumi}.}
\tightlist
\item
  Коли точка колокації належить елементу границі, по якому проводять
  інтегрування, переважають діагональні елементи у матрицях для МГЕ і
  МПГЕ, що сприяє добрій обумовленості матриці при розв'язуванні системи
  лінійних алгебраїчних рівнянь. Дещо нижча (в порівнянні із МГЕ)
  діагональна перевага у МПГЕ не вплинула на якість обчислень.
\item
  При обчисленні поверхневих зусиль в МГЕ з'являються особливості при
  обчисленні інтегралів, які існують в сенсі Коші, що вимагає
  попереднього аналітичного виділення особливостей (головного значення).
  У МПГЕ всі інтеграли розглядаються як інтеграли без особливостей, а це
  дозволяє у разі потреби обмежуватися тільки числовим інтегруванням.
\item
  Програмне забезпечення у випадку МПГЕ -- єдине для інтегрування по
  внутрішніх і приграничних елементах (криволінійних контурах), а у МГЕ
  слід будувати окреме для граничних (по криволінійних відрізках) і
  внутрішніх елементів.
\item
  В МПГЕ є додатковий параметр методу (ширина приграничної смуги), який
  дає додаткову ступінь вільності при роботі з цим методом. У МГЕ такого
  параметра немає. Як ми бачили з наведеного вище прикладу, в задачах
  теорії пружності він відіграє суттєву роль при збільшені точності
  методу.
\end{enumerate}

Розділ 4

\hypertarget{ux437ux430ux434ux430ux447ux430-ux43fux440ux443ux436ux43dux43eux441ux442ux456-ux443-ux43aux443ux441ux43aux43eux432ux43e-ux43eux434ux43dux43eux440ux456ux434ux43dux43eux43cux443-ux43fux43bux43eux441ux43aux43eux43cux443-ux442ux456ux43bux456}{%
\section[Задача пружності у кусково-однорідному плоскому
тілі]{\texorpdfstring{\protect\hypertarget{anchor-53}{}{}Задача
пружності у кусково-однорідному плоскому
тілі}{Задача пружності у кусково-однорідному плоскому тілі}}\label{ux437ux430ux434ux430ux447ux430-ux43fux440ux443ux436ux43dux43eux441ux442ux456-ux443-ux43aux443ux441ux43aux43eux432ux43e-ux43eux434ux43dux43eux440ux456ux434ux43dux43eux43cux443-ux43fux43bux43eux441ux43aux43eux43cux443-ux442ux456ux43bux456}}

\hypertarget{ux43fux43eux441ux442ux430ux43dux43eux432ux43aux430-ux437ux430ux434ux430ux447ux456-2}{%
\subsection[4.1. Постановка
задачі]{\texorpdfstring{\protect\hypertarget{anchor-54}{}{}4.1.
Постановка
задачі}{4.1. Постановка задачі}}\label{ux43fux43eux441ux442ux430ux43dux43eux432ux43aux430-ux437ux430ux434ux430ux447ux456-2}}

Розглянемо в умовах плоскої деформації кусково-однорідне тіло, що займає
область \[{\Omega\subset R^{2}}{}\] і склада­ється з \emph{M }частин
\[\Omega_{m}{}\] \[{({m = 1},\text{.}\text{.}\text{.},M)}{}\], які,
перебуваючи в ідеальному механічному контакті, характери­зу­ються
постійними модулем зсуву \[\mu_{m}{}\] та коефіцієн­том Пуассона
\[\nu_{m}{}\]. На частинах
\[{\partial \Omega_{m}^{1}\subset\partial\Omega_{m},\quad\partial \Omega_{m}^{2}\subset\partial\Omega_{m}}{}\]
краю \[{\partial\Omega}{}\] задано від­по­відно пове­дінку компонент
\[{u_{i}^{(m)}(x)}{}\], \[{t_{i}^{(m)}(x)}{}\] (\emph{i=1,2}) векторів
переміщень \[{u^{(m)}(x)}{}\] та поверхневих зусиль \[{t^{(m)}(x)}{}\],
на частинах \[{\partial \Omega_{m}^{s}\subset\partial\Omega_{m}}{}\]
(\emph{s=3,4}) -- змішані граничні умови, а всере­динi областей
\[\Omega_{m}{}\] дiють масові сили
\[{\psi^{m}(x{) = (}\psi_{1}^{m}(x),\psi_{2}^{m}(x))}{}\], де
\[{{x = (}x_{1},x_{2})}{}\]. Зрозуміло, що
\[{{\mspace{9mu} \cup_{m = 1}^{M}{}_{s = 1}^{4}}\partial {\Omega_{m}^{s} = \partial}\Omega}{}\].

Для визначення компонент вектора перемі­щень
\[{{u^{(m)} = u^{(m)}}(x)}{}\] в \[\Omega_{m}{}\] записано рівняння

\[{\mu_{m}{u_{i,\mathit{\text{jj}}}^{(m)} + \mu_{m}}{u_{j,\mathit{\text{ji}}}^{(m)}/(}{1 - 2}\nu_{m}{) = {- \psi_{i}^{(m)}}},\mspace{9mu} x\in\Omega_{m},}{}\]\[{i,{j = 1,2},}{}\](4.1)

контактні

\[{{u_{i}^{(m)} = u_{i}^{(s)}},\mspace{9mu}{t_{i}^{(m)} = t_{i}^{(s)}},\qquad x\in\partial{\Omega_{\mathit{\text{ms}}} = \partial}{\Omega_{m} \cap \partial}\Omega_{s},\mspace{9mu}{s > m},}{}\](4.2)

\textsubscript{та граничні умови}

\[{{u_{i}^{(m)} = u_{\mathit{\text{ig}}}^{(m)}},\mspace{9mu} x\in\partial\Omega_{m}^{1},\mspace{9mu}{t_{i}^{(m)} = t_{\mathit{\text{ig}}}^{(m)}},\mspace{9mu} x\in\partial\Omega_{m}^{2},}{}\]\includegraphics[width=1.722in,height=0.2638in]{Pictures/100003FC000069D500001037FC66A6B2E8D9D5EA.wmf}

\[{{u_{2}^{(m)} = u_{2g}^{(m)}},\mspace{9mu}{t_{1}^{(m)} = t_{1g}^{(m)}},\mspace{9mu} x\in\partial\Omega_{m}^{4},}{}\]\[{{}_{j = 1}^{4}\partial{\Omega_{m}^{j} = \partial}{\Omega \cap \partial}\Omega_{m}\text{.}}{}\]\textsubscript{}(4.3)

Тут
\[{{t_{i}^{(m)} = \sigma_{\mathit{\text{ij}}}^{(m)}}n_{j}^{(m)},}{}\]\[\sigma_{\mathit{\text{ij}}}^{(m)}{}\]--
компоненти відповідно век­торів поверхневих зу­силь та тензора
напру­жень, \[n_{j}^{(m)}{}\] -- компоненти одиничного вектора
зов­ніш­ньої нормалі \[n^{(m)}{}\]\textsubscript{ }до границі
\[{\partial\Omega_{m}}{}\].

\hypertarget{ux456ux43dux442ux435ux433ux440ux430ux43bux44cux43dux435-ux437ux43eux431ux440ux430ux436ux435ux43dux43dux44f-ux440ux43eux437ux432ux44fux437ux43aux443}{%
\subsection[4.2. Інтегральне зображення
розв'язку]{\texorpdfstring{\protect\hypertarget{anchor-55}{}{}4.2.
Інтегральне зображення
розв'язку}{4.2. Інтегральне зображення розв'язку}}\label{ux456ux43dux442ux435ux433ux440ux430ux43bux44cux43dux435-ux437ux43eux431ux440ux430ux436ux435ux43dux43dux44f-ux440ux43eux437ux432ux44fux437ux43aux443}}

Для побудови розв'язків задачі (4.1)-(4.3) застосуємо МГЕ, а також МПГЕ,
щоб провести порівняння результатів. Уведемо множину \[{R^{2}(M)}{}\],
складену з \emph{M} просторів \[R^{2}{}\], кожен з яких надалі
позначатимемо через \[R_{m}^{2}{}\], тобто
\[{R^{2}(M{) = {}_{m = 1}^{M}}R_{m}^{2}}{}\]. Постулюватимемо, що
\[{R^{2}(M)}{}\] володіє такими властивостями
{[}\protect\hyperlink{anchor-24}{14}{]}:
\[{{{R_{m}^{2} \cap R^{2}} = {\Omega^{m} \cup \partial}}\Omega^{m}}{}\],
\includegraphics[width=1.25in,height=0.2362in]{Pictures/10000280000069D5000013FE4BA2CD78804701F8.wmf},
\[{{{R_{m}^{2} \cap R_{s}^{2}} = \partial}\Omega^{\mathit{\text{ms}}}}{}\].

Для кожного \emph{m }розв'язок \[u^{(m)}{}\] шукаємо в \[R_{m}^{2}{}\].
З цією метою розглянемо такі області \[{B_{m}\subset R_{m}^{2}}{}\], що
\[{\Omega_{m}\subset B_{m},\quad\partial{\Omega_{m} \cap \partial}{B_{m} = \varnothing}}{}\].
У приграничних областях \[{{G^{m} = B_{m}}{}_{m}}{}\] і на краях
\[{\partial\Omega_{m}}{}\]\textsubscript{ }уведемо відповідно
пригра­нич­нi \[G_{v}^{m}{}\] та граничні
\[{{\Gamma_{v}^{m} = \partial}{G_{v}^{m} \cap \partial}\Omega_{m}\mspace{9mu}}{}\]\[{({v = 1},\text{.}\text{.}\text{.},V_{m})}{}\]\textsubscript{
}елементи з невiдомими компонентами ``фiктивних'' ма­сових сил
\[{\phi_{\mathit{\text{iv}}}^{\mathit{\gamma m}}(x)}{}\], при­чому
\[{\gamma\in{\{{G,\Gamma}\}},}{}\]\[{\mathit{\text{mes}}\mspace{9mu}{G_{v}^{m} = 2}}{}\],
\[{{}_{v = 1}^{V_{m}}G_{v}{= G}}{}\],
\[{{{G_{v}^{m} \cap G_{w}^{m}} = \varnothing},}{}\]
\[{{{\Gamma_{v}^{m} \cap \Gamma_{w}^{m}} = \varnothing},}{}\] при
\[{v \neq w}{}\],
\[{{}_{v = 1}^{V_{m}}\Gamma_{v}^{m}{{= \partial}\Omega_{m}}}{}\].

Тоді замість рівняння (4.1) для \[{x\in R_{m}^{2}}{}\] матимемо

\[{\mu_{m}{u_{i,\mathit{\text{jj}}}^{(\mathit{\gamma m})} + \mu_{m}}{u_{j,\mathit{\text{ji}}}^{(\mathit{\gamma m})}/(}{1 - 2}\nu_{m}{) = {{- {\sum_{v = 1}^{V_{m}}{\phi_{\mathit{\text{iv}}}^{\mathit{\gamma m}}(x)\chi_{v}^{\mathit{\gamma m}}}}} - \psi_{i}^{(m)}}}\text{.}}{}\]\textsubscript{}(4.4)

Інтегральні зображення розв'язку рівнянь (4.4), а також одержані на його
основі інтегральні зображення компонент вектора поверхневих зусиль,
тензорів деформацій і напружень запишемо так:

\[{{u_{i}^{(\mathit{\gamma m})} = F_{i}^{\mathit{\gamma m}}}(x,U_{\mathit{\text{ij}}}^{m}{) + C_{i}^{\mathit{\gamma m}}},\mspace{54mu}{t_{i}^{(\mathit{\gamma m})} = F_{i}^{\mathit{\gamma m}}}(x,T_{\mathit{\text{ij}}}^{m}),\quad}{}\]\textsubscript{}(4.5)

\[{{\varepsilon_{e}^{(\mathit{\gamma m})} = F_{e}^{\mathit{\gamma m}}}(x,E_{\mathit{\text{ek}}}^{m}),\mspace{54mu}{\sigma_{e}^{(\mathit{\gamma m})} = F_{e}^{\mathit{\gamma m}}}(x,S_{\mathit{\text{ek}}}^{m}),}{}\]\textsubscript{}(4.6)

де
\[{F_{a}^{\mathit{\gamma m}}(x,\Phi_{\mathit{\text{aj}}}{) =}}{}\]\[{\sum_{v = 1}^{V_{m}}{\int_{\gamma_{v}^{m}}{\Phi_{\mathit{\text{aj}}}{(x,\xi)\phi_{\mathit{\text{jv}}}^{\mathit{\gamma m}}}{(\xi)\mathit{d\gamma}_{v}^{m}}{(\xi{) + {\int_{\Omega_{m}}{\Phi_{\mathit{\text{aj}}}{(x,\xi)\psi_{j}^{(m)}}{(\xi)d\Omega_{m}}{(\xi)}}}}}}},}{}\]

\[{U_{\mathit{\text{ij}}}^{m}(x,\xi{) = D_{1m}}(D_{2m}\delta_{\mathit{\text{ij}}}\text{ln}{r - y_{i}}y_{j}r^{- 2}),\mspace{9mu}{D_{1m} = {( - 8}}\mathit{\text{πμ}_{\mathrm{m}}}({1 - \nu_{m}}))^{- 1},\mspace{9mu}{D_{2m} = {3 - 4}}\nu_{m},}{}\]
\[{{r^{2} = {y_{1}^{2} + y_{2}^{2}}},}{}\]

\[{{E_{\mathit{\text{ek}}}^{m} = E_{\mathit{\text{ijk}}}^{m} = 0,5}({U_{\mathit{\text{ik}},j}^{m} + U_{\mathit{\text{jk}},i}^{m}}),\mspace{9mu}{S_{\mathit{\text{ek}}}^{m} = S_{\mathit{\text{ijk}}}^{m} = 2}\mu_{m}(\nu_{m}({1 - 2}\nu_{m})^{- 1}({E_{\text{11}k}^{m} + E_{\text{22}k}^{m}}){\delta_{\mathit{\text{ij}}} + E_{\mathit{\text{ijk}}}^{m}}),}{}\]\[{{e = 1,2,3},}{}\]\emph{
k}=1,2,

\[{T_{\mathit{\text{ij}}}^{m}(x,\xi{) = S_{\mathit{\text{ijk}}}^{m}}(x,\xi)n_{k}^{(m)},}{}\]\[{{y_{i} = {x_{i} - \xi_{i}}},\quad{\xi = (}\xi_{1},\xi_{2})\in R^{2},}{}\]\emph{}\textsubscript{\emph{ij}}
символ Кронекера, \emph{}\textsubscript{\emph{ij}} - компоненти
тензора деформацій,\emph{
}\[{{\eta_{1} = \eta_{\text{11}}},\mspace{9mu}{\eta_{2} = \eta_{\text{22}}},\mspace{9mu}{\eta_{3} = \eta_{\text{12}}},\mspace{9mu}\eta\in{\{{\varepsilon,\sigma}\}},}{}\]\textsubscript{
}\[\chi_{v}^{\mathit{\gamma m}}{}\]  характе­ристична функція елемента
\[\gamma_{v}^{m}{}\].

\hypertarget{ux43fux43eux431ux443ux434ux43eux432ux430-ux43bux456ux43dux456ux439ux43dux43eux457-ux441ux438ux441ux442ux435ux43cux438-ux440ux456ux432ux43dux44fux43dux44c-ux434ux43bux44f-ux432ux438ux437ux43dux430ux447ux435ux43dux43dux44f-ux43dux435ux432ux456ux434ux43eux43cux438ux445-ux43cux430ux441ux43eux432ux438ux445-ux441ux438ux43b}{%
\subsection[4.3. Побудова лінійної системи рівнянь для визначення
невідомих масових
сил]{\texorpdfstring{\protect\hypertarget{anchor-56}{}{}4.3. Побудова
лінійної системи рівнянь для визначення невідомих масових
сил}{4.3. Побудова лінійної системи рівнянь для визначення невідомих масових сил}}\label{ux43fux43eux431ux443ux434ux43eux432ux430-ux43bux456ux43dux456ux439ux43dux43eux457-ux441ux438ux441ux442ux435ux43cux438-ux440ux456ux432ux43dux44fux43dux44c-ux434ux43bux44f-ux432ux438ux437ux43dux430ux447ux435ux43dux43dux44f-ux43dux435ux432ux456ux434ux43eux43cux438ux445-ux43cux430ux441ux43eux432ux438ux445-ux441ux438ux43b}}

Застосовуючи (4.5) для задоволення умов (4.2), (4.3) на границях розділу
середовищ та на \[{\partial \Omega}{}\] i враховуючи, що рівнодіюча
всiх сил вздовж осі \emph{Ox}\textsubscript{\emph{i}}\emph{
}в\[R_{m}^{2}{}\] рівна нулю, одержи­мо спiввiдношення, якi зв'язують
невiдомi
\[{\phi_{\mathit{\text{vj}}}^{\mathit{\gamma m}},\mspace{9mu} C_{i}^{\mathit{\gamma m}}}{}\]
із заданими на краю області функціями та відомими масовими силами.

Загалом кажучи, явно проiнте­гру­ва­ти вказані спiввiд­ношення у
реальних прикладних задачах теорії пружності є доволі складно або й
неможливо. Тому скористуємося числовими методами. Найпростiше здiйснити
перехід до дискретного аналогу так.

1. Кожну невiдому функцiю
\[{\phi_{\mathit{\text{vj}}}^{\mathit{\gamma m}}(x)}{}\] замiнимо
невiдомою константою \[d_{\mathit{\text{vj}}}^{\mathit{\gamma m}}{}\].

2. Для задоволення граничних та контактних умов використаємо поточкову
колокацiю в серединi кожного граничного елемента.

3. Області \[\Omega_{m}{}\] дискретизуємо внутрiшнiми комiрками
\[{\Omega_{\mathit{\text{mq}}}\mspace{9mu}({q = \overline{1,Q_{m}}})\text{.}}{}\]

Тодi система лiнiйних алгебричних рiвнянь для знаходження невідо­мих
констант \[d_{\mathit{\text{vj}}}^{\mathit{\gamma m}}{}\] та
\[C_{i}^{\mathit{\gamma m}}{}\] матиме вигляд:

\[{{\sum_{v = 1}^{V_{m}}d_{\mathit{\text{jv}}}^{\mathit{\gamma m}}}{\int_{\gamma_{v}^{m}}{U_{\mathit{\text{ij}}}^{m}{(x^{w}}{,\xi)\mathit{d\gamma}_{v}^{m}}{(\xi)}}}{{- {\sum_{v = 1}^{V_{s}}d_{\mathit{\text{jv}}}^{\mathit{\gamma s}}}}{\int_{\gamma_{v}^{s}}{U_{\mathit{\text{ij}}}^{s}{(x^{w}}{,\xi)\mathit{d\gamma}_{v}^{s}}{(\xi)}}}}{+ C_{i}^{\mathit{\gamma m}}}{- C_{i}^{\mathit{\gamma s}}} =}{}\]\[{{= {\sum\limits_{q = 1}^{Q_{s}}{\int\limits_{\Omega_{\mathit{\text{sq}}}}{U_{\mathit{\text{ij}}}^{s}(x^{w},\xi)\psi_{j}^{(s)}(\xi)d\Omega_{\mathit{\text{sq}}}(\xi)}}}}{- {\sum\limits_{q = 1}^{Q_{m}}{\int\limits_{\Omega_{\mathit{\text{mq}}}}{U_{\mathit{\text{ij}}}^{m}{(x^{w}}{,\xi)\psi_{j}^{(m)}}{(\xi)d\Omega_{\mathit{\text{mq}}}}{(\xi),}}}}}{\mspace{9mu} x^{w}}{\in\partial\Omega_{\mathit{\text{ms}}}}{,{s > m},}}{}\]

\[{{\sum_{v = 1}^{V_{m}}d_{\mathit{\text{jv}}}^{\mathit{\gamma m}}}{\int_{\gamma_{v}^{m}}{T_{\mathit{\text{ij}}}^{m}{(x^{w}}{,\xi)\mathit{d\gamma}_{v}^{m}}{(\xi)}}}{{- {\sum_{v = 1}^{V_{s}}d_{\mathit{\text{jv}}}^{\mathit{\gamma s}}}}{\int_{\gamma_{v}^{s}}{T_{\mathit{\text{ij}}}^{s}{(x^{w}}{,\xi)\mathit{d\gamma}_{v}^{s}}{(\xi)}}}} =}{}\]\[{{= {\sum\limits_{q = 1}^{Q_{s}}{\int\limits_{\Omega_{\mathit{\text{sq}}}}{T_{\mathit{\text{ij}}}^{s}(x^{w},\xi)\psi_{j}^{(s)}(\xi)d\Omega_{\mathit{\text{sq}}}(\xi)}}}}{- {\sum\limits_{q = 1}^{Q_{m}}{\int\limits_{\Omega_{\mathit{\text{mq}}}}{T_{\mathit{\text{ij}}}^{m}{(x^{w}}{,\xi)\psi_{j}^{(m)}}{(\xi)d\Omega_{\mathit{\text{mq}}}}{(\xi),}}}}}{\mspace{9mu} x^{w}}{\in\partial\Omega_{\mathit{\text{ms}}}}{,{s > m},}}{}\]

\[{{\sum\limits_{v = 1}^{V_{m}}d_{\mathit{\text{jv}}}^{\mathit{\gamma m}}}{\int\limits_{\gamma_{v}^{m}}{U_{\mathit{\text{ij}}}^{m}{(x^{w}}{,\xi)\mathit{d\gamma}_{v}^{m}}}}{+ C_{i}^{m}}{= u_{\mathit{\text{ig}}}^{(m)}}{(x^{w}}{) - {\sum\limits_{q = 1}^{Q_{m}}{\int\limits_{\Omega_{\mathit{\text{mq}}}}{U_{\mathit{\text{ij}}}^{m}{(x^{w}}{,\xi)\psi_{j}^{(m)}}{(\xi)d\Omega_{\mathit{\text{mq}}}}{,x^{w}}{\in\partial\Omega_{m}^{1}}{\partial\Omega_{m}^{3}}{\partial\Omega_{m}^{4}}}}}},}{}\]

\[{{\sum\limits_{v = 1}^{V_{m}}d_{\mathit{\text{jv}}}^{\mathit{\gamma m}}}{\int\limits_{\gamma_{v}^{m}}{T_{\mathit{\text{ij}}}^{m}{(x^{w}}{,\xi)\mathit{d\gamma}_{v}^{m}}{(\xi)}}}{= t_{\mathit{\text{ig}}}^{(m)}}{(x^{w}}{) - {\sum\limits_{q = 1}^{Q_{m}}{\int\limits_{\Omega_{\mathit{\text{mq}}}}{T_{\mathit{\text{ij}}}^{m}{(x^{w}}{,\xi)\psi_{j}^{(m)}}{(\xi)d\Omega_{\mathit{\text{mq}}}}{(\xi),x^{w}}{\in\partial\Omega_{m}^{2}}{\partial\Omega_{m}^{3}}{\partial\Omega_{m}^{4}}}}}},}{}\]

\[{{\sum\limits_{v = 1}^{V_{m}}d_{\mathit{\text{iv}}}^{\mathit{\gamma m}}}{\int\limits_{\gamma_{v}^{m}}{\mathit{d\gamma}_{v}^{m}{(\xi)}}}{= {- {\sum\limits_{q = 1}^{Q_{m}}{\int\limits_{\Omega_{\mathit{\text{mq}}}}{\psi_{i}^{(m)}{(\xi)d\Omega_{\mathit{\text{mq}}}}{(\xi)\text{.}}}}}}}}{}\]

Пiдставивши одержанi як розв'язок останньої системи значення
\[d_{\mathit{\text{vj}}}^{\mathit{\gamma m}}{}\] та
\[C_{i}^{\mathit{\gamma m}}{}\] в (4.5), (4.6) знайдемо розв'язок
крайової задачi (4.4), (4.2), (4.3) для точок спостереження, як
внутрiшнiх, включаючи дiлянки контакту, так i на границі тіла.
Зауважимо, що інтеграли по \[\Gamma_{v}^{m}{}\] від функцій
\[T_{\mathit{\text{ij}}}^{m}{}\] при \[{x^{w} = \xi^{v}}{}\]
обчислюються в сенсі Коші.

\hypertarget{ux43fux43eux431ux443ux434ux43eux432ux430-ux43bux456ux43dux456ux439ux43dux43eux457-ux441ux438ux441ux442ux435ux43cux438-ux440ux456ux432ux43dux44fux43dux44c-ux434ux43bux44f-ux432ux438ux437ux43dux430ux447ux435ux43dux43dux44f-ux43dux435ux432ux456ux434ux43eux43cux438ux445-ux43cux430ux441ux43eux432ux438ux445-ux441ux438ux43b-1}{%
\subsection[4.4. Побудова лінійної системи рівнянь для визначення
невідомих масових
сил]{\texorpdfstring{\protect\hypertarget{anchor-57}{}{}4.4. Побудова
лінійної системи рівнянь для визначення невідомих масових
сил}{4.4. Побудова лінійної системи рівнянь для визначення невідомих масових сил}}\label{ux43fux43eux431ux443ux434ux43eux432ux430-ux43bux456ux43dux456ux439ux43dux43eux457-ux441ux438ux441ux442ux435ux43cux438-ux440ux456ux432ux43dux44fux43dux44c-ux434ux43bux44f-ux432ux438ux437ux43dux430ux447ux435ux43dux43dux44f-ux43dux435ux432ux456ux434ux43eux43cux438ux445-ux43cux430ux441ux43eux432ux438ux445-ux441ux438ux43b-1}}

Для числових досліджень області \[{\Omega,\Omega_{2}}{}\] вибирались
так:
\[{\Omega = {\{{(x_{1},x_{2}):{0 < x_{1} < 3},\quad{0 < x_{2} < 2}}\}}}{}\],
\[{{\Omega_{2} = {\{{(x_{1},x_{2}):1\text{.}{{5 - a_{2}} < x_{1} < 1}\text{.}{5 + a_{2}},\quad 1\text{.}{{5 - a_{2}} < x_{2} < 1}\text{.}{5 + a_{2}}}\}}},}{}\]
\[{{\Omega_{1} = \Omega}{}_{2}}{}\]. Функції
\[{u_{\mathit{\text{ig}}}^{(m)},t_{\mathit{\text{ig}}}^{(m)}}{}\],
задані на краю ∂Ω, мали вигляд: \[{{t_{1g}^{(1)} = 0},}{}\]
\[{{t_{2g}^{(1)} = 0},}{}\]
\[{x\in{\Gamma_{1} = {\{{(x_{1},x_{2}):{0 < x_{1} < 3},\quad{x_{2} = 0}}\}}}}{}\],

\[{{u_{1g}^{(1)} = 0},}{}\]
\[{{u_{2g}^{(1)} = 0}\text{.}\text{01}({x_{2} - 1}),}{}\]\[{x\in{\Gamma_{2} \cup \Gamma_{4}},{\Gamma_{2} = {\{{(x_{1},x_{2}):{x_{1} = 3},\mspace{9mu}{0 < x_{2} < 2}}\}}}}{}\],\[{\Gamma_{4} = {\{{(x_{1},x_{2}):{x_{1} = 0},\mspace{9mu}{0 < x_{2} < 2}}\}}}{}\],

\[{{t_{1g}^{(1)} = 0},}{}\] \[{{u_{2g}^{(1)} = 0}\text{.}\text{01},}{}\]
\[{x\in{\Gamma_{3} \cap \partial}\Omega_{1}}{}\],
\[{{t_{2g}^{(1)} = 0},}{}\] \[{{u_{2g}^{(2)} = 0}\text{.}\text{01},}{}\]
\[{x\in{\Gamma_{3} \cap \partial}\Omega_{2}}{}\],
\[{\Gamma_{3} = {\{{(x_{1},x_{2}):{0 < x_{1} < 3},\quad{x_{2} = 2}}\}}}{}\].

Задачу розв'язували з допомогою МГЕ та МПГЕ при використанні повної
дискретизації приграничних областей \[G^{m}{}\], тобто коли
\[{{}_{v = 1}^{V_{m}}G_{v}{= G}}{}\],
\[{{G_{v}^{m} \cap G_{w}^{m}} = \varnothing}{}\] при \[{v \neq w}{}\].
Порівню­вали точність розв'язків при задоволенні контактних (рис. 4.1)
та граничних (рис.4.2) умов. На рис. 4.1, 4.2 зображено величини похибок
\[{\mathit{\delta u}_{1}^{\gamma} = {u_{1}^{(\gamma 2)} - u_{1}^{(\gamma 1)}}}{}\],
\[{\mathit{\delta u}_{2}^{\gamma} = {u_{2}^{(\gamma 2)} - u_{2}^{(\gamma 1)}}}{}\]
на лінії контакту \[{\partial\Omega_{\text{12}}}{}\] та на краю
\[{\partial\Omega}{}\]. Як бачимо, при збільшенні ширини приграничної
смуги \emph{h }точність розв'язків, одержаних МПГЕ, зростає і при
\[{{h = a_{2} = 0}\text{.}5}{}\] стає співмірною з точністю, яку
забезпечує МГЕ для задоволення умов контакту, а при задоволенні
граничних умов МПГЕ виявився точнішим, ніж МГЕ, для будь-якого значення
\emph{h}. Тому усі наступні обчислення проводились для МПГЕ з
\[{{h = 0}\text{.}5}{}\]. Однак варто зауважити, що при використанні
дискретизації з перекриттям {[}\protect\hyperlink{anchor-58}{26}{]}
приграничної області \[G^{1}{}\], тобто коли
\[{{G_{v}^{1} \cap G_{w}^{2}} \neq \varnothing}{}\] при
\[{v \neq w}{}\], точність виконання умов контакту за допомогою МПГЕ
була вищою і при менших значеннях \emph{h}.

На рис. 4.3 зображено ізолінії, отримані для переміщень
\[u_{2}^{(\Gamma)}{}\] та \[u_{2}^{(G)}{}\], а на рис. 4.4, 4.5 --
ізолінії компонент тензора деформацій
\[{\varepsilon_{\text{11}}^{(G)},\mspace{9mu}\varepsilon_{\text{12}}^{(G)},\mspace{9mu}\varepsilon_{\text{22}}^{(G)}}{}\]
та компонент тензора напружень
\[{\sigma_{\text{11}}^{(G)},\mspace{9mu}\sigma_{\text{12}}^{(G)},\mspace{9mu}\sigma_{\text{22}}^{(G)}}{}\]
відповідно у випадку, коли матеріли областей мали характеристики
\[{{\mu_{1} = \text{8076}}\text{.}9,\mspace{9mu}{\nu_{1} = 0}\text{.}3}{}\]\textsubscript{
}(сталь конструкційна),
\[{{\mu_{2} = \text{4198}}\text{.}\text{47},\mspace{9mu}{\nu_{2} = 0}\text{.}\text{31}}{}\]
(мідь), де
\[{{f^{(\gamma)} = {\sum_{m = 1}^{2}f^{(\mathit{\gamma m})}}}\chi_{m}}{}\],
\[\chi_{m}{}\] -- характеристична функція області \[\Omega_{m}{}\]. Як
бачимо, найбільш інформативними є головні деформації та напруження (а не
зсувні), оскільки саме за їх концентрацією можна виділити межу розділу
середовищ

\includegraphics[width=2.2354in,height=1.4854in]{./ObjectReplacements/Obj683}\includegraphics[width=2.2354in,height=1.4571in]{./ObjectReplacements/Obj684}

аб

Рис. 4.1. Графіки залежності похибок
\[\mathit{\delta u}_{1}^{\Gamma}{}\],
\[\mathit{\delta u}_{2}^{\Gamma}{}\] (криві 1) та
\[\mathit{\delta u}_{1}^{G}{}\], \[\mathit{\delta u}_{2}^{G}{}\] при
\[{{h = 0}\text{.}3,\mspace{9mu} 0\text{.}5}{}\](криві 2,3), \emph{s }--
довжина частини лінії контакту від точки (1;2) до точки спосте­реження.

\includegraphics[width=2.35in,height=1.5571in]{./ObjectReplacements/Obj690}\includegraphics[width=2.35in,height=1.5571in]{./ObjectReplacements/Obj691}

аб

Рис. 4.2. Графіки залежності похибок
\[\mathit{\delta u}_{1}^{\Gamma}{}\],
\[\mathit{\delta u}_{2}^{\Gamma}{}\] (криві 1) та
\[\mathit{\delta u}_{1}^{G}{}\], \[\mathit{\delta u}_{2}^{G}{}\] при
\[{{h = 0}\text{.}3,\mspace{9mu} 0\text{.}5}{}\](криві 2,3), \emph{l }--
довжина частини краю \[{\partial\Omega}{}\] від точки (0;0) до точки
спосте­реження.

\includegraphics[width=2.3752in,height=1.7638in]{Pictures/10000000000001B200000142518DBD8948CAB6A2.png}
\includegraphics[width=2.5835in,height=1.778in]{Pictures/10000000000001920000011720055E2E8B56F310.png}

а б

Рис. 4.3. Ізолінії переміщень \[u_{2}^{(\Gamma)}{}\] (а) та
\[u_{2}^{(G)}{}\] (б).

\includegraphics[width=2.6252in,height=1.9307in]{Pictures/1000480C000069FC00004E166E0E1B59377515EE.wmf}\includegraphics[width=2.611in,height=1.9445in]{Pictures/10001E0E000069FC00004E1623230C7E2AF8F613.wmf}

а б

\includegraphics[width=2.611in,height=1.9307in]{Pictures/100060EC000069FF00004E37DACB50A76A3B7340.wmf}

в

Рис. 4.4. Ізолінії компонент тензора деформацій
\[\varepsilon_{\text{11}}^{(G)}{}\] (а),
\[\varepsilon_{\text{12}}^{(G)}{}\] (б),
\[\varepsilon_{\text{22}}^{(G)}{}\] (в).

\includegraphics[width=2.6252in,height=1.9307in]{Pictures/10003418000069FF00004E22715200E9DD9A7C3D.wmf}\includegraphics[width=2.6252in,height=1.9307in]{Pictures/10002B8A000069FF00004E22808F510A9216789D.wmf}

а б

\includegraphics[width=2.6252in,height=1.9307in]{Pictures/10004E16000069FF00004E22139C0A5442EA0870.wmf}

в

Рис. 4.5. Ізолінії компонент тензора напружень
\[\sigma_{\text{11}}^{(G)}{}\] (а), \[\sigma_{\text{12}}^{(G)}{}\] (б),
\[\sigma_{\text{22}}^{(G)}{}\] (в).

З метою вивчення впливу включення \[\Omega_{2}{}\] на область
\[\Omega_{1}{}\] проводили дослідження напружено-деформованого стану для
декількох значень \[{\mu_{2},\mspace{9mu}\nu_{2}}{}\] при фіксованих
значеннях \[{\mu_{1},\mspace{9mu}\nu_{1}}{}\]. На рис. 4.6 приведено
графіки переміщень \[u_{2}^{(G)}{}\] на лініях
\emph{x}\textsubscript{1}=1,5 і \emph{x}\textsubscript{1}=1,9 для різних
значень модулів зсуву та коефіцієн­тів Пуассона.

\includegraphics[width=2.75in,height=2.2071in]{./ObjectReplacements/Obj711}\includegraphics[width=2.7783in,height=2.1217in]{./ObjectReplacements/Obj712}

а б

\includegraphics[width=2.7783in,height=2.1in]{./ObjectReplacements/Obj713}

в

Рис. 4.6. Графіки \[u_{2}^{(G)}{}\] на лініях
\emph{x}\textsubscript{1}=1,5, \emph{x}\textsubscript{1}=1,9 і
\emph{x}\textsubscript{1}=2,1 у випадку
\[{{\mu_{1} = \text{8076}}\text{.}9,\mspace{9mu}{\nu_{1} = 0}\text{.}3}{}\]\textsubscript{
}для таких значень коефіцієнтів\textsubscript{
}\[{\mu_{2},\mspace{9mu}\nu_{2}}{}\]:

1 --
\[{{\mu_{2} = \text{4198}}\text{.}\text{47},\mspace{9mu}{\nu_{2} = 0}\text{.}\text{31}}{}\],
2 --
\[{{\mu_{2} = \text{12000}},\mspace{9mu}{\nu_{2} = 0}\text{.}\text{31}}{}\],

3
--\[{{\mu_{2} = \text{4198}}\text{.}\text{47},\mspace{9mu}{\nu_{2} = 0}\text{.}2}{}\],
4 --
\[{{\mu_{2} = \text{8076}}\text{.}9,\mspace{9mu}{\nu_{2} = 0}\text{.}4}{}\].

\hypertarget{ux43fux456ux434ux441ux443ux43cux43eux43a-2}{%
\subsection[4.5.
Підсумок]{\texorpdfstring{\protect\hypertarget{anchor-59}{}{}4.5.
Підсумок}{4.5. Підсумок}}\label{ux43fux456ux434ux441ux443ux43cux43eux43a-2}}

Проведені дослідження показали, що МПГЕ забезпечує вищу\emph{ }точність
розрахунків напружено-деформованого стану у порівнянні з МГЕ при
використанні однакової кількості елементів та однакового ступеня
апроксимації невідомих функцій ``фіктивних'' масових сил. Це
обґрунто­вується тим, що пригранична область згладжує вплив введених у
ній функцій.

Точність обчислень при знаходженні напружено-деформованого стану
залежить від вдалого вибору форми приграничних елементів та значення
висоти \[h{}\] області \[G^{m}{}\], натомість зміна шаблону інтегрування
суттєвого значення не мала. Рекомендації щодо вибору параметра \[h{}\]
можуть бути наступними: для модельних задач з відомими аналітичними
розв'язками варто виходити з апостеріорних оцінок отриманих числових
результатів, для інших -- порівнювати результати обчислень шуканих
функцій у певних ділянках краю із заданими на них граничними умовами, а
також точність задоволення умов контакту на лініях контакту.

Відзначимо, що зростання (спадання) модуля зсуву в області
\[\Omega_{2}{}\] спричинює вищі (нижчі) значення компоненти
\[u_{2}^{(G)}{}\], а зростання (спадання) коефіцієнта Пуасона -- менші
(більші) значення цієї компоненти при дослідженні кусково-однорідного
тіла.

Порівняльний аналіз застосування МГЕ та МПГЕ при знаходженні
напружено-деформованого стану у кусково-однорідних тілах свідчить про
можливість поєднання цих методів, зокрема, на краю області варто
використовувати приграничні елементи, а на лінії контакту -- можна
граничні. При цьому слід пам'ятати, що тоді необхідно створювати одне
програмне забезпечення (ПЗ) для інтегрування по внутрішніх і
приграничних елементах, а інше -- по граничних, а також те, що при
обчисленні поверхневих зусиль МГЕ з'являються інтеграли в сенсі Коші, а
це вимагає попереднього аналітичного виділення особливостей (головного
значення). Тому дослідникам, які прагнуть уніфікувати своє ПЗ та бажають
обмежитись тільки числовим інтегрування краще користуватись тільки МПГЕ.

Розділ 5

\hypertarget{ux437ux430ux434ux430ux447ux430-ux442ux435ux440ux43cux43eux43fux440ux443ux436ux43dux43eux441ux442ux456-ux432-ux43fux43bux43eux441ux43aux43eux43cux443-ux442ux456ux43bux456-ux456ux437-ux43bux43eux43aux430ux43bux44cux43dux43e-ux43dux435ux43eux434ux43dux43eux440ux456ux434ux43dux438ux43c-ux432ux43aux43bux44eux447ux435ux43dux43dux44fux43c}{%
\section[Задача термопружності в плоскому тілі із локально-неоднорідним
включенням]{\texorpdfstring{\protect\hypertarget{anchor-60}{}{}Задача
термопружності в плоскому тілі із локально-неоднорідним
включенням}{Задача термопружності в плоскому тілі із локально-неоднорідним включенням}}\label{ux437ux430ux434ux430ux447ux430-ux442ux435ux440ux43cux43eux43fux440ux443ux436ux43dux43eux441ux442ux456-ux432-ux43fux43bux43eux441ux43aux43eux43cux443-ux442ux456ux43bux456-ux456ux437-ux43bux43eux43aux430ux43bux44cux43dux43e-ux43dux435ux43eux434ux43dux43eux440ux456ux434ux43dux438ux43c-ux432ux43aux43bux44eux447ux435ux43dux43dux44fux43c}}

\hypertarget{ux43fux43eux441ux442ux430ux43dux43eux432ux43aux430-ux437ux430ux434ux430ux447}{%
\subsection[5.1. Постановка
задач]{\texorpdfstring{\protect\hypertarget{anchor-61}{}{}5.1.
Постановка
задач}{5.1. Постановка задач}}\label{ux43fux43eux441ux442ux430ux43dux43eux432ux43aux430-ux437ux430ux434ux430ux447}}

Розглянемо тіло, що займає область
\includegraphics[width=0.1354in,height=0.1217in]{./ObjectReplacements/Obj726}
із краєм
\includegraphics[width=0.1146in,height=0.1217in]{./ObjectReplacements/Obj727}.
Нехай коефіцієнт теплопровідності \[{\lambda_{t}(x)}{}\] та коефіцієнти
Ляме \[{\lambda(x),\mspace{9mu}\mu(x)}{}\] матеріалу тіла описуються
неперервними функціями, які набувають сталих значень скрізь у
\includegraphics[width=0.2217in,height=0.2071in]{./ObjectReplacements/Obj730},
за винятком локальних областей неоднорідності
\[{\Omega_{m}\subset\Omega}{}\] (\[{m = \overline{1,M}}{}\]), причому
\[{{\Omega_{m} \cap \Omega_{l}} = \varnothing}{}\](\[{m \neq l}{}\]), де
вони залежать від декартових координат \[{{x = (}x_{1},x_{2})}{}\], а
саме

\[{\lambda_{t}(x{) = {\lambda_{t0} + {\sum\limits_{m = 1}^{M}{\lambda_{\mathit{\text{tm}}}(x{) \cdot \chi_{m}}(x)}}}}}{}\],
\[{\mu(x{) = {\mu_{0} + {\sum\limits_{m = 1}^{M}{\mu_{m}(x{) \cdot \chi_{m}}(x)}}}}}{}\],
\[{\lambda(x{) = {\lambda_{0} + {\sum\limits_{m = 1}^{M}{\lambda_{m}(x{) \cdot \chi_{m}}(x)}}}}}{}\].(5.1)

Тут \[{\chi_{m}(x)}{}\] -- характеристична функція області
\[\Omega_{m}{}\]; \[\lambda_{t0}{}\],
\includegraphics[width=0.15in,height=0.1783in]{./ObjectReplacements/Obj742},
\includegraphics[width=0.15in,height=0.1783in]{./ObjectReplacements/Obj743}
-- коефіцієнти в \[{\Omega{}_{m = 1}^{M}\Omega_{m}}{}\];
\[{{\lambda_{t0} + \lambda_{\mathit{\text{tm}}}}(x)}{}\],
\[{{\lambda_{0} + \lambda_{m}}(x)}{}\],
\[{{\mu_{t0} + \mu_{\mathit{\text{tm}}}}(x)}{}\] -- коефіцієнти області
Ω\textsubscript{\emph{m}}, причому
\[{\lambda_{\mathit{\text{tm}}}(x){|_{\Gamma_{m}} = 0}}{}\],
\[{\lambda_{m}(x){|_{\Gamma_{m}} = 0}}{}\],
\[{\mu_{m}(x){|_{\Gamma_{m}} = 0}}{}\], де Г\textsubscript{\emph{m}} --
край області Ω\textsubscript{\emph{m}}.

Підставивши у рівняння теплопровідності та рівняння рівноваги в
переміщеннях вирази (5.1), співвідношення Дюгамеля-Неймана

\[{\sigma_{\mathit{\text{ij}}}(x{) = \lambda_{0}}\delta_{\mathit{\text{ij}}}\varepsilon_{\mathit{\text{kk}}}(x{) + 2}\mu_{0}\varepsilon_{\mathit{\text{ij}}}(x{) - \beta_{0}}{\theta + {{\sum\limits_{m = 1}^{M}{\lbrack\lambda_{m}(x)\delta_{\mathit{\text{ij}}}\varepsilon_{\mathit{\text{kk}}}(x{) + 2}\mu_{m}(x)\varepsilon_{\mathit{\text{ij}}}(x{) - \beta_{m}}(x)\theta(x)\rbrack}} \cdot \chi_{m}}}(x)}{}\].
\emph{i,j} = 1,2,

та співвідношеннями Коші, одержимо систему диференціальних рівнянь:

\[{\mathit{\Delta\theta} = {- f_{\theta}}}{}\],(5.2)

\[{\mu_{0}{u_{i,\mathit{\text{jj}}} + (}{\lambda_{0} + \mu_{0}}){u_{j,\mathit{\text{ji}}} = {{- f_{\mathit{\text{ij}},j}} + \mathit{\text{βθ}_{\mathrm{,i}}} + \beta_{,i}}}\theta}{}\],(5.3)

де \[\Delta{}\] -- оператор Лапласа,
\[{{f_{\theta} = f_{\theta}}(\theta_{,i}{) = {\sum\limits_{m = 1}^{M}{\chi_{m}\lambda_{\mathit{\text{tm}},i}{\theta_{,i}/\lambda_{\mathit{\text{tm}}}}}}}}{}\],
\[{{f_{\mathit{\text{ij}}} = f_{\mathit{\text{ij}}}}(\varepsilon{) = {\sum\limits_{m = 1}^{M}{\lbrack\lambda_{m}\delta_{\mathit{\text{ij}}}{\varepsilon_{\mathit{\text{kk}}} + 2}\mu_{m}\varepsilon_{\mathit{\text{ij}}}\rbrack\chi_{m}}}}}{}\];
\[{{\beta_{0} = (}2{\mu_{0} + 3}\lambda_{0}{) \cdot \alpha}}{}\],
\[{\beta_{m}(x{) = (}2\mu_{m}(x{) + 3}\lambda_{m}(x){) \cdot \alpha}}{}\];
\[{\beta(x{) = {\beta_{0} + {\sum\limits_{m = 1}^{M}{\beta_{m}(x{) \cdot \chi_{m}}(x)}}}}}{}\],
\includegraphics[width=0.1646in,height=0.2071in]{./ObjectReplacements/Obj760},
\includegraphics[width=0.1354in,height=0.2071in]{./ObjectReplacements/Obj761}
-- компоненти відповідно тензора напружень та тензора деформацій;
\includegraphics[width=0.1354in,height=0.1929in]{./ObjectReplacements/Obj762}
-- компоненти вектора переміщень; \[\theta{}\] -- стаціонарне
температурне поле; \[\alpha{}\] -- коефіцієнт теплового розширення
матеріалу;
\includegraphics[width=0.1646in,height=0.2071in]{./ObjectReplacements/Obj765}
-- символ Кронекера.

Вважатимемо, що на P\textsubscript{1}, P\textsubscript{2 } задані
температура й тепловий потік

\[{\theta{|_{P_{1}} = \theta_{p}}}{}\],
\[{q{|_{P_{2}} = q_{p}}}{}\](5.4)

а на P\textsubscript{1}, P\textsubscript{2 } -- переміщення та
поверхневі зусилля

\[{u_{i}{|_{P_{1}} = u_{\mathit{\text{pi}}}}}{}\],
\[{t_{i}{|_{P_{2}} = t_{\mathit{\text{pi}}}}}{}\], (5.5)

де \[{{P_{1} \cup P_{2}} = \Gamma}{}\],
\[{{P_{1} \cap P_{2}} = \varnothing}{}\],
\includegraphics[width=0.4854in,height=0.2071in]{./ObjectReplacements/Obj772},
\includegraphics[width=0.1354in,height=0.2071in]{./ObjectReplacements/Obj773}
-- компоненти одиничного вектора зовнішньої нормалі до краю
\[\Gamma{}\]. Отже, для визначення температурних переміщень
\includegraphics[width=0.1283in,height=0.1783in]{./ObjectReplacements/Obj775}
маємо незв'язану змішану крайову задачу (5.2)-(5.5).

Апроксимувавши компоненти невідомих похідної від температури
\[\theta_{,i}{}\] та тензора деформацій
\includegraphics[width=0.2071in,height=0.2783in]{./ObjectReplacements/Obj777}
в областях Ω\textsubscript{m} функціями \[\theta_{,i}^{m}{}\],
\[\varepsilon_{\mathit{\text{ij}}}^{m}{}\] і використавши їх у
\[f_{\theta}{}\],
\includegraphics[width=0.3071in,height=0.2783in]{./ObjectReplacements/Obj781},
перейдемо від крайової задачі відносно \[\theta{}\],
\includegraphics[width=0.1354in,height=0.1354in]{./ObjectReplacements/Obj783}
до крайової задачі відносно \[\theta^{}{}\],
\includegraphics[width=0.2071in,height=0.25in]{./ObjectReplacements/Obj785}

\[{\mathit{\Delta\theta}^{} = {- f_{\theta}}}{}\],(5.6)

\[{\theta^{}{|_{P_{1}} = \theta_{p}}}{}\],\emph{}\[{q^{}{|_{P_{2}} = q_{p}}}{}\],(5.7)

\[{\mu_{0}{u_{i,\mathit{\text{jj}}}^{} + (}{\lambda_{0} + \mu_{0}}){u_{j,\mathit{\text{ji}}}^{} = {- f_{\mathit{\text{ij}},j}}}(\varepsilon^{}{) + \mathit{\text{βθ}_{\mathrm{,i}}^{\mathrm{}}} + \beta_{,i}}\theta^{}}{}\],(5.8)

\[{u_{i}^{}{|_{P_{1}} = u_{\mathit{\text{pi}}}}}{}\],
\[{t_{i}^{}{|_{P_{2}} = t_{\mathit{\text{pi}}}}}{}\].(5.9)

\hypertarget{ux43cux435ux442ux43eux434-ux43fux43eux454ux434ux43dux430ux43dux43dux44f-ux433ux440ux430ux43dux438ux447ux43dux438ux445-ux435ux43bux435ux43cux435ux43dux442ux456ux432-ux456ux437-ux441ux43aux456ux43dux447ux435ux43dux438ux43cux438-ux435ux43bux435ux43cux435ux43dux442ux430ux43cux438-1}{%
\subsection[5.2. Метод поєднання граничних елементів із скінченими
елементами]{\texorpdfstring{\protect\hypertarget{anchor-62}{}{}5.2.
Метод поєднання граничних елементів із скінченими
елементами}{5.2. Метод поєднання граничних елементів із скінченими елементами}}\label{ux43cux435ux442ux43eux434-ux43fux43eux454ux434ux43dux430ux43dux43dux44f-ux433ux440ux430ux43dux438ux447ux43dux438ux445-ux435ux43bux435ux43cux435ux43dux442ux456ux432-ux456ux437-ux441ux43aux456ux43dux447ux435ux43dux438ux43cux438-ux435ux43bux435ux43cux435ux43dux442ux430ux43cux438-1}}

Для побудови розв'язків задачі (5.6)-(5.9) застосуємо МГЕ, а також МПГЕ
у поєднанні з ермітовими скінченними елементами в областях локальних
неоднорідностей. З цією метою розглянемо таку область
\[{B\subset R^{2}}{}\], що
\[{\Omega\subset B,\quad\partial{\Omega \cap \partial}{B = \varnothing}}{}\].
Позначимо приграничну область \[{G = B}{}\]\textsubscript{. }Розв'язки
рівнянь (5.6), (5.8) зобразимо у вигляді:

\includegraphics[width=4.0417in,height=0.5in]{Pictures/10000664000069D500000D3B73D83C73BD123AFD.wmf}(5.10)

\[{u_{i}^{\gamma}(X{) = {\int\limits_{\gamma}{E_{\mathit{\text{ij}}}(X,Y)\phi_{j}^{\gamma}(X)\mathit{d\gamma}(Y{) + {\sum\limits_{m = 1}^{M}{\int\limits_{\Omega_{m}}{E_{\mathit{\text{ij}}}(X,Y)f_{\mathit{\text{jk}},k}(\varepsilon^{\gamma_{m}}(Y))d\Omega_{m}{(Y{) -}}}}}}}}}}{}\]

\[{{- {\int\limits_{\Omega}{E_{\mathit{\text{ij}}}(X,Y)\lbrack\beta(Y)\theta_{,j}(Y{) + \beta_{,j}}(Y)\theta(Y)\rbrack d\Omega(Y)}}} + C_{i}}{}\],(5.11)

де \[{\gamma\in{\{{G,\Gamma}\}}\text{.}}{}\]\textsubscript{
}Застосувавши до (5.10), (5.11) теорему Гріна інтегрування частинами,
одержимо зображення для температури та компонент вектора переміщень:

\[{\theta^{\gamma}(X{) = {\int\limits_{\gamma}{U(X,Y)\phi_{\theta}^{\gamma}(X)\mathit{d\gamma}(Y{) - {\sum\limits_{m = 1}^{M}{{\int\limits_{\Omega_{m}}{U_{,i}(X,Y)f_{\theta}(\theta^{\gamma_{m}}(Y))d\Omega_{m}{(Y{) + C_{\theta}}}}},}}}}}}}{}\]
(5.12)

\[{u_{i}^{\gamma}(X{) = {\int\limits_{\gamma}{E_{\mathit{\text{ij}}}(X,Y)\phi_{j}^{\gamma}(X)\mathit{d\gamma}(Y{) - {\sum\limits_{m = 1}^{M}{\int\limits_{\Omega_{m}}{E_{\mathit{\text{ij}},k}(X,Y)f_{\mathit{\text{jk}}}(\varepsilon^{\gamma_{m}}(Y))d\Omega_{m}{(Y{) -}}}}}}}}}}{}\]

\[{{- {\int\limits_{\Gamma}{E_{\mathit{\text{ij}}}(X,Y)\beta(Y)\theta(Y)n_{j}(Y)\mathit{d\Gamma}(Y{) +}}}}{{\int\limits_{\Omega}{\theta(Y)\lbrack E_{\mathit{\text{ij}},j}(X,Y)\beta(Y{) - E_{\mathit{\text{ij}}}}(X,Y)\beta_{,j}(Y)\rbrack d\Omega(Y)}} + C_{i}}}{}\](5.13)

і на основі них інтегральні зображення теплового потоку та компонент
деформацій, напружень та поверхневих зусиль

\[{q^{\gamma}(X{) = {\int\limits_{\gamma}{Q(X,Y)\phi_{\theta}^{\gamma}(X)\mathit{d\gamma}(Y{) + {\sum\limits_{m = 1}^{M}{{\int\limits_{\Omega_{m}}{Q_{,i}(X,Y)f_{\theta}(\theta^{\gamma_{m}}(Y))d\Omega_{m}{(Y)}}},}}}}}}}{}\]
(5.14)

\[{\begin{matrix}
{\left\{ {} \right\}\left\{ {\varepsilon_{\mathit{\text{ij}}}^{\gamma}(X)} \right\}\left\{ {\sigma_{\mathit{\text{ij}}}^{\gamma}(X)} \right\}} \\
\end{matrix}{{\{\}} = {\int\limits_{\gamma}{\begin{matrix}
{\left\{ {} \right\}\left\{ {B_{\mathit{\text{ijk}}}(X,Y)} \right\}\left\{ {T_{\mathit{\text{ijk}}}(X,Y)} \right\}} \\
\end{matrix}{\{\}}\phi_{k}^{\gamma}(Y)\mathit{d\gamma}(Y{) - {\sum\limits_{m = 1}^{M}{\int\limits_{\Omega_{m}}{\begin{matrix}
{\left\{ {} \right\}\left\{ {B_{\mathit{\text{ijk}},l}(X,Y)} \right\}\left\{ {T_{\mathit{\text{ijk}},l}(X,Y)} \right\}} \\
\end{matrix}{\{\}}f_{\mathit{\text{kl}}}(\varepsilon^{\gamma_{m}}(Y))d\Omega_{m}{(Y)}}}}}}}}}{}\]
-

\[{{{- {\int\limits_{\Gamma}{\beta(Y)\theta(Y)}}}\begin{matrix}
{\left\{ {} \right\}\left\{ B_{\mathit{\text{ijk}}} \right\}\left\{ T_{\mathit{\text{ijk}}} \right\}} \\
\end{matrix}{\{\}}}n_{k}(Y)\mathit{d\Gamma}(Y{) + {\int\limits_{\Omega}{\theta(Y)\beta(Y)}}}\begin{matrix}
{\left\{ {} \right\}\left\{ B_{\mathit{\text{ijk}},k} \right\}\left\{ T_{\mathit{\text{ijk}},k} \right\}} \\
\end{matrix}{\{\}}d\Omega(Y{) - {\sum\limits_{m = 1}^{M}{\int\limits_{\Omega_{m}}{\theta(Y)}}}}\beta_{m,k}{(Y)}\begin{matrix}
{\left\{ {} \right\}\left\{ B_{\mathit{\text{ijk}}} \right\}\left\{ T_{\mathit{\text{ijk}}} \right\}} \\
\end{matrix}{\{\}}d\Omega_{m}}{(Y)}{{+ f_{\mathit{\text{ij}}}}{(\varepsilon^{\gamma_{m}}}{(X))}\begin{matrix}
{\left\{ {} \right\}\left\{ 0 \right\}\left\{ 1 \right\}} \\
\end{matrix}{\{\}}}{}{}\],(15)

де \[\] -- фундаментальний розв'язок стаціонарного рівняння
теплопровідності,
\[{E_{\mathit{\text{ij}}}(X,Y{) = E_{\mathit{\text{ij}}} = \frac{\lambda_{0} + \mu_{0}}{4\mathit{\text{πμ}_{\mathrm{0}}}({\lambda_{0} + \mu_{0}})}}\left\{ {{- \frac{{\lambda_{0} + 3}\mu_{0}}{\lambda_{0} + \mu_{0}}}\text{ln}{\mathit{r\delta}_{\mathit{\text{ij}}} + r_{,i}}r_{,j}} \right\}}{}\]
-- фундаментальний розв'язок системи рівнянь задачі пружності;
\[{{Y = (}Y_{1},Y_{2}),\mspace{9mu} Y_{1},Y_{2}}{}\]--- система
координат, що співпадає зі системою \[{x_{1},x_{2}}{}\] і
використовується для опису точки, в якій діють невідомі фіктивні джерела
тепла \[\phi_{\theta}^{\gamma}{}\] чи прикладені компоненти невідомих
фіктивних поверхневих зусиль \[\phi_{j}^{\Gamma}{}\] чи масових сил
\[\phi_{j}^{G}{}\];
\includegraphics[width=1.3071in,height=0.2354in]{./ObjectReplacements/Obj811};
\[C_{\theta}{}\],
\includegraphics[width=0.1929in,height=0.2354in]{./ObjectReplacements/Obj813}
невідомі сталі,
\[{B_{\mathit{\text{ijk}}}(X,Y{) = B_{\mathit{\text{ijk}}} = \frac{1}{2}}({E_{\mathit{\text{ik}},j} + E_{\mathit{\text{jk}},i}})}{}\],
\[{T_{\mathit{\text{jk}}}(X,Y{) = T_{\mathit{\text{jk}}} = \lambda_{0}}\delta_{\mathit{\text{ij}}}{B_{\mathit{\text{llk}}} + 2}\mu_{0}B_{\mathit{\text{ijk}}}}{}\],
\[{F_{\mathit{\text{ij}}}(X,Y{) = F_{\mathit{\text{ij}}} = T_{\mathit{\text{ijk}}}}n_{k}}{}\].

\hypertarget{ux434ux438ux441ux43aux440ux435ux442ux438ux437ux430ux446ux456ux44f-ux433ux435ux43eux43cux435ux442ux440ux456ux457-ux442ux456ux43bux430-1}{%
\subsection[5.3. Дискретизація геометрії
тіла]{\texorpdfstring{\protect\hypertarget{anchor-63}{}{}5.3.
Дискретизація геометрії
тіла}{5.3. Дискретизація геометрії тіла}}\label{ux434ux438ux441ux43aux440ux435ux442ux438ux437ux430ux446ux456ux44f-ux433ux435ux43eux43cux435ux442ux440ux456ux457-ux442ux456ux43bux430-1}}

У приграничній області \emph{G} і на ділянках краю
\[{P_{1},P_{2}}{}\]\textsubscript{ }уведемо відповідно пригра­нич­нi
\[G_{v}{}\] та граничні
\[{{\Gamma_{v} = \partial}{G_{v} \cap \Gamma}\mspace{9mu}}{}\]\[{({v = 1},\text{.}\text{.}\text{.},V)}{}\]\textsubscript{
}елементи, при­чому
\[{\mathit{\text{mes}}\mspace{9mu}{G_{\nu} = 2}}{}\],
\[{{}_{v = 1}^{V}{G_{v} = G}}{}\],
\[{{{G_{v} \cap G_{w}} = \varnothing},}{}\]
\[{{{\Gamma_{v} \cap \Gamma_{w}} = \varnothing},}{}\] при
\[{v \neq w}{}\], \[{{}_{v = 1}^{V_{1}}\Gamma_{v}{= P_{1}}}{}\],
\[{{}_{v = {V_{1} + 1}}^{V}\Gamma_{v}{= P_{2}}}{}\]. Геометрію
приграничних і граничних елементів моделюємо за допомогою вектора
\includegraphics[width=0.1783in,height=0.1783in]{./ObjectReplacements/Obj828}
інтерполяційних функцій у локальній системі координат. Вздовж кожного з
елементів \[\Gamma_{v}{}\] та по площі елемента
\[G_{v}{}\]\textsubscript{ }здійснимо апроксимацію невідомих функцій
\[\phi_{\theta}^{\gamma}{}\], \[\phi_{j}^{\gamma}{}\] за допомогою
інтерполяційних поліномів
\[{\phi_{\theta}^{\gamma}(\eta){|_{\gamma_{\nu}} = \psi^{T}}{(\eta{) \cdot \phi_{\mathit{\text{θν}}}^{\gamma}}}}{}\],
\[{\phi_{j}^{\gamma}(\eta){|_{\gamma_{\nu}} = \psi^{T}}{(\eta{) \cdot \phi_{\mathit{j\nu}}^{\gamma}}}}{}\],
де \[\phi_{\mathit{\theta v}}^{\gamma}{}\],
\[\phi_{\mathit{\text{jv}}}^{\gamma}{}\] --- вектори невідомих вузлових
апроксимацій функцій \[\phi_{\theta}^{\gamma}{}\],
\[\phi_{j}^{\gamma}{}\] на ν -ому елементі.

Області Ω\textsubscript{\emph{m}} дискретизуємо системою ермітових
скінчених елементів Ω\textsubscript{\emph{ms}}
(\emph{s}=\[\overline{1,S_{m}}{}\]), зобразивши функції
\[\theta^{\mathit{\gamma m}}{}\],
\[\varepsilon_{\mathit{\text{ij}}}^{\mathit{\gamma m}}{}\] на кожному з
них пробними функціями

\[{\theta^{\mathit{\gamma m}}{|_{\Omega_{\mathit{\text{ms}}}} = \xi^{T}}{(\zeta_{1}}{,\zeta_{2}}{) \cdot \theta^{\gamma\mathit{\text{ms}}}}}{}\],
\[{\varepsilon_{\mathit{\text{ij}}}^{\mathit{\gamma m}}{|_{\Omega_{\mathit{\text{ms}}}} = \xi^{T}}{(\zeta_{1}}{,\zeta_{2}}{) \cdot \varepsilon_{\mathit{\text{ij}}}^{\gamma\mathit{\text{ms}}}}}{}\],

де
\[{\theta^{\gamma\mathit{\text{ms}}},\mspace{9mu}\varepsilon_{\mathit{\text{ij}}}^{\gamma\mathit{\text{ms}}}}{}\]
--- вузлові вектори значень температури \[\theta^{\mathit{\gamma m}}{}\]
та компоненти
\[\varepsilon_{\mathit{\text{ij}}}^{\mathit{\gamma m}}{}\]тензора
деформацій на \emph{s}-тому елементі,
\includegraphics[width=0.1354in,height=0.2354in]{./ObjectReplacements/Obj847}
--- вектор базових функцій у локальній системі координат.

На основі таких апроксимацій одержимо дискретні аналоги формул
(5.12)-(5.15)

\[{\theta^{\gamma}(X{) = U^{\mathit{\text{γν}}}}(X{{) \cdot \phi_{\mathit{\text{θν}}}^{\gamma}} + U_{i}^{\mathit{\text{ms}}}}(X{{) \cdot \lambda_{t,i}^{\mathit{\text{ms}}}/\lambda_{t}^{\mathit{\text{ms}}}} + C_{\theta}}}{}\],(5.16)

\[{q^{\gamma}(X{) = Q^{\mathit{\text{γν}}}}(X{{) \cdot \phi_{\mathit{\text{θν}}}^{\gamma}} + Q_{i}^{\mathit{\text{ms}}}}(X{) \cdot \lambda_{t,i}^{\mathit{\text{ms}}}/\lambda_{t}^{\mathit{\text{ms}}}}}{}\],(5.17)

\[{u_{i}^{\gamma}(X{) = E_{\mathit{\text{ij}}}^{\mathit{\text{γν}}}}(X{{) \cdot \phi_{\mathit{j\nu}}^{\gamma}} + E_{i}^{\lambda\mathit{\text{ms}}}}(X{{) \cdot \varepsilon_{\mathit{\text{kk}}}^{\gamma\mathit{\text{ms}}}} + E_{\mathit{\text{ijl}}}^{\mu\mathit{\text{ms}}}}(X{{) \cdot \varepsilon_{\mathit{\text{jl}}}^{\gamma\mathit{\text{ms}}}} + C_{i}}}{}\],
(5.18)

\[{\begin{matrix}
{\left\{ \varepsilon_{\mathit{\text{ij}}}^{\gamma} \right\}\left\{ \sigma_{\mathit{\text{ij}}}^{\gamma} \right\}} \\
\end{matrix}{\{\}}(X{) =}\begin{matrix}
{\left\{ B_{\mathit{\text{ijk}}}^{\nu} \right\}\left\{ T_{\mathit{\text{ijk}}}^{\nu} \right\}} \\
\end{matrix}{\{\}}(X{{) \cdot \phi_{\mathit{k\nu}}^{\gamma}} +}\begin{matrix}
{\left\{ B_{\mathit{\text{ij}}}^{\lambda\mathit{\text{ms}}} \right\}\left\{ T_{\mathit{\text{ij}}}^{\lambda\mathit{\text{ms}}} \right\}} \\
\end{matrix}{\{\}}(X{{) \cdot \varepsilon_{\mathit{\text{kk}}}^{\gamma\mathit{\text{ms}}}} +}\begin{matrix}
{\left\{ B_{\mathit{\text{ijkl}}}^{\mu\mathit{\text{ms}}} \right\}\left\{ T_{\mathit{\text{ijkl}}}^{\mu\mathit{\text{ms}}} \right\}} \\
\end{matrix}{\{\}}(X{) \cdot \varepsilon_{\mathit{\text{kl}}}^{\gamma\mathit{\text{ms}}}}}{}\],
(5.19)

де

\[{\begin{Bmatrix}
U^{\Gamma_{v}} \\
Q^{\Gamma_{v}} \\
\end{Bmatrix}{(X{) = {\int\limits_{- 1}^{1}\begin{Bmatrix}
U^{\Gamma_{v}} \\
Q^{\Gamma_{v}} \\
\end{Bmatrix}}}}\psi^{T}J_{\mathit{\Gamma\nu}}\mathit{d\eta}}{}\],
\[{\begin{Bmatrix}
U^{\Gamma_{v}} \\
Q^{\Gamma_{v}} \\
\end{Bmatrix}{(X{) = {\int\limits_{- 1}^{1}\begin{Bmatrix}
U^{\Gamma_{v}} \\
Q^{\Gamma_{v}} \\
\end{Bmatrix}}}}\psi^{T}J_{\mathit{\Gamma\nu}}\mathit{d\eta}}{}\]

\[{\begin{matrix}
{\left\{ E_{\mathit{\text{ij}}}^{\Gamma_{v}} \right\}\left\{ B_{\mathit{\text{ijk}}}^{\mathit{\Gamma\nu}} \right\}\left\{ T_{\mathit{\text{ijk}}}^{\mathit{\Gamma\nu}} \right\}} \\
\end{matrix}{\{\}}(X{) = {\int\limits_{- 1}^{1}{\begin{matrix}
{\left\{ E_{\mathit{\text{ij}}} \right\}\left\{ B_{\mathit{\text{ijk}}} \right\}\left\{ T_{\mathit{\text{ijk}}} \right\}} \\
\end{matrix}{\{\}}}}}\psi^{T}J_{\mathit{\Gamma\nu}}\mathit{d\eta}}{}\];\[{\begin{matrix}
{\left\{ E_{\mathit{\text{ij}}}^{G_{v}} \right\}\left\{ B_{\mathit{\text{ijk}}}^{\mathit{G\nu}} \right\}\left\{ T_{\mathit{\text{ijk}}}^{\mathit{G\nu}} \right\}} \\
\end{matrix}{\{\}}(X{) = {\int\limits_{- 1}^{1}{{\int\limits_{- 1}^{1}{\begin{matrix}
{\left\{ E_{\mathit{\text{ij}}} \right\}\left\{ B_{\mathit{\text{ijk}}} \right\}\left\{ T_{\mathit{\text{ijk}}} \right\}} \\
\end{matrix}{\{\}}}}\psi^{T}J_{\mathit{G\nu}}\mathit{d\eta}_{1}\mathit{d\eta}_{2}}}}}{}\];

\[{\begin{matrix}
{\left\{ E_{i}^{\lambda\mathit{\text{ms}}} \right\}\left\{ B_{\mathit{\text{ij}}}^{\lambda\mathit{\text{ms}}} \right\}\left\{ T_{\mathit{\text{ij}}}^{\lambda\mathit{\text{ms}}} \right\}} \\
\end{matrix}{\{\}}(X{) = {- {\int\limits_{- 1}^{1}{{\int\limits_{- 1}^{1}{\begin{matrix}
{\left\{ E_{\mathit{\text{ij}},j} \right\}\left\{ B_{\mathit{\text{ijk}},k} \right\}\left\{ T_{\mathit{\text{ijk}},k} \right\}} \\
\end{matrix}{\{\}}\lambda_{\mathit{\text{ms}}}\xi^{T}H_{\mathit{\text{ms}}}\mathit{d\zeta}_{1}{\mathit{d\zeta}_{2} + \lambda_{\mathit{\text{ms}}}}}}\begin{matrix}
{\left\{ 0 \right\}\left\{ 0 \right\}\left\{ \delta_{\mathit{\text{ij}}} \right\}} \\
\end{matrix}{\{\}}\xi^{T}\chi_{\mathit{\text{ms}}}(X)}}}}}{}\],

\[{\begin{matrix}
{\left\{ E_{\mathit{\text{ijl}}}^{\mu\mathit{\text{ms}}} \right\}\left\{ B_{\mathit{\text{ijkl}}}^{\mu\mathit{\text{ms}}} \right\}\left\{ T_{\mathit{\text{ijkl}}}^{\mu\mathit{\text{ms}}} \right\}} \\
\end{matrix}{\{\}}(X{) = {- 2}}{\int\limits_{- 1}^{1}{{\int\limits_{- 1}^{1}{\begin{matrix}
{\left\{ E_{\mathit{\text{ij}},l} \right\}\left\{ B_{\mathit{\text{ijk}},k} \right\}\left\{ T_{\mathit{\text{ijk}},k} \right\}} \\
\end{matrix}{\{\}}\mu_{\mathit{\text{ms}}}\xi^{T}H_{\mathit{\text{ms}}}\mathit{d\zeta}_{1}{\mathit{d\zeta}_{2} + 2}\mu_{\mathit{\text{ms}}}}}\begin{matrix}
{\left\{ 0 \right\}\left\{ 0 \right\}\left\{ 1 \right\}} \\
\end{matrix}{\{\}}\xi^{T}\chi_{\mathit{\text{ms}}}(X)}}}{}\].

\[J_{\Gamma_{v}}{}\],\[J_{G_{v}}{}\],\[H_{\mathit{\text{ms}}}{}\] --
якобіани переходу від змінних \[\eta{}\], \[\eta_{1}{}\],
\[\eta_{2}{}\];\[\zeta_{1}{}\], \[\zeta_{2}{}\] до Х відповідно.
\[\chi_{\mathit{\text{ms}}}{}\]-- характеристична функція елемента
\[\Omega_{\mathit{\text{ms}}}{}\], \[\lambda_{\mathit{\text{ms}}}{}\],
\[\mu_{\mathit{\text{ms}}}{}\] -- значення функцій
\[{\lambda_{m}(X)}{}\], \[{\mu_{m}(X)}{}\] при
\[{X\in\Omega_{\mathit{\text{ms}}}}{}\].

\hypertarget{ux43fux43eux431ux443ux434ux43eux432ux430-ux441ux438ux441ux442ux435ux43cux438-ux440ux456ux432ux43dux44fux43dux44c-ux434ux43bux44f-ux437ux43dux430ux445ux43eux434ux436ux435ux43dux43dux44f-ux432ux443ux437ux43bux43eux432ux438ux445-ux437ux43dux430ux447ux435ux43dux44c-2}{%
\subsection[5.4. Побудова системи рівнянь для знаходження вузлових
значень]{\texorpdfstring{\protect\hypertarget{anchor-64}{}{}5.4.
Побудова системи рівнянь для знаходження вузлових
значень}{5.4. Побудова системи рівнянь для знаходження вузлових значень}}\label{ux43fux43eux431ux443ux434ux43eux432ux430-ux441ux438ux441ux442ux435ux43cux438-ux440ux456ux432ux43dux44fux43dux44c-ux434ux43bux44f-ux437ux43dux430ux445ux43eux434ux436ux435ux43dux43dux44f-ux432ux443ux437ux43bux43eux432ux438ux445-ux437ux43dux430ux447ux435ux43dux44c-2}}

Уведемо функції нев'язок
\[{R_{\mathit{\text{ij}}}^{\gamma\mathit{\text{ms}}}(x{) = \varepsilon_{\mathit{\text{ij}}}^{\gamma}}(x{) - \varepsilon_{\mathit{\text{ij}}}^{\gamma\mathit{\text{ms}}}}(x)}{}\]
на
\[{\underset{s = 1}{\overset{S_{m}}{}}\Omega_{\mathit{\text{ms}}}}{}\]
(\[{m = \overline{1,M}}{}\]),
\[{R_{1i}^{\gamma}(x{) = u_{i}^{\gamma}}(x{) - u_{\mathit{\text{pi}}}}(x)}{}\]
на \[P_{1}{}\] і
\[{R_{2i}^{\gamma}(x{) = t_{i}^{\gamma}}(x{) - t_{\mathit{\text{pi}}}}(x)}{}\]
на \[P_{2}{}\].

Унаслідок граничного переходу та дискретизації виразів для переміщень
(5.9) та зусиль (5.10), одержимо їх дискретні аналоги у точці
\[{X_{0}{\in\underset{\nu = 1}{\overset{V}{\cup}}\Gamma_{\nu}}}{}\]:

\[{u_{i}^{\gamma}(X_{0}{) = E_{\mathit{\text{ij}}}^{\mathit{\text{γν}}}}(X_{0}{{) \cdot \phi_{\mathit{j\nu}}^{\gamma}} + E_{i}^{\lambda\mathit{\text{ms}}}}(X_{0}{{) \cdot \varepsilon_{\mathit{\text{kk}}}^{\gamma\mathit{\text{ms}}}} + E_{\mathit{\text{ijl}}}^{\mu\mathit{\text{ms}}}}(X_{0}{{) \cdot \varepsilon_{\mathit{\text{jl}}}^{\gamma\mathit{\text{ms}}}} + C_{i}}}{}\],
(5.11)

\[{t_{i}^{\Gamma}(X_{0}{) = {\pm \frac{1}{2}}}\delta_{\mathit{\text{ij}}}\psi^{T}(X_{0}{{) \cdot \phi_{\mathit{j\nu}}^{\Gamma}} + F_{\mathit{\text{ij}}}^{\mathit{\Gamma\nu}}}(X_{0}{{) \cdot \phi_{\mathit{j\nu}}^{\Gamma}} + F_{i}^{\lambda\mathit{\text{ms}}}}(X_{0}{{) \cdot \varepsilon_{\mathit{\text{kk}}}^{\Gamma\mathit{\text{ms}}}} + F_{\mathit{\text{ijl}}}^{\mu\mathit{\text{ms}}}}(X_{0}{) \cdot \varepsilon_{\mathit{\text{jl}}}^{\Gamma\mathit{\text{ms}}}}}{}\],\[{t_{i}^{G}(X_{0}{) = F_{\mathit{\text{ij}}}^{\mathit{G\nu}}}(X_{0}{{) \cdot \phi_{\mathit{j\nu}}^{G}} + F_{i}^{\lambda\mathit{\text{ms}}}}(X_{0}{{) \cdot \varepsilon_{\mathit{\text{kk}}}^{\mathit{\text{Gms}}}} + F_{\mathit{\text{ijl}}}^{\mu\mathit{\text{ms}}}}(X_{0}{) \cdot \varepsilon_{\mathit{\text{jl}}}^{\mathit{\text{Gms}}}}}{}\](5.12)

причому точка X\textsubscript{0} така, що в ній існує єдина дотична до
\[{\underset{\nu = 1}{\overset{V}{}}\Gamma_{\nu}}{}\]; інтеграли
\[{F_{\mathit{\text{ij}}}^{\mathit{\Gamma\nu}}(X_{0})}{}\] слід розуміти
у сенсі головного значення Коші, а всі решта --- у сенсі Рімана.

Для визначення невідомих векторів вузлових значень фіктивних поверхневих
зусиль \[\phi_{\mathit{j\nu}}^{\Gamma}{}\] та масових сил
\[\phi_{\mathit{j\nu}}^{G}{}\], векторів вузлових значень компонент
тензора
\[\varepsilon_{\mathit{\text{ij}}}^{\gamma\mathit{\text{ms}}}{}\] в
областях неоднорідностей та вектора
\includegraphics[width=0.6783in,height=0.2929in]{./ObjectReplacements/Obj889}
побудуємо систему лінійних алгебричних рівнянь за методом зважених
нев'язок. Враховуємо для цього і дискретні аналоги

\begin{quote}
\[{{{W^{\mathit{\text{γν}}} \cdot \phi_{\mathit{j\nu}}^{\gamma}} + {W_{j}^{\lambda\mathit{\text{ms}}} \cdot \varepsilon_{\mathit{\text{kk}}}^{\gamma\mathit{\text{ms}}}} + {W_{l}^{\mu\mathit{\text{ms}}} \cdot \varepsilon_{\mathit{\text{jl}}}^{\gamma\mathit{\text{ms}}}}} = 0}{}\]
(5.13)
\end{quote}

умов

\[{\int\limits_{\gamma}{\phi_{j}^{\gamma}(Y)\mathit{d\gamma}(Y{) + {\sum\limits_{m = 1}^{M}{\int\limits_{\Omega_{m}}{f_{\mathit{\text{jk}},k}(\varepsilon^{\gamma_{m}}(Y))d\Omega_{m}{(Y{) = 0}}}}}}}}{}\],

де

\[{W^{\mathit{\Gamma\nu}} = {\int\limits_{- 1}^{1}{\psi^{T}J_{\mathit{\Gamma\nu}}\mathit{d\eta}}}}{}\],
\[{W^{\mathit{G\nu}} = {\int\limits_{- 1}^{1}{\int\limits_{- 1}^{1}{\psi^{T}J_{\mathit{G\nu}}\mathit{d\eta}_{1}\mathit{d\eta}_{2}}}}}{}\],
\[{W_{j}^{\lambda\mathit{\text{ms}}} = {\int\limits_{- 1}^{1}{\int\limits_{- 1}^{1}{\lambda_{\mathit{\text{ms}}}\xi^{T}H_{\mathit{\text{ms}}}\mathit{d\zeta}_{1}\mathit{d\zeta}_{2}}}}}{}\],
\[{{W_{j}^{\mu\mathit{\text{ms}}} = 2}{\int\limits_{- 1}^{1}{\int\limits_{- 1}^{1}{\mu_{\mathit{\text{ms}}}\xi^{T}H_{\mathit{\text{ms}}}\mathit{d\zeta}_{1}\mathit{d\zeta}_{2}}}}}{}\].

У матричній формі система для знаходження невідомих
\[\phi_{\mathit{\text{jv}}}^{\gamma}{}\],
\[\varepsilon_{\mathit{\text{ij}}}^{\gamma\mathit{\text{ms}}}{}\] та
\[C_{j}{}\] набуде такого загального вигляду

\[{\begin{bmatrix}
\begin{matrix}
E^{\gamma} \\
F^{\gamma} \\
B^{\gamma} \\
W^{\gamma} \\
\end{matrix} & {\begin{matrix}
E^{1} \\
F^{1} \\
B^{1} \\
W^{1} \\
\end{matrix}\text{.}\text{.}\text{.}\begin{matrix}
E^{M} \\
F^{M} \\
B^{M} \\
W^{M} \\
\end{matrix}} & \begin{matrix}
I \\
0 \\
0 \\
0 \\
\end{matrix} \\
\end{bmatrix}{\begin{bmatrix}
\phi^{\gamma} \\
\varepsilon^{1} \\
 \vdots \\
\varepsilon^{M} \\
C \\
\end{bmatrix} = \begin{bmatrix}
u^{p} \\
t^{p} \\
0 \\
0 \\
\end{bmatrix}}}{}\],(5.14)

де \[\phi^{\gamma}{}\] --- вектор невідомих вузлових значень фіктивних
поверхневих зусиль або масових сил у всьому дискретному аналозі
\[{\underset{\nu = 1}{\overset{V}{}}\gamma_{v}}{}\] краю тіла або
приграничної до нього області;

\[\varepsilon^{m}{}\] --- вектор невідомих вузлових значень компонент
деформацій та їх похідних у дискретному аналозі
\[{\underset{s = 1}{\overset{S_{m}}{}}\Omega_{\mathit{\text{ms}}}}{}\]
(\[{m = \overline{1,M}}{}\]) області локальної неоднорідності
Ω\textsubscript{m};

u\textsuperscript{p}, t\textsuperscript{p} --- вектори значень заданих
на краю тіла переміщень та поверхневих зусиль.

Елементи блоків глобальної матриці системи є сумами внесків окремих
елементів (5.10)-(5.13). А саме блок \[E^{\gamma}{}\] сформований з
\[E_{\mathit{\text{ij}}}^{\mathit{\text{γν}}}{}\]; блок \[E^{m}{}\] ---
з \[E_{i}^{\lambda\mathit{\text{ms}}}{}\] та
\[E_{\mathit{\text{ijl}}}^{\mu\mathit{\text{ms}}}{}\]; блок
\[F^{\gamma}{}\] --- з
\[F_{\mathit{\text{ik}}}^{\mathit{\text{γν}}}{}\]; блок \[F^{m}{}\] ---
з \[F_{i}^{\lambda\mathit{\text{ms}}}{}\] та
\[F_{\mathit{\text{ijl}}}^{\mu\mathit{\text{ms}}}{}\]; блок
\[B^{\gamma}{}\]--- з
\[B_{\mathit{\text{ijk}}}^{\mathit{\text{γν}}}{}\]; блок \[B^{m}{}\] ---
з \[B_{\mathit{\text{ij}}}^{\lambda\mathit{\text{ms}}}{}\] та
\[B_{\mathit{\text{ijkl}}}^{\mu\mathit{\text{ms}}}{}\]; блок
\[W^{\gamma}{}\]--- з \[W^{\mathit{\text{γν}}}{}\]; блок \[W^{m}{}\] ---
з \[W_{j}^{\lambda\mathit{\text{ms}}}{}\] та
\[W_{l}^{\mu\mathit{\text{ms}}}{}\].

У довільній точці
\includegraphics[width=0.8854in,height=0.2071in]{./ObjectReplacements/Obj925}
переміщення, деформації, напруження та поверх­неві зусилля визначають з
формул (5.9), (5.10), використовуючи розв'язок системи (5.14).

\hypertarget{ux43fux43eux440ux456ux432ux43dux44fux43dux43dux44f-ux447ux438ux441ux43bux43eux432ux438ux445-ux440ux435ux437ux443ux43bux44cux442ux430ux442ux456ux432-1}{%
\subsection[5.5. Порівняння числових
результатів]{\texorpdfstring{\protect\hypertarget{anchor-65}{}{}5.5.
Порівняння числових
результатів}{5.5. Порівняння числових результатів}}\label{ux43fux43eux440ux456ux432ux43dux44fux43dux43dux44f-ux447ux438ux441ux43bux43eux432ux438ux445-ux440ux435ux437ux443ux43bux44cux442ux430ux442ux456ux432-1}}

Наведемо результати досліджень, які проведені для тіла, що займає
область

Ω=\{0.0\textless=x\textsubscript{1}\textless=2.0;
0.0\textless=x\textsubscript{2}\textless=2.0 \}

і містить у собі наступну неоднорідність

Ω\textsubscript{g}=\{0.7\textless=x\textsubscript{1}\textless=1.3;
0.7\textless=x\textsubscript{2}\textless=1.3 \}

для уникнення громіздкості запису тут і надалі не будемо приводити
розмірності декартових координат, які, як і решта величин, розглядаються
в системі СІ. Зрозуміло, що в цьому випадку гранична поверхня тіла
описується виразом
\[{{\Gamma = \underset{i = 1}{\overset{4}{}}}\Gamma_{i}}{}\] , де

Приклад:

\[{x_{1}\in\lbrack 0,2\rbrack}{}\], \[{x_{2}\in\lbrack 0,2\rbrack}{}\].

\includegraphics[width=3.2646in,height=2.65in]{./ObjectReplacements/Obj929}

Рис. 5.1

На всій межі задані переміщення і температура:

u\textsubscript{1}=0, u\textsubscript{2}=0,
\[{\theta = 0}{}\]\[{x\in\Gamma_{1}}{}\];

u\textsubscript{1}=0, u\textsubscript{2}=0.01 * x\textsubscript{2},
\[{\theta = x_{2}}{}\] \[{x\in\Gamma_{2}}{}\];

u\textsubscript{1}=0, u\textsubscript{2}=0.02, \[{\theta = 2}{}\]
\[{x\in\Gamma_{3}}{}\];

u\textsubscript{1}=0, u\textsubscript{2}=0.01 * (2-x\textsubscript{2}),
\[{\theta = {2 - x_{2}}}{}\]\[{x\in\Gamma_{4}}{}\].

Для числових досліджень виби­рались \[{{\nu = 0}\text{.}\text{22}}{}\],
\[{\mu = \text{8200}}{}\] \[{x\in\Omega{}_{1}}{}\],
\[{{\nu = 0}\text{.}\text{22}}{}\],
\[{{\mu = \text{8200+\ 10000\ *}}(0\text{.}\text{7-x}_{2})\text{*}(x_{2}\text{-1}\text{.}3)}{}\]
\[{x\in\Omega_{1}}{}\].

Авторами планується продовжити дослідження задачі термопружності
запропонованим методом.

Графік u\textsubscript{2}:

\includegraphics[width=4.6252in,height=4.2083in]{Pictures/10004F10000069DC00006944FC787C14749A9BFB.wmf}

Рис. 5.2

\includegraphics[width=4.6665in,height=4.639in]{Pictures/100045A4000069D5000069426E9B709A759244AF.wmf}

Рис. 5.3\[\sigma_{\text{11}}{}\]

\includegraphics[width=4.5138in,height=4.5in]{Pictures/10006B88000069E90000695252D04DC508966658.wmf}

Рис. 5.4\[\sigma_{\text{22}}{}\]

\hypertarget{ux43fux456ux434ux441ux443ux43cux43eux43a-3}{%
\subsection[5.6.
Підсумок]{\texorpdfstring{\protect\hypertarget{anchor-66}{}{}5.6.
Підсумок}{5.6. Підсумок}}\label{ux43fux456ux434ux441ux443ux43cux43eux43a-3}}

\textbf{Розділ 6}

\hypertarget{ux43fux440ux43eux433ux440ux430ux43cux43dux435-ux441ux435ux440ux435ux434ux43eux432ux438ux449ux435-ace}{%
\section[Програмне середовище
ACE]{\texorpdfstring{\protect\hypertarget{anchor-67}{}{}Програмне
середовище
ACE}{Програмне середовище ACE}}\label{ux43fux440ux43eux433ux440ux430ux43cux43dux435-ux441ux435ux440ux435ux434ux43eux432ux438ux449ux435-ace}}

\hypertarget{ux43cux43eux434ux443ux43bux456-ux441ux438ux441ux442ux435ux43cux438}{%
\subsection[6.1. Модулі
системи]{\texorpdfstring{\protect\hypertarget{anchor-68}{}{}6.1. Модулі
системи}{6.1. Модулі системи}}\label{ux43cux43eux434ux443ux43bux456-ux441ux438ux441ux442ux435ux43cux438}}

\hypertarget{ux43eux431ux491ux440ux443ux43dux442ux443ux432ux430ux43dux43dux44f-ux432ux438ux431ux43eux440ux443-ux43eux431ux454ux43aux442ux43dux43e-ux43eux440ux456ux454ux43dux442ux43eux432ux430ux43dux43eux433ux43e-ux43fux456ux434ux445ux43eux434ux443}{%
\subsubsection{6.2.1 Обґрунтування вибору об'єктно-орієнтованого
підходу}\label{ux43eux431ux491ux440ux443ux43dux442ux443ux432ux430ux43dux43dux44f-ux432ux438ux431ux43eux440ux443-ux43eux431ux454ux43aux442ux43dux43e-ux43eux440ux456ux454ux43dux442ux43eux432ux430ux43dux43eux433ux43e-ux43fux456ux434ux445ux43eux434ux443}}

Протягом багатьох років підходи до програмування систем мінялися в
залежності від потреб часу. Спочатку програми скаладалися на машинній
мові, пізніше виникли асемблери, які давали змогу транслювати команди
зрозумілі людині в машинні коди. Щоб спростити написання програми і
приблизити мови програмування до людської мови виникли макропроцесори,
які перетворюють макрокоманди у команди асемблера. Далі із
мікроасемблерних бібліотек з'явилися мови процедурного типу, які мали
власний транслятор у машинні коди.

Процедурні мови програмування набули широкого розповсюдження і
використовувалися для побудови обчислювальних алгоритмів. Найбільшими
проблемами, з якими стикаються люди у цих мовах програмування є невисока
аналогія із реальними об'єктами, які моделюються. Процедури дозволяються
реалізувати алгоритми і розбити їх на підзадачі.

Об'єктно-орієнтоване програмування - одна з парадигм програмування, на
відміну від процедурного програмування дозволяє ширші можливості
застосування. У процедурному ("алгоритмічному") програмуванні
(програмування за допомогою мов Fortran, Бейсік, Pascal та їм подібних
"алгоритмічних" мов) уся увага зосереджується на розробці та проробці
системи взаємодіючих процедур та функцій, часто згрупованих у модулі та
бібліотеки за семантичними та іншими ознаками, котрі реалізують
алгоритми необхідні для функціонування програми або операційної системи.
Дані у таких програмах зберігаються у глобальних (відносно окремих
процедур) змінних і передаються у процедури та функції як параметри.

На відміну від того в об'єктно-орієнтованому програмуванні
(використовують мови Simula-67, С++, Smalltalk, Java, Python та їм
подібні) дані та методи (процедури) пов'язані з ними є комбінованими у
класи (об'єкти), що відповідають онтологічним сутностям прикладної
області або є допоміжними у програмі. Тоді конкретні значення
полей(змінних) об'єкта визначають його стан, а його методи дозволяють
іншим об'єктам програми взаємодіяти з ним. Доступні методи об'єкта ще
називають його інтерфейсом (контрактом).

ООП базується на трьох парадигмах: інкапсуляція, успадкування,
поліморфізм.

\begin{enumerate}
\def\labelenumi{\arabic{enumi}.}
\tightlist
\item
  Інкапсуляція - об'єкт приховує свою внутрішню реалізацію, залишаючи
  для користувачів інтерфейси взаємодії
\item
  Успадкування - нащадок об'єкта отримує ті ж самі властивості, що й
  об'єкт-предок.
\item
  Поліморфізм - об'єкт змінює свою поведінку у рамках інтерфейсу,
  успадкованому від батьківських об'єктів
\end{enumerate}

Основні проблеми програмного забезпечення С++

\begin{enumerate}
\def\labelenumi{\arabic{enumi}.}
\tightlist
\item
  Вказівники
\item
  Бібліотечні класи перевантажені функціональністю
\item
  При процедурному стилі внутрішні дані видимі всім
\item
  Різні компілятори мають свої особливості
\end{enumerate}

Переваги

\begin{enumerate}
\def\labelenumi{\arabic{enumi}.}
\tightlist
\item
  С++ мова, компілятори для якої є на багатьох платформах
\item
  Багатошарова структура (розбиття на модулі) дозволяє рознесення різних
  частин програми на різні фізичні машини для підвищення продуктивності
  системи
\item
  Внутрішні дані потрібні для певного об'єкту залишаються недосяжними
  для інших об'єктів
\item
  Реалізовано ряд алгоритмів, які вже не треба повторно писати
\item
  Масштабованість -- без зміни існуючого коду можна додати новий код
\item
  Багатошарова структура програми дозволяє міняти один із шарів без
  зміни іншого
\item
  Дані, що виводяться можна без особливих зусиль писати у довільному
  форматі, оскільки сам розв'язок задачі подається у зручному вигляді
\end{enumerate}

\hypertarget{ux43eux441ux43dux43eux432ux43dux456-ux43fux456ux434ux445ux43eux434ux438}{%
\subsubsection{6.2.2 Основні
підходи}\label{ux43eux441ux43dux43eux432ux43dux456-ux43fux456ux434ux445ux43eux434ux438}}

Патерни застосовуються для того щоб описати найтиповіші рішення.

Система складається із наступних підсистем, які можна зобразити у
вигляді модулів:

\begin{enumerate}
\def\labelenumi{\arabic{enumi}.}
\tightlist
\item
  Введення даних
\item
  Класів загального використання
\item
  Обчислень
\item
  Виведення даних
\end{enumerate}

\includegraphics[width=4.2646in,height=2.6646in]{./ObjectReplacements/Obj946}

Рис. 6.1

\hypertarget{ux43cux43eux434ux443ux43bux44c-ux432ux432ux435ux434ux435ux43dux43dux44f-ux434ux430ux43dux438ux445}{%
\subsection[6.2. Модуль введення
даних]{\texorpdfstring{\protect\hypertarget{anchor-69}{}{}6.2. Модуль
введення
даних}{6.2. Модуль введення даних}}\label{ux43cux43eux434ux443ux43bux44c-ux432ux432ux435ux434ux435ux43dux43dux44f-ux434ux430ux43dux438ux445}}

Даний модуль містить набір класів для читання і аналізу вхідних даних із
файлів.

\includegraphics[width=5.0429in,height=2.8854in]{./ObjectReplacements/Obj947}

Рис. 6.2

\hypertarget{ux43aux43bux430ux441ux456ux432-ux437ux430ux433ux430ux43bux44cux43dux43eux433ux43e-ux432ux438ux43aux43eux440ux438ux441ux442ux430ux43dux43dux44f}{%
\subsection{6.3. Класів загального
використання}\label{ux43aux43bux430ux441ux456ux432-ux437ux430ux433ux430ux43bux44cux43dux43eux433ux43e-ux432ux438ux43aux43eux440ux438ux441ux442ux430ux43dux43dux44f}}

Класи, які всюди потрібні.

\includegraphics[width=4.2071in,height=2in]{./ObjectReplacements/Obj948}

Рис. 6.3

\includegraphics[width=2.4429in,height=1.9854in]{./ObjectReplacements/Obj949}

Рис 6.4

\hypertarget{ux43cux43eux434ux443ux43bux44c-ux43eux431ux447ux438ux441ux43bux435ux43dux44c}{%
\subsection[6.4. Модуль
обчислень]{\texorpdfstring{\protect\hypertarget{anchor-70}{}{}6.4.
Модуль
обчислень}{6.4. Модуль обчислень}}\label{ux43cux43eux434ux443ux43bux44c-ux43eux431ux447ux438ux441ux43bux435ux43dux44c}}

Основні алгоритми реалізовані у середовищі ACE містяться у цьому модулі.

\includegraphics[width=6.2646in,height=2.7354in]{./ObjectReplacements/Obj950}

Рис. 6.5

\includegraphics[width=3.2646in,height=2.9283in]{./ObjectReplacements/Obj951}

Рис. 6.6

\hypertarget{ux43cux43eux434ux443ux43bux44c-ux432ux438ux432ux435ux434ux435ux43dux43dux44f-ux434ux430ux43dux438ux445}{%
\subsection[6.5. Модуль виведення
даних]{\texorpdfstring{\protect\hypertarget{anchor-71}{}{}6.5. Модуль
виведення
даних}{6.5. Модуль виведення даних}}\label{ux43cux43eux434ux443ux43bux44c-ux432ux438ux432ux435ux434ux435ux43dux43dux44f-ux434ux430ux43dux438ux445}}

Даний модуль виводить дані у представленні зручному для обробки різними
графічними пакетами.

\hypertarget{ux432ux437ux430ux454ux43cux43eux434ux456ux44f-ux441ux438ux441ux442ux435ux43cux438-ux456ux437-ux43aux43eux440ux438ux441ux442ux443ux432ux430ux447ux435ux43c}{%
\subsection[6.6. Взаємодія системи із
користувачем]{\texorpdfstring{\protect\hypertarget{anchor-72}{}{}6.6.
Взаємодія системи із
користувачем}{6.6. Взаємодія системи із користувачем}}\label{ux432ux437ux430ux454ux43cux43eux434ux456ux44f-ux441ux438ux441ux442ux435ux43cux438-ux456ux437-ux43aux43eux440ux438ux441ux442ux443ux432ux430ux447ux435ux43c}}

Система має доволі простий графічний інтерфейс, чим спрощує роботу із
нею.

\hypertarget{ux43fux456ux434ux441ux443ux43cux43eux43a-4}{%
\subsection[6.7.
Підсумок]{\texorpdfstring{\protect\hypertarget{anchor-73}{}{}6.7.
Підсумок}{6.7. Підсумок}}\label{ux43fux456ux434ux441ux443ux43cux43eux43a-4}}

Розроблене середовище гнучке і масштабоване і може застосовуватися для
розв'язування інших задач.

\hypertarget{ux432ux438ux441ux43dux43eux432ux43aux438}{%
\section[Висновки]{\texorpdfstring{\protect\hypertarget{anchor-74}{}{}Висновки}{Висновки}}\label{ux432ux438ux441ux43dux43eux432ux43aux438}}

йцкйцкйц

\hypertarget{ux441ux43fux438ux441ux43eux43a-ux43bux456ux442ux435ux440ux430ux442ux443ux440ux438}{%
\section[Список
літератури]{\texorpdfstring{\protect\hypertarget{anchor-75}{}{}Список
літератури}{Список літератури}}\label{ux441ux43fux438ux441ux43eux43a-ux43bux456ux442ux435ux440ux430ux442ux443ux440ux438}}

\begin{enumerate}
\def\labelenumi{\arabic{enumi}.}
\tightlist
\item
  \emph{Бахвалов Н.С., Жидков Н.П., Кобельков Г.М. }Численые методы.
  --М.: Наука, 1987. -- 598с.
\end{enumerate}

\begin{enumerate}
\def\labelenumi{\arabic{enumi}.}
\tightlist
\item
  \protect\hypertarget{anchor-2}{}{}\emph{Бенерджи П., Баттерфилд Р}.
  Метод граничных элементов в прикладных науках. --- М.: Мир, 1984. --
  494 с.
\end{enumerate}

\begin{enumerate}
\def\labelenumi{\arabic{enumi}.}
\tightlist
\item
  \emph{Бобик О.І.} Рівняння математичної фізики (Теорія потенціалу,
  рівняння Гельмгольца, задача Штурма-Ліувіля, спеціальні функції) :
  Текст лекцій. -Львів: Ред.вид.відділ. 1990. -84с.
\end{enumerate}

\begin{enumerate}
\def\labelenumi{\arabic{enumi}.}
\tightlist
\item
  \protect\hypertarget{anchor-3}{}{}\emph{Бреббия К., Теллес Ж., Вроубел
  Л}. Методы граничных элементов. --- М.: Мир, 1987.-- 524 с.
\end{enumerate}

\begin{enumerate}
\def\labelenumi{\arabic{enumi}.}
\tightlist
\item
  \emph{Григоренко Я.М., Грицько Є.Г., Журавчак Л.М. }Застосування
  скінченних різниць і приграничних елементів в задачі пружності для
  неоднорідного тіла // Доп. АН України. -- 1993. - №4. --С. 49-53.
\end{enumerate}

\begin{enumerate}
\def\labelenumi{\arabic{enumi}.}
\tightlist
\item
  \protect\hypertarget{anchor-4}{}{}\emph{Григоренко Я.М., Грицько Є.Г.,
  Журавчак Л.М. }Застосування скінченних різниць і приграничних
  елементів в задачі пружності для неоднорідного тіла // Доп. АН
  України. -- 1993. - №4. --С. 49-53.
\end{enumerate}

\begin{enumerate}
\def\labelenumi{\arabic{enumi}.}
\tightlist
\item
  \protect\hypertarget{anchor-5}{}{}\emph{Грицько Є.Г., Гудзь Р.В.,
  Журавчак Л.М.} Метод граничних елементів та ермітові скінченні
  елементи у задачах пружності для тіл з неоднорідностями //
  Фізико-хімічна механіка матеріалів. №5, 1999. С. 99--101.
\end{enumerate}

\begin{enumerate}
\def\labelenumi{\arabic{enumi}.}
\tightlist
\item
  \emph{Грицько Є.Г., Журавчак Л.М. }Численно-аналитический способ
  решения нелинейной задачи теплопроводности для прямой призмы // Мат.
  методы и физ.-мех. поля. -- К.: Наук. думка, 1990. -- Вып. 31. -- С.
  95 -- 99.
\end{enumerate}

\begin{enumerate}
\def\labelenumi{\arabic{enumi}.}
\tightlist
\item
  \protect\hypertarget{anchor-6}{}{}\emph{Громадка ІІ Т., Лей Ч.}
  Комплексный метод граничных элементов в инженерных задачах. -- М.:
  Мир, 1990. -- 303с.
\end{enumerate}

\begin{enumerate}
\def\labelenumi{\arabic{enumi}.}
\tightlist
\item
  \emph{ Гудзь Р.В.,} \emph{Петльований А.Т. }Комплексний метод
  граничних елементів при моделюванні фізичних процесів у тілах з
  композиційних матеріалів // Дев'ята Всеукр. наук. конф. ``Сучасні
  проблеми прикл. матем. та інформатики. Львів, 2002. С. 37.
\end{enumerate}

\begin{enumerate}
\def\labelenumi{\arabic{enumi}.}
\tightlist
\item
  \emph{ Гудзь Р.В.,} \emph{Петльований А.Т. }Моделювання стаціонарного
  процесу теплопровідності у тілі зі сукупністю локальних
  неоднорідностей за допомогою комплексного методу граничних елементів
  // Восьма Всеукр. наук. конф. ``Сучасні проблеми прикл. матем. та
  інформатики. Львів, 2001. С. 52.
\end{enumerate}

\begin{enumerate}
\def\labelenumi{\arabic{enumi}.}
\tightlist
\item
  \emph{ }\protect\hypertarget{anchor-7}{}{}\emph{Гудзь Р.В.} Визначення
  розв'язку плоскої стаціонарної задачі теплопровідності для
  локально-неоднорідної області з використанням непрямого методу
  граничних елементів // Вісн. Львів. ун-ту. Сер. мех.-мат. 1995. Вип.
  42. --- С. 74-78.
\end{enumerate}

\begin{enumerate}
\def\labelenumi{\arabic{enumi}.}
\tightlist
\item
  \emph{ Дульев Г}.\emph{Н}.\emph{, Заричняк Ю.П. }Теплопроводимость
  смесей и композиционных материалов. Справочная книга. Л.: ``Энергия'',
  1974. -- 264 с.
\end{enumerate}

\begin{enumerate}
\def\labelenumi{\arabic{enumi}.}
\tightlist
\item
  \emph{ }\protect\hypertarget{anchor-24}{}{}\emph{Журавчак Л.М.,
  Грицько Є.Г.} Метод приграничних елементів у прикладних задачах
  математичної фізики // КВ ІГФ НАН України, 1996. - 220 с.
\end{enumerate}

\begin{enumerate}
\def\labelenumi{\arabic{enumi}.}
\tightlist
\item
  \emph{ Зенкевич О., Морган К. }Конечные элементы и аппроксимация. --
  М.:Мир, 1986.---318 с.
\end{enumerate}

\begin{enumerate}
\def\labelenumi{\arabic{enumi}.}
\tightlist
\item
  \emph{ }\protect\hypertarget{anchor-8}{}{}\emph{Крауч С., Старфилд А.}
  Методы граничных элементов в механике твердого тела. --- М.: Мир,
  1997. -- 328 с.
\end{enumerate}

\begin{enumerate}
\def\labelenumi{\arabic{enumi}.}
\tightlist
\item
  \protect\hypertarget{anchor-11}{}{}\emph{Купрадзе В.Д.} Методы
  потенциала в теории упругости. --М.: Физматиз, 1963. -- 472с.
\end{enumerate}

\begin{enumerate}
\def\labelenumi{\arabic{enumi}.}
\tightlist
\item
  \emph{Марчук Г.И.}, \emph{Агошков В.И.} Введение в
  проекционно-сеточные методы. --М.: Наука, 1981. -- 416с.
\end{enumerate}

\begin{enumerate}
\def\labelenumi{\arabic{enumi}.}
\tightlist
\item
  \protect\hypertarget{anchor-9}{}{}\emph{Михлин С.Г.} Численая
  реализация вариационных методов. --М.: Наука, 1966. -- 432с.
\end{enumerate}

\begin{enumerate}
\def\labelenumi{\arabic{enumi}.}
\tightlist
\item
  \protect\hypertarget{anchor-10}{}{}\emph{Мусхешивили Н.И.} Некоторые
  основные задачи математической теории упругости. --М.: Наука, 1966. --
  708с.
\end{enumerate}

\begin{enumerate}
\def\labelenumi{\arabic{enumi}.}
\tightlist
\item
  \emph{Норри Д., де Фриз Ж.} Введение в метод конечных элементов. ---
  М.: Мир, 1981. --- 304 с.\emph{ }
\end{enumerate}

\begin{enumerate}
\def\labelenumi{\arabic{enumi}.}
\tightlist
\item
  \emph{ Положий Г.М.} Уравнения математической физики. --М: Высшая
  школа. 1964. --559с.
\end{enumerate}

\begin{enumerate}
\def\labelenumi{\arabic{enumi}.}
\tightlist
\item
  \emph{ Работнов Ю.Н.} Механика деформируемого твёрдого тела. --М.:
  Наука, 1988. -- 712с.
\end{enumerate}

\begin{enumerate}
\def\labelenumi{\arabic{enumi}.}
\tightlist
\item
  \emph{ }\protect\hypertarget{anchor-39}{}{}\emph{Ректорис К.}
  Вариационные методы в математической физике и технике. --- М.: Мир,
  1985. --- 590 с.
\end{enumerate}

\begin{enumerate}
\def\labelenumi{\arabic{enumi}.}
\tightlist
\item
  \emph{ Савула Я.Г., Шинкаренко Г.А.} Метод скінчених елементів.
  --Львів: Вища школа. 1976. -- 79с.
\end{enumerate}

\begin{enumerate}
\def\labelenumi{\arabic{enumi}.}
\tightlist
\item
  \emph{ }\protect\hypertarget{anchor-58}{}{}\emph{Журавчак Л.М.}
  Порівняння розв'язків задач теорії пружності для різних пригра­ничних
  елемен­тів // Фіз.-хім. меха­ніка матеріалів, 2002. -- № 6. -- С.
  79-84.
\end{enumerate}

\begin{enumerate}
\def\labelenumi{\arabic{enumi}.}
\tightlist
\item
  \protect\hypertarget{anchor-15}{}{}\emph{Гузь А.Н., Немиш Ю.Н.}
  Статика упругих тел неканонической формы. -- К.: Наук. думка, 1984. --
  280 с. (Пространственные задачи теории упругости и пластичности, в
  6-ти томах, Т.2).
\end{enumerate}

\begin{enumerate}
\def\labelenumi{\arabic{enumi}.}
\tightlist
\item
  \protect\hypertarget{anchor-16}{}{}\emph{Стадник М.М.} Концентрація
  напружень біля пружного еліпсоїдального включення у безмежному тілі //
  Фіз.-хім.меха¬ніка матеріалів, 2002. -- № 6. -- С. 25-30.
\end{enumerate}

\begin{enumerate}
\def\labelenumi{\arabic{enumi}.}
\tightlist
\item
  \protect\hypertarget{anchor-17}{}{}\emph{Рвачев В.Л., Синекоп Н.С.}
  Метод R-функций в задачах теории упругости и пластичности. -- К.:
  Наук. думка, 1990. -- 216 с.
\end{enumerate}

\begin{enumerate}
\def\labelenumi{\arabic{enumi}.}
\tightlist
\item
  \protect\hypertarget{anchor-18}{}{}\emph{Саркисян В.С., Керопян А.В.}
  К решению двух контактных задач для упругих тел с двумя разнородными
  конечными стрингерами // Мат. методи і фіз.-мех. поля, 2003. -- Т. 46,
  № 2. -- С. 114-121.
\end{enumerate}

\begin{enumerate}
\def\labelenumi{\arabic{enumi}.}
\tightlist
\item
  \protect\hypertarget{anchor-19}{}{}\emph{Бережницький Л.Т., Мазурак
  Л.П., Качур П.С.} Плоска задача для ізотропного тіла з включенням,
  оточеним пружним прошарком // Фіз.-хім. меха¬ніка матеріалів, 2000. --
  № 3. -- С. 27-34.
\end{enumerate}

\begin{enumerate}
\def\labelenumi{\arabic{enumi}.}
\tightlist
\item
  \protect\hypertarget{anchor-20}{}{}\emph{Григоренко Я.М., Василенко
  А.Т.} Задачи статики неоднородных анизотропных оболочек. -- М.: Наука,
  1992. -- 336 с.
\end{enumerate}

\begin{enumerate}
\def\labelenumi{\arabic{enumi}.}
\tightlist
\item
  \protect\hypertarget{anchor-21}{}{}\emph{Головач Н.П., Дияк І.І.}
  Метод декомпозиції області та комбінований
  скінченно-гранично-елементний аналіз задач пружності // Фіз.-хім.
  меха¬ніка матеріалів, 2000. -- № 1. -- С. 115-117.
\end{enumerate}

\begin{enumerate}
\def\labelenumi{\arabic{enumi}.}
\tightlist
\item
  \protect\hypertarget{anchor-22}{}{}\emph{Вишневский К.В. Кушнир Р.М.}
  Граничные интегральные уравнения для тела с инородными включениями //
  Мат. методи і фіз.-мех. поля, 1996. -- Т. 39, № 1. -- С. 37-41.
\end{enumerate}

\begin{enumerate}
\def\labelenumi{\arabic{enumi}.}
\tightlist
\item
  \protect\hypertarget{anchor-23}{}{}\emph{Лавренюк В.І., Терещенко
  В.М.} Напружений стан кусково-однорідних тіл при дії нестаціо¬нарних
  теплових полів // Мат. методи і фіз.-мех. поля, 1997. -- Т. 40, № 1.
  -- С. 53-58.
\end{enumerate}

\begin{enumerate}
\def\labelenumi{\arabic{enumi}.}
\tightlist
\item
  \protect\hypertarget{anchor-25}{}{}\emph{Грицько Є.Г., Журавчак Л.М.}
  Порівняння методів гранич¬них і пригра¬ничних елемен¬тів для
  розв'язування задач теорії пружності // Фіз.-хім. меха¬ніка
  матеріалів, 1997. -- № 3. -- С. 50-56.
\end{enumerate}

\begin{enumerate}
\def\labelenumi{\arabic{enumi}.}
\tightlist
\item
  \protect\hypertarget{anchor-26}{}{}\emph{Грицько Є.Г., Журавчак Л.М.,
  Фітель Г.В., Шуміліна Н.В.} Автоматизація число¬вих дос¬лід¬жень
  методом пригра¬нич¬них елементів фізичних по¬лів у много¬кутни¬ках //
  Вісник Львів. ун-ту. Сер. прикл. мат. та інформ.  2001.  Вип. 3. 
  C. 100-105.
\end{enumerate}
