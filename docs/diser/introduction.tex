\textbf{Актуальність теми.} 
Суперкомп’ютерні технології та системи є одним з головних чинників, що визначають конкурентоспроможність економіки держави та рівень її технологічного розвитку. Без їх використання неможливо створювати сучасну техніку, нові матеріали, лікарські засоби, моделювати та прогнозувати екологічні, технологічні та інші загрози. Понад 60\% світових високопродуктивних обчислювальних ресурсів використовують для розв’язання складних задач економіки, управління фінансами, промисловості, нафтогазової та інженерно-геологічної галузей, решта -- орієнтовані на задачі обороноздатності та науки.

Аналіз світових тенденцій застосування супер-ЕОМ у Росії, США, ЄС та інших країнах свідчить, що цей напрямок є одним з пріоритетних. 

В Україні, яка має високий науково-технічний потенціал у галузі створення сучасних високопродуктивних інтелектуальних інформаційних технологій, створено ряд кластерних комплексів, які застосовуються при розв'язанні складних задач господарського комплексу, науки та промисловості.


\textbf{Зв'язок роботи з науковими програмами, планами, темами.}
Дисертаційна робота виконувалась у рамках наступних держбюджетних дослідницьких тем:



\textbf{Мета і завдання дослідження.}
Метою роботи є подальший розвиток архітектури та базового програмного забезпечення високопродуктивних обчислювальних комплексів за рахунок використання сучасних технологій віртуалізації обчислювальних систем та  застосування графічних прискорювачів.
% Результати мають точно відповідати задачам

Основні задачі дослідження:
\begin{itemize}
 \item розробка архітектури;
 \item побудова методики;
 \item побудова моделі;
 \item розробка методів;
 \end{itemize}

%Об'єкт - процес або явище, яке приводить до проблемної ситуації, яка підлягає розгляду.

\textit{Об'єкт дослідження} -- процес проектування високопродуктивних обчислювальних комплексів з кластерною архітектурою.

\textit{Предмет дослідження} -- архітектура та базове програмне забезпечення високопродуктивних обчислювальних комплексів.

\textit{Методи дослідження.} Для розв'язання сформульованих задач у дисертаційній роботі використовуються результати та методи загальної теорії систем, теорії інформації, теорії алгоритмів, теорії ймовірностей, математичної статистики та математичного аналізу.

\textbf{Наукова новизна отриманих результатів.}
Основними результатами, які визначають наукову новизну і виносяться на захист, є наступні:

\begin{itemize}
\item удосконалено;
\item вперше запропоновано;
\item розвинуто методи.
\end{itemize}


\textbf{Практичне значення отриманих результатів.}
Отримані в дисертаційній роботі результати мають практичне значення, і є внеском у ... Розв'язані задачі в області архітектури дозволяють зменшити вартість побудови, знизити енергоспоживання, збільшити надійність та продуктивність СКА.

Практичне значення визначають наступні результати:

\begin{itemize}
\item розроблено; 
\item розвинуто;
\item досліджено.
\end{itemize}

Результати роботи використані при ....

%Hidden cite for correct bibliography ordering
\nocite{getero-art,getero-taganrog-art,supercomp-architecture-art,security-art,boib-art,srp-art,portal-art,web,web-portal} 
\nocite{supercomp-architecture-proceed,policy,50anniv-art,steking,cluster-grid-management-system,cluster-stats,transport,web-portal-abrau,hpcua-scheduling,hpcua-gpu,scms-congress2011}

\textbf{Особистий внесок здобувача.}
Усі наукові результати дисертаційної роботи отримані автором особисто, вони опубліковані в 9 статтях у фахових журналах. У працях, виконаних у співавторстві, здобувачу належать такі результати: \cite{getero-art,getero-taganrog-art}~--~методика побудови системи керування гетерогенним кластерним комплексом.


\textbf{Апробація результатів.}
Основні результати досліджень доповідались на міжнародних наукових конференціях і обговорювались на наукових семінарах: 


12, 13 Міжнародній науковій конференції ``Knowledge-Dialogue-Solution'' (Вар\-на, Болгарія, 2006, 2007);
4, 6 Міжнародній науковій конференції ``Искусственный интеллект, Ин\-те\-ллек\-ту\-аль\-ные многопроцессорные системы'' (смт. Ка\-ци\-ве\-лі, Крим, Україна, 2006, 2008);
науковій молодіжній міжнародній школі ``Высокопроизводительные вычислительные системы'' (смт. Кацивелі, Крим, Україна, 2006);


\textbf{Публікації.}
Основні результати дисертації опубліковано в 20 наукових працях. З них 9 -- у фахових виданнях та одинадцять тез доповідей міжнародних наукових конференціях.

\textbf{Структура та обсяг роботи.}
Дисертаційна робота складається із вступу, чотирьох розділів, розбитих на підрозділи, висновків, 4-х додатків (додаток Г містить акти впровадження дисертаційної роботи) та списку використаних джерел, що містить 112 найменувань. Повний обсяг роботи -- 183 сторінки. 

\vspace{1em}
\begin{center}
\textbf{ОСНОВНИЙ ЗМІСТ РОБОТИ}
\end{center}

Далі короткий текст по основним результатам.