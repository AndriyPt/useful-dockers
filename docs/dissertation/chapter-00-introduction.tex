\chapter{ВСТУП}

%Hidden cite for correct bibliography ordering
\nocite{bahvalov-et-al,benerdge-et-al} 

\textbf{Актуальність теми.} 
Більшість елементів сучасного приводу, конструкцій чи побутові речі виготовлені із гетерогенного матеріалу, зокрема із композиційного матеріалу або композиту. Тому виникає потреба у розробці методів, які б дозволяли розраховувати напружено-деформований стан, а також температуру.

Широко вживаними протягом кількох десятиріч методом розв'язування задач такого класу є проекційно-сітковий метод або метод скінчених елементів (МСЕ). Завдяки таким властивостям як розрідженість і симетричність матриць для класу задач у локально-неоднорідних тілах, різні варіанти МСЕ набули популярності.

Із МСЕ цілком успішно може конкурувати, а іноді бути ефективнішим,
велике сімейство методів граничних елементів
(МГЕ) \cite{benerdge-et-al} {[}4{]}{[}6{]}{[}7{]}{[}9{]}{[}12{]}{[}16{]}, які базуються
на ідеях розроблених С.Г.Міхліном{[}19{]}, Н.І.Мусхелішвілі{[}20{]},
В.Д.Купрадзе{[}17{]} та інші. Ці обставині особливо сприяють такі
переваги, як зменшення на одиницю геометричної розмірності задачі,
простота розв'язування для безмежних областей, можливість обчислювати у
лінійних задачах значення шуканих величин у довільній точці без нової
апроксимації, простота дискретизації.

У свою чергу серед класу МГЕ можна виділити такі методи як комплексний
метод граничних елементів (КМГЕ) і метод при граничних елементів (МПГЕ).
На відміну від КМГЕ, який є видозміненим варіантом класичного МГЕ, МПГЕ
базується на більш радикальніших ідеях. У порівнянні із МГЕ, МПГЕ
простіший для реалізації, проте у більшості випадків дає точніші
результати при тих самих затратах обчислювальних ресурсів.

Наукове завдання дисертаційної роботи -- розробка ефективного методу для
розв'язування задач термопружності у композитах є актуальною проблемою.

\textbf{Зв'язок роботи з науковими програмами, планами та
темами.} Робота виконувалась на кафедрі програмування львівського
національного університету імені Івана Франка, а також у інституті
прикладних проблем механіки і математики імені Ярослава Підстригача НАН
України, відділ термомеханіки в рамках держбюджетної теми ``Розробка
аналітико-чисельних методів дослідження напруженого стану неоднорідних
тіл з тепловими та залишковими деформаціями і дефектами структури''.
(!!! Номер теми !!!)

\textbf{Мета роботи та задачі дослідження.} Метою роботи є:

\begin{itemize}
\item
  побудова нових методів розв'язування задач термопружності у композитах
\item
  дослідження ефективності нових методів
\item
  порівняння швидкості збіжності і використання ресурсів із відомими
  методами
\end{itemize}

\emph{Об'єктом} дослідження є композити.

\emph{Предметом} дослідження є нові методи розв'язування задач
термопружності для композитів.

\emph{Методи дослідження.} Для розв'язування поставлених задач
використовується метод поєднання граничних і скінчених елементів.

\textbf{Наукова новизна одержаних результатів.} Наступні
результати отримані у роботі є новими:

\begin{enumerate}
\def\labelenumi{\arabic{enumi}.}
\item
  побудований метод на основі КМГЕ і МСЕ для розв'язування задачі
  теплопровідності у тілах із локальними неоднорідностями
\item
  для тіл із локальними неоднорідностями побудований метод на основі
  МПГЕ і МСЕ для розв'язування задач:

  \begin{enumerate}
  \def\labelenumii{\alph{enumii}.}
  \item
    пружності
  \item
    термопружності
  \end{enumerate}
\end{enumerate}

\textbf{Практичне значення отриманих результаті.} Отримані
результати дозволяють ефективніше розв'язувати актуальні проблеми, які
ставить перед нами народне господарство.

\textbf{Апробація результатів дисертації.} Основні результати
виконаних досліджень доповідалися та обговорювалися на:

\begin{enumerate}
\def\labelenumi{\arabic{enumi}.}
\item
  Восьмій Всеукраїнській науковій конференції (25-27 вересня 2001 р., м.
  Львів) ``Сучасні проблеми прикладної математики та інформатики'';
\item
  Дев'ятій Всеукраїнській науковій конференції (24-26 вересня 2002 р.,
  м. Львів) ``Сучасні проблеми прикладної математики та інформатики'';
\item
  Десятій Всеукраїнській науковій конференції (23-25 вересня 2003 р., м.
  Львів) ``Сучасні проблеми прикладної математики та інформатики'';
\item
  Одинадцятій Всеукраїнській науковій конференції (!!! 2004р., м. Львів)
  ``Сучасні проблеми прикладної математики та інформатики'';
\item
  Дванадцятій Всеукраїнській науковій конференції (4-6 жовтня 2005р., м.
  Львів) ``Сучасні проблеми прикладної математики та інформатики'';
\item
  Науковому семінарі ІППММ НАНУ імені Я.С.Підстригача, присвяченої
  пам'яті Є.Г.Грицька (22 квітня 2003 р., м. Львів);
\item
  Шостій Міжнародній науковій конференції (26-29 травня 2003 р., м.
  Львів) ``Математичні проблеми механіки неоднорідних структур'';
\item
  Семінар інституту прикладних проблем механіки та математики ім. Я.
  Підстригача
\end{enumerate}

У повному обсязі робота доповідалася на семінарі кафедри програмування
львівського національного університету імені Івана Франка.

\textbf{Публікації.} Результати досліджень висвітлені в 4 статтях
у журналах з переліку фахових видань ВАК України, в тому числі у
доповідях НАН України.

\textbf{Особистий внесок здобувача.} Всі результати дисертаційної
роботи отримано здобувачем самостійно.

\textbf{Структура та обсяг роботи.} Дисертація складається із
вступу і шести розділів основної частини, висновків, списку використаної
літератури із 37 найменувань.

У першому розділі описано підходи до розв'язування задач термопружності
і підхід, який був вибраний у даній роботи. Проаналізовано переваги і
недоліки кожного із запропонованих методів.

Другий розділ містить опис двох підходів для розв'язування задачі
теплопровідності. Перший підхід базується на ідеї предиктор-коректор, а
другий -- на рівняннях нев'язки у точках областей локальної
неоднорідності коефіцієнта теплопровідності.

Метод адитивного розчеплення оператора застосований для задачі пружності
для локально-неоднорідного тіла запропонований у третьому розділі. Даний
підхід був апробований для двох методів МГЕ і МПГЕ. Наведено порівняння
чисельних результатів для цих двох методів, а також порівняння методів
по інших критеріях, таких як простота реалізації, точність, діагональне
переважання матриць.

Четвертий розділ містить порівняльний аналіз МГЕ і МПГЕ при розв'язуванні задачі пружності у кусково-однорідному тілі. Наводяться рекомендації щодо застосування даних підходів для розв'язування задач такого класу.

У п'ятому розділі наведено підхід на основі МГЕ і МПГЕ для розв'язування
задачі термопружності у тілі із локально-неоднорідними властивостями
фізичних параметрів.

Програмне середовище Aspirant's Calculations Envinronment (ACE) було
створене для простого і зручного програмування запропонованих методів.
Структура даного середовища описана у шостому розділі. Також дано
рекомендації відносно використання даного середовища для програмування
подібних методів, яке вимагає невеликих зусиль спрямованих тільки на
побудову алгоритму, але не на організацію взаємодії між об'єктами.

\textbf{Усталені домовленості.} У цій праці кома перед індексом
означає похідну по змінній, яка відповідає цьому індексу, тобто
 ${f_{,j} = \partial}{f/\partial}x_{j}$, ${{f_{,ij} = \partial^{2}}{f/\partial}x_{i}\partial x_{j}}{}$,
при цьому викори­стовується німе сумування за індек­сами, які
повторюються.
