\chapter{ЗАДАЧА ТЕПЛОПРОВІДНІСТЬ У М'ЯКИХ ТКАНИНАХ ТІЛА ЛЮДИНИ}

%Hidden cite for correct bibliography ordering
\nocite{bahvalov-et-al,benerdge-et-al,lung-cancer} 

\section{Огляд проблеми}

Використання новітніх підходів для виявлення аномалій в тілі людини, а також для виявлення ушкоджень після 
електро-коагуляції.

Тепловий ендоскоп як описано тут

Thermal endoscope based on cost-effective LWIR camera cores
Dumitru Scutelnic a, Giacomo Marchioro a, Salvatore Siracusano b, Paolo Fiorini a,
Riccardo Muradore a, Claudia Daffara a,

Також роблять для Da Vinci такі ендоскопи.

Потенційно новий прилад користь якого можна апробувати у симуляції.

Для цього треба створити модель м'яких тканин людини із ураженими клітинами. 

Симуляції уражених (ракових) клітин під час малоінвазиdної хіругії.

Рак легенів (\cite{lung-cancer}).

Традиційно за модель біотеплового обміну у тілі людини беруть рівняння Пеннеса.

pennes-1998-analysis-of-tissue-and-arterial-blood-temperatures-in-the-resting-human-forearm

Воно було сформульовано автором у його відомій статті у 1948 році і відтоді використовується для таких розрахуків.

% Для валідації моделі сформулюєм

\section{Формулюваня плоскої задачі}

Спочатку розв'яжемо плоский випадок із стаціонарним розподілом температури.
Візьмемо перетин області проведеннея малоінвазивного втручання

На (рис. \ref{fig:thermo_2d_domain}) зображено область перерізу, яка складається із трьох областей $\Omega_1$, 
$\Omega_2$ і $\Omega_3$. Зовнішня границя $\Gamma$ складається із двох частин $\Gamma_1$ і $\Gamma_2$, тобто 
$\Gamma=\Gamma_1\cap\Gamma_2$. 

\begin{figure}[ht!]
    \centering
    \begin{tikzpicture}
        \draw (1,3) rectangle (11,7);
        \draw (1,1) rectangle (11,3);
        \draw (6,2.5) circle (0.5);

        \draw[thick,->] (0,0) -- (12,0) node[anchor=north west] {x};
        \draw[thick,->] (0,0) -- (0,9) node[anchor=south east] {y};        
    
        \foreach \x in {0,...,4}
            \pgfmathtruncatemacro{\label}{\x * 10}
            \draw ( \x * 2.5 cm,1pt) -- ( \x * 2.5 cm,-1pt) node[anchor=north] {\label};
        \foreach \y in {0,...,3}
            \pgfmathtruncatemacro{\label}{\y * 10}
            \draw (1pt,\y * 2.5 cm) -- (-1pt,\y * 2.5 cm) node[anchor=east] {\label};
        
        \node at (11.5,2) {a};
        \node at (11.5,5) {b};
        \node at (6,7.5) {c};

        \node at (8,2) {$\Omega_1$};
        \node at (8,5) {$\Omega_2$};
        \node at (6,2.5) {$\Omega_3$};
        \node at (6,0.5) {$\Gamma_1$};

    \end{tikzpicture}
    
    \caption{Переріз тканин}
    \label{fig:thermo_2d_domain}
\end{figure}

Задача формулюється так:

\begin{equation}
    \label{eqn:thermo_2d_1}
    \nabla \cdot (k \nabla u) + \omega \rho_b C_b (T_c - u) + q_m  = 0
\end{equation}

На частині границі $\Gamma_2$ задані умови Діріхле, а на $\Gamma_1$ - Неймана.

\begin{equation}
    \label{eqn:thermo_2d_2}
    u(z) = 36, z \in \Gamma_2
\end{equation}

\begin{equation}
    \label{eqn:thermo_2d_3}
    \dfrac{\partial{u(z)}}{\partial{n}} = 0, z \in \Gamma_1
\end{equation}

Коефіцієнт тепропровідності візьмемо на основі статті

Tumor Lung Visualization and Localization through Virtual
Reality and Thermal Feedback Interface
Samir Benbelkacem 1, * , Nadia Zenati-Henda 1 , Nabil Zerrouki 1 , Adel Oulefki 1 , Sos Agaian 2 ,
Mostefa Masmoudi 1 , Ahmed Bentaleb 3 and Alex Liew

and

Tarwidi, D. Godunov method for multiprobe cryosurgery simulation with complex-shaped tumors. AIP Conf. Proc. 2016,


Модель злоякісного утворення взята із статті

Analysis of temperature behavior in biological tissue in photothermal therapy according
to laser irradiation angle
Donghyuk Kim and Hyunjung Kim

Коефіцієнт тепловідності у пухлині рівний 0.495 Вт/мК
У прилеглих тканинах 0.445 Вт/мК і у здорових тканинах 0.19 Вт/мК.
Радіус злоякісної пухлини співрадає із областю $$\Omega_3$$.



Звідси випливає, що коефіцієнт тепропровідності залежить від координат і може бути наближений кусково ламаною функцією 
або функціями вищого порядку, наприклад квадратичною. Для даного прикладу пропонується використати:

\begin{equation}
    \label{eqn:thermo_2d_thermal_conductivity}

    \dfrac{\partial{u(z)}}{\partial{n}} = 0, z \in \Gamma_1
\end{equation}


Відповідно якщо задати 


\section{Дискретизація рівнянь}

\section{Розв'язок}

\section{Формулюваня просторової задачі}

\section{Дискретизація рівнянь}

\section{Розв'язок}

\section{Висновок}
