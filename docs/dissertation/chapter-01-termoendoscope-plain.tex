\chapter{ЗАДАЧА ТЕПЛОПРОВІДНІСТЬ У М'ЯКИХ ТКАНИНАХ ТІЛА ЛЮДИНИ}

%Hidden cite for correct bibliography ordering
\nocite{bahvalov-et-al,tumor-thermal-modell,lung-tumor-thermal-conductivity,database-tissue-properties} 

\section{Огляд проблеми}

Використання новітніх підходів для виявлення аномалій в тілі людини, а також для виявлення ушкоджень після 
електро-коагуляції.

Тепловий ендоскоп як описано тут

Thermal endoscope based on cost-effective LWIR camera cores
Dumitru Scutelnic a, Giacomo Marchioro a, Salvatore Siracusano b, Paolo Fiorini a,
Riccardo Muradore a, Claudia Daffara a,

Також роблять для Da Vinci такі ендоскопи.

Потенційно новий прилад користь якого можна апробувати у симуляції.

Для цього треба створити модель м'яких тканин людини із ураженими клітинами. 

Симуляції уражених (ракових) клітин під час малоінвазиdної хіругії.

Рак легенів (\cite{lung-tumor-thermal-conductivity}).

Традиційно за модель біотеплового обміну у тілі людини беруть рівняння Пеннеса.

pennes-1998-analysis-of-tissue-and-arterial-blood-temperatures-in-the-resting-human-forearm

Воно було сформульовано автором у його відомій статті у 1948 році і відтоді використовується для таких розрахуків.

% Для валідації моделі сформулюєм

\section{Плоска задача}

\subsection{Формулюваня задачі}

Спочатку розв'яжемо плоский випадок із стаціонарним розподілом температури.
Візьмемо перетин області проведеннея малоінвазивного втручання

На (рис. \ref{fig:thermo_2d_domain}) зображено область перерізу, яка складається із трьох областей $\Omega_1$, 
$\Omega_2$ і $\Omega_3$. Зовнішня границя $\Gamma$ складається із двох частин $\Gamma_1$ і $\Gamma_2$, тобто 
$\Gamma=\Gamma_1\cap\Gamma_2$. 

\begin{figure}[ht!]
    \centering
    \begin{tikzpicture}
        \draw (1,3) rectangle (11,7);
        \draw (1,1) rectangle (11,3);
        \draw (6,2.5) circle (0.5);

        \draw[thick,->] (0,0) -- (12,0) node[anchor=north west] {x};
        \draw[thick,->] (0,0) -- (0,9) node[anchor=south east] {y};        
    
        \foreach \x in {0,...,4}
            \pgfmathtruncatemacro{\label}{\x * 10}
            \draw ( \x * 2.5 cm,1pt) -- ( \x * 2.5 cm,-1pt) node[anchor=north] {\label};
        \foreach \y in {0,...,3}
            \pgfmathtruncatemacro{\label}{\y * 10}
            \draw (1pt,\y * 2.5 cm) -- (-1pt,\y * 2.5 cm) node[anchor=east] {\label};
        
        \node at (11.5,2) {a};
        \node at (11.5,5) {b};
        \node at (6,7.5) {c};

        \node at (8,2) {$\Omega_1$};
        \node at (8,5) {$\Omega_2$};
        \node at (6,2.5) {$\Omega_3$};
        \node at (6,0.5) {$\Gamma_1$};

    \end{tikzpicture}
    
    \caption{Переріз тканин}
    \label{fig:thermo_2d_domain}
\end{figure}

\noindent Параметр a рівний 8 см, b - 16 см, а с дорівнює 40 см. Радіус кола $\Omega_3$ r дорівнює 5 мм.

\noindent Задача формулюється так:

\begin{equation}
    \label{eqn:thermo_2d_eqn}
    \nabla \cdot (k \nabla u) + \omega \rho_b C_b (T_c - u) + q_m  = 0
\end{equation}

\noindent На частині границі $\Gamma_2$ задані умови Діріхле, а на $\Gamma_1$ - Неймана.

\begin{equation}
    \label{eqn:thermo_2d_cond_1}
    u(z) = 36, z \in \Gamma_2
\end{equation}

\begin{equation}
    \label{eqn:thermo_2d_cond_2}
    \dfrac{\partial{u(z)}}{\partial{n}} = 0, z \in \Gamma_1
\end{equation}

\noindentУ рівнянні \ref{eqn:thermo_2d_eqn} використовуються наступні позначення:

\begin{center}
    \begin{tabular}{|c|c|} 
        \hline
            k & коефіцієнт тепловідності \\
        \hline
            \(\omega\) & перфузія крові через орган \\
        \hline
            \(\rho_b\) & густина крові \\
        \hline
            \(C_b\) & питома теплоємність крові \\
        \hline
            \(T_c\) & внутрішня температура тіла \\
        \hline
            \(q_m\) & метаболічне тепло у тканині \\
        \hline
    \end{tabular}
\end{center}

\textbf{Коефіцієнт тепропровідності} Коефіцієнт тепропровідності у ракових утвореннях взято із статті 
\cite{lung-tumor-thermal-conductivity}. Модель злоякісного утворення взята із статті \cite{tumor-thermal-model}.
Коефіцієнт тепловідності у пухлині рівний 0.495 Вт/м/К, у прилеглих тканинах 0.445 Вт/м/К і у здорових тканинах 
0.19 Вт/м/К. Злоякісна пухлини предаставлена областю $\Omega_3$. Відповідно коефіцієнт теплопровідності k буде залежати
від координат і його можна представити у вигляді наступої функції:

\begin{equation}
    \label{eqn:thermo_2d_thermal_conductivity}
    k(x, y) = (0.495 - 0.19)cos(\frac{\pi}{2r^2}((x - x_3)^2 + (y - y_3)^2))\chi_3 + 0.19
\end{equation}

\noindent де $\chi_3$ - характеристична функція області $\Omega_3$, a ($x_3$, $y_3$) - центр області.

\noindent На (рис. \ref{fig:thermo_2d_thermal_conductivity}) зображено графік коефіцієнта теплопровідності. 
Припускається, що центр області $\Omega_3$ знаходиться у точці (0, 0) і $y_3$ зафіксована на 0, міняється тільки 
координата x. Для наочності функцію помножили на 5. 

\begin{figure}[ht!]
    \centering
    \begin{tikzpicture}[domain=-5:5]
        \draw[->] (-7,0) -- (7,0) node[right] {$x$};
        \draw[->] (0,-1) -- (0,5) node[above] {$5k(x, y_3)$};
    
        \draw[color=blue] plot (\x, {5 * (0.305 * cos(1.57 / 25 * (\x * \x r)) + 0.19)});
        \draw[color=blue] (-6,0.95) -- (-5,0.95);
        \draw[color=blue] (5,0.95) -- (6,0.95);
    \end{tikzpicture}
    \caption{Коефіцієнт тепловідності}
    \label{fig:thermo_2d_thermal_conductivity}
\end{figure}

\textbf{інші параметри} \cite[Pennes then modeled the arm as a long cylinder and calculated the
steady-state radial temperature profile. In this theoretical prediction, since the blood perfusion rate $\omega$
could not be directly measured, Pennes adjusted this parameter in his model to fit the solution to his
experimental data for a fixed, representative ambient temperature and metabolic heating rate. The fitted
value of blood perfusion rate $\omega$ was found to be between 1.2 and 1.8 mL blood/min/100 g tissue, which
is a typical range of values for resting human skeletal muscle]{kutz-zhu-heat-transfer-biological-systems}

\noindent Головним джерелом даних для фізичних параметрів органів тіла була онлайн база даних 
\cite{database-tissue-properties}. Значення параметрів:

\begin{center}
    \begin{tabular}{|c|c|c|} 
        \hline
            \(\omega\) & 1.5 & mL blood/min/100 g \\
        \hline
            \(\rho_b\) & 1050 & \(\text{кг}/\text{м}^3\) \\
        \hline
            \(C_b\) & 3617 & Дж/кг/С  \\
        \hline
            \(T_c\) & 37 &  \textcelsius \\
        \hline
            \(q_m\) & 4.2 & KK/Kg/\textcelsius \\
        \hline
    \end{tabular}
\end{center}

\subsection{Дискретизація рівнянь}

\subsection{Розв'язок}

\section{Просторова задача}

\subsection{Формулюваня задачі}

\subsection{Дискретизація рівнянь}

\subsection{Розв'язок}

\section{Висновок}
