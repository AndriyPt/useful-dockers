\chapter{МЕТОДИКА ПОЄДНАННЯ ГРАНИЧНИХ І СКІНЧЕНИХ ЕЛЕМЕНТІВ}

%Hidden cite for correct bibliography ordering
\nocite{bahvalov-et-al,benerdge-et-al} 

\section{Огляд методів розв'язування задач термопружності у композитах}

Існують різні підходи, які найбільш часто використовують при дослідженні напружено-деформованого стану тіл.
Коротко зупинимось на деяких з них. З допомогою методу збурення форми
границі {[}\protect\hyperlink{anchor-15}{27}{]} на основі загальних
розв'язків задач теорії пружності для сфери, циліндра і еліпсоїда
обертання одержані наближені розв'язки задач для ізотропного середовища
з включеннями, гранична поверхня яких близька до вказаних. Головною
перевагою цього методу є те, що розв'язок задачі одержується в
замкнутому вигляді, а істотним недоліком те, що для побудови розв'язку
форма поверхні кусково-однорідного тіла і включень має вирішальне
значення. Тому клас задач, які розв'язуються цим методом, обмежений
тілами, граничні поверхні та поверхні розділу середовищ яких є
поверхнями другого порядку та близькими до них, зокрема, в
{[}\protect\hyperlink{anchor-16}{28}{]} розглянута для тривимірного
ізотропного пружного тіла з еліпсоїдальним включенням під дією
одновісного розтягу пружна задача, яка зведена до розв'язування системи
трьох інтегро-диференційних рівнянь відносно стрибків напружень і
переміщень на поверхнях включення, знайдено точний розв'язок цих рівнянь
і виписано формули для наближеного обчислення напружень у матриці і
включенні.

Однак при розв'язуванні багатьох практично важливих задач теорії
пружності потрібно розглядати тіла, поверхні контакту в яких мають
найрізноманітнішу форму. В зв'язку з цим широкого застосування набув
метод R-функцій {[}\protect\hyperlink{anchor-17}{29}{]}, який дає
можливість точно задовольняти граничні умови та умови спряження на
поверхнях контакту, але вимагає розвинутого спеціального математичного
та програмного забезпечення.

У багатьох роботах автори використовують деякі допущення про характер
контакту, зокрема в {[}\protect\hyperlink{anchor-18}{30}{]} розглянуто
задачі про контактну взаємодію двох скінченних стрингерів (накладок) з
різних пружних матеріалів із лінійно-деформованими основами (плоска
деформація) у вигляді пружної нескінченної смуги й анізотропної
півплощини в допущенні, що один із стрингерів перебуває в ідеальному
контакті з основою, а інший контактує з нею через тонкий шар клею.
Задачі зводяться до систем інтегро-алгебричних рівнянь другого роду. У
роботі {[}\protect\hyperlink{anchor-19}{31}{]} за допомогою аналітичних
комплекс­них потенціалів побудовано алгоритм розв'язування плоскої
задачі теорії пружності для трьохкомпонентної композиції, в якій границя
розділу матриці і включення моделюється пружним прошарком, а також дано
кількісну оцінку залежності узагальнених коефіцієнтів концентрації
напружень біля вершини криволі­нійного вирізу і абсолютно жосткого
включення від типу навантаження, пружних характеристик і ширини пружного
прошарку.

Оскільки аналітичний розв'язок багатьох практично важливих задач
(наприклад, для тіл з вирізами і неоднорідними включеннями) часто
утруднений, широкого використання набули числові методи, зокрема,
найбільш повно вивчені методи скінченних різниць
{[}\protect\hyperlink{anchor-20}{32}{]} та скінченних елементів. Вони
застосовуються здебільшого до плоских задач для розрахунку
напружено-деформованого стану пластин, оболонок, осесиметричних та
кусково-однорідних тіл.

Все більшої популярності при розв'язуванні задач набувають МГЕ
{[}\protect\hyperlink{anchor-2}{2}{]}, які ґрунтуються на ідеях,
розроблених С.Г.Міхліним, Н.І.Мусхелішвілі, В.Д.Купрадзе. Перевагами
методу граничних елементів (МГЕ), зокрема, є зниження на одиницю
розмірності задачі внаслідок дискретизації тільки граничної поверхні,
інтегральне зображення розв'язку є кращим, ніж числове диференціювання в
МСЕ, значно простіше досліджуються задачі для нескінчених областей.
Недоліком МГЕ є неможливість розгляду неперервно-неоднорідних тіл, а
також ускладнення алгоритму внаслідок обчислення сингулярних інтегралів.
Порівняння переваг різних методів привело до ідеї доцільності їх
поєднання і до створення на їх основі ефективних числово-аналітичних
методик, які дозволяють оптимізувати процес розв'язування {[}6{]}. У
роботі {[}\protect\hyperlink{anchor-21}{33}{]} побудовано об'єднаний
підхід МГЕ та МСЕ для розв'язання двовимірної статичної задачі теорії
пружності, що дозволяє звести вихідну задачу до послідовності розв'язків
підзадач у підобластях кожним із методів окремо, та розглянуто задачу
про плоску деформацію жорстко защемленого на одній із поверхонь шару під
рівномірно розподіленим навантеженням.

Одним з ефективних методів розрахунку напружень в кусково-однорідних
тілах є підхід, який ґрунтується на математичній постановці відповідних
узагальнених задач спряження, коли на основі частково вироджених
диференціальних рівнянь квазістатичної задачі термопружності для тіла з
чужорідними включеннями отримуються граничні інтегральні рівняння, єдині
для всієї розглядуваної області, які розв'язуються за допомогою прямого
МГЕ, при цьому враховуються умови ідеального та неідеального контакту
{[}\protect\hyperlink{anchor-22}{34},
\protect\hyperlink{anchor-23}{35}{]}.

МПГЕ {[}\protect\hyperlink{anchor-24}{14}{]}, який виник під впливом
методів граничних інтегральних рівнянь, проекційно-сіткових, занурення,
граничних і скінченних елементів та теорії синтезу математичних
алгоритмів, можна розглядати як один з варіантів методу джерела і
віднести його до непрямих методів досліджень, оскільки введені для
одержання розв'язку задачі невідомі не є фізичними змінними. У роботі
{[}\protect\hyperlink{anchor-25}{36},
\protect\hyperlink{anchor-26}{37}{]} зроблено порівняльний аналіз
теоретичних та обчислювальних особливостей МГЕ та МПГЕ для другої
основної задачі пружності та досліджено точність отриманих числових
результатів.

У даній роботі на основі підходів, запропонованих в
{[}\protect\hyperlink{anchor-2}{2}, \protect\hyperlink{anchor-19}{31},
\protect\hyperlink{anchor-24}{14}{]} для розв'язу­вання змішаної
статичної задачі теорії пружності неоднорідних тіл, змодельовано
напружено-деформований стан об'єктів та порів­няно точ­ність отриманих
МПГЕ і МГЕ розв'язків.

\section{Метод приграничних елементів}

TODO

\section{Суть методу поєднання граничних і приграничних елементів}

Одним із недоліків МГЕ є неможливість його застосування у тілах із
локальними неоднорідностями. Для розв'язання цієї проблеми був
запропонований підхід {[}\protect\hyperlink{anchor-5}{7}{]}. Основна
ідея даного підходу полягає у розчеплення оператора задачі на дві
частини, одна із них представляє оператор для однорідного тіла, для якої
МГЕ застосовний, а друга частина містить невідому функцій і
переміщується у праву частину рівняння. Ми отримуємо рівняння, яке за
своїм виглядом нагадує рівняння Пуасона, проте права частина містить
функцію залежну від шуканої функції. Для розв'язування даної проблеми
застосовуються різні підходи, а саме: а) метод пре диктор-коректор б)
метод нев'язок

У наступних розділах подані використання зазначених методик для задач
теплопровідності, пружності і термопружності у тілах із локальними
неоднорідностями.

