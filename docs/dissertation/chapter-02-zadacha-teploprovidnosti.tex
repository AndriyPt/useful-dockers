\chapter{ЗАДАЧА ТЕПЛОПРОВІДНОСТІ ДЛЯ ПЛОСКОГО ТІЛА ІЗ ЛОКАЛЬНО-НЕОДНОРІДНИМ ВКЛЮЧЕННЯМ}

%Hidden cite for correct bibliography ordering
\nocite{bahvalov-et-al,benerdge-et-al} 

\section{Постановка задачі}

Нехай характеристика середовища $ \Lambda (z) $ стосовно певного фізичного процесу,
модельованого потенціалом безвихрового векторного поля,
(теплопровідності, дифузії, електромагнітостатики, пружного руху
твердого тіла, течією ґрунтових вод тощо) є неперервною функцією
декартових координат $ z = (x, y) $,
сталою скрізь у плоскій однозв'язній області $ \Omega $ з простим
замкнутим краєм $ \Gamma = {\Gamma_{1} \cup \Gamma_{2}} $, за
винятком скінченої сукупності локальних неоднорідних включень
$ \Omega_{k}\subset\Omega $ ( $ \Omega_k \underset {k \neq l} \cap \Omega_l = \varnothing $, 
$\partial \Omega_k \cap \Gamma = \varnothing $, $ k,l = \overline{1,K} $).

Розглянемо змішану крайову задачу для основного рівняння стаціонарного
поля в локально-неоднорідному середовищі

% \[\] \[\](2.1)

% \[\] \[{z\in\Gamma_{1},}{}\]\[\] \[{z\in\Gamma_{2},}{}\](2.2)

% де

% \[\] \[\],(2.3)

% \includegraphics[width=0.9571in,height=0.2071in]{./ObjectReplacements/Obj118};\includegraphics[width=0.3783in,height=0.2354in]{./ObjectReplacements/Obj119}
% -- якісна характеристика провідності середовища в
% \[\Omega_{k}{}\]\includegraphics[width=0.5854in,height=0.2929in]{./ObjectReplacements/Obj121},
% причому \[{\lambda_{k}{\left( z \right) \geq 0}}{}\],
% \[{\underset{\Omega_{k}}{\text{max}}\lambda_{k}\left( z \right){= 1}}{}\],
% \[{\lambda_{k}\left( z \right){_{z\in\partial\Omega_{k}} = 0}}{}\];
% \[{\chi_{k}\left( z \right)}{}\] -- характеристична функція області
% \[\Omega_{k}{}\].

% Враховуючи (2.3), задача (2.1)(2.2) набуде вигляду:

% \includegraphics[width=1.25in,height=0.4in]{./ObjectReplacements/Obj127}
% в
% \includegraphics[width=0.1646in,height=0.15in]{./ObjectReplacements/Obj128},(2.4)

% \includegraphics[width=0.3646in,height=0.1929in]{./ObjectReplacements/Obj129}
% на
% \includegraphics[width=0.1646in,height=0.2071in]{./ObjectReplacements/Obj130};\includegraphics[width=0.5283in,height=0.3854in]{./ObjectReplacements/Obj131}
% на
% \includegraphics[width=0.1783in,height=0.2071in]{./ObjectReplacements/Obj132},(2.5)

% де
% \includegraphics[width=1.0146in,height=0.4283in]{./ObjectReplacements/Obj133},
% \includegraphics[width=0.15in,height=0.1646in]{./ObjectReplacements/Obj134}
% - оператор Гамільтона.

% \hypertarget{ux43cux435ux442ux43eux434-ux43fux440ux435ux434ux438ux43aux442ux43eux440-ux43aux43eux440ux435ux43aux442ux43eux440}{%
% \subsection[2.2. Метод
% предиктор-коректор]{\texorpdfstring{\protect\hypertarget{anchor-30}{}{}2.2.
% Метод
% предиктор-коректор}{2.2. Метод предиктор-коректор}}\label{ux43cux435ux442ux43eux434-ux43fux440ux435ux434ux438ux43aux442ux43eux440-ux43aux43eux440ux435ux43aux442ux43eux440}}

% \hypertarget{ux441ux443ux442ux44c-ux43cux435ux442ux43eux434ux443}{%
% \subsubsection[2.2.1. Суть
% методу]{\texorpdfstring{\protect\hypertarget{anchor-31}{}{}2.2.1. Суть
% методу}{2.2.1. Суть методу}}\label{ux441ux443ux442ux44c-ux43cux435ux442ux43eux434ux443}}

% Розглянемо послідовність плоских змішаних крайових задач для основного
% рівняння стаціонарного поля

% \[{\mathit{\text{div}}{\left( {\Lambda_{n}\left( z \right)\mathit{\text{grad}}\phi_{n}\left( z \right)} \right) = 0},}{}\]
% \[{z\in\Omega}{}\];(2.6)

% \[{\phi_{n}{\left( z \right) = \phi_{s}}\left( z \right),}{}\]
% \[{z\in\Gamma_{1}}{}\];(2.7)

% \[{{\frac{\partial\phi_{n}\left( z \right)}{\partial\overrightarrow{n}} = {- \psi_{s}}}\left( z \right),}{}\]
% \[{z\in\Gamma_{2}}{}\](2.8)

% за умови, що характеристика середовища
% \[{\Lambda_{n}(z)({n = \overline{0,N}})}{}\] (коефіцієнт
% теплопровідності, коефіцієнт дифузії тощо) для кожної з цих задач є
% неперервною функцією декартових координат
% \[{z = \left( {x,y} \right)}{}\], сталою скрізь в однозв'язній області
% \[\Omega{}\] з простим замкнутим краєм
% \[{\Gamma = {\Gamma_{1} \cup \Gamma_{2}}}{}\], за винятком скінченої
% сукупності неоднорідних включень \[{\Omega_{k}\subset\Omega}{}\]
% (\[{\Omega_{k}\underset{k \neq l}{\mathit{intersect}}{\Omega_{l} = \varnothing}}{}\],
% \[{\partial\Omega_{k}\mathit{intersect}{\Gamma =}}{}\],
% \[{k,{l = \overline{1,K}}}{}\]), геометрія та конфігурація розташування
% яких фіксовані для всіх \[n{}\]одночасно. А саме:

% \[{\Lambda_{n}{\left( z \right) = {1 + {\sum\limits_{k = 1}^{K}{a_{\mathit{\text{nk}}}\lambda_{k}\left( z \right)\chi_{k}\left( z \right)}}}},}{}\]
% \[{z\in\Omega,}{}\](2.9)

% де \[{\lambda_{k}\left( z \right)}{}\] -- якісна характеристика
% провідності середовища в області \[\Omega_{k}{}\]
% (\[{k = \overline{1,K}}{}\]), причому
% \[{\lambda_{k}{\left( z \right) \geq 0}}{}\],
% \[{\underset{\Omega_{k}}{\text{max}}\lambda_{k}\left( z \right){= 1}}{}\],
% \[{\lambda_{k}\left( z \right){_{z\in\partial\Omega_{k}} = 0}}{}\];
% \[{\chi_{k}\left( z \right)}{}\] -- характеристична функція області
% \[\Omega_{k}{}\]; \[\left\{ a_{\mathit{\text{nk}}} \right\}{}\] --
% монотонна послідовність (\[{n = \overline{0,N}}{}\]) для кожного
% конкретного \[k{}\], така, що \[{a_{0k} = 0}{}\],
% \[{a_{\mathit{\text{nk}}} > {- 1}}{}\].

% Застосуємо до оператора основного рівняння стаціонарного поля (2.6),
% враховуючи (2.9), операції методу адитивного розщеплення операторів.
% Внаслідок цього (2.6) набуде вигляду

% \[{\nabla^{2}{\phi_{n} = P_{n}}\left\lbrack \phi_{n} \right\rbrack}{}\]
% в \[\Omega{}\],(2.10)

% де
% \[{{\nabla = \frac{\partial}{\partial x}}{\overrightarrow{i} + \frac{\partial}{\partial y}}\overrightarrow{j}}{}\]
% -- оператор Гамільтона;
% \[{P_{n}{\left\lbrack {} \right\rbrack = {- \Lambda_{n}^{- 1}}}{\sum\limits_{k = 1}^{K}{a_{\mathit{\text{nk}}}\nabla\lambda_{k}\nabla\left\lbrack {} \right\rbrack}}\chi_{k}}{}\]
% -- оператор, який враховує додатковий вплив неоднорідностей.

% Отже, маємо послідовність змішаних задач (2.10), (2.7), (2.8) для
% визначення розв'язків \[{\phi_{n}\left( z \right)}{}\]
% (\[{n = \overline{0,N}}{}\]).

% \hypertarget{ux43fux43eux431ux443ux434ux43eux432ux430-ux43fux440ux43eux446ux435ux434ux443ux440ux438-ux442ux438ux43fux443-ux43fux440ux43eux433ux43dux43eux437ux43aux43eux440ux435ux43aux442ux443ux432ux430ux43dux43dux44f}{%
% \subsubsection[2.2.2. Побудова процедури типу
% прогноз--коректування]{\texorpdfstring{\protect\hypertarget{anchor-32}{}{}\protect\hypertarget{anchor-33}{}{}2.2.2.
% Побудова процедури типу
% прогноз--коректування}{2.2.2. Побудова процедури типу прогноз--коректування}}\label{ux43fux43eux431ux443ux434ux43eux432ux430-ux43fux440ux43eux446ux435ux434ux443ux440ux438-ux442ux438ux43fux443-ux43fux440ux43eux433ux43dux43eux437ux43aux43eux440ux435ux43aux442ux443ux432ux430ux43dux43dux44f}}

% Для вибраної конкретної\textbf{ }геометрії неоднорідних включень та
% конкретної конфігурації їхнього розташування в \[\Omega{}\] побудуємо
% єдину процедуру, яка дає змогу розв'язувати відразу цілу низку задач
% типу (2.10), (2.7), (2.8), що відрізняються характеристиками середовища
% \[\Lambda_{n}{}\](\[{n = \overline{0,N}}{}\]). У цьому разі внаслідок
% послідовного застосування аналогічно до прийому попереднього моделювання
% шуканого розв'язку у правій частині (2.10) перейдемо до задач відносно
% \[{{\phi_{n}^{} = \phi_{n}^{}}\left( z \right)}{}\].

% Ітерації пропонованої процедури для визначення послідовності функцій
% \[\phi_{i}^{}{}\]задамо такими формулами:

% \[{\nabla^{2}{\phi_{i}^{{}^{m}} = P_{i}}\left\lbrack {\hat{\phi}}_{i^{m - 1}} \right\rbrack}{}\]
% в \[\Omega{}\];(2.11)

% \[{\phi_{i}^{{}^{m}} = \phi_{s}}{}\] на \[\Gamma_{1}{}\];(2.12)

% \[{\frac{\partial\phi_{i}^{{}^{m}}}{\partial\overrightarrow{n}} = {- \psi_{s}}}{}\]
% на \[\Gamma_{2}{}\];(2.13)

% \[{i = \overline{0,N}}{}\]; \[{m = \overline{1,M_{i}}}{}\],

% де \[{\hat{\phi}}_{i}^{m - 1}{}\]-- апроксимація чергової ітерації
% розв'язку \[\phi_{i}^{{}^{m - 1}}{}\]задачі при характеристиці
% середовища \[\Lambda_{i}{}\], причому
% \[{{\hat{\phi}}_{i + 1}^{0} = {\hat{\phi}}_{i}^{M_{i}}}{}\];
% \[\]\[{\hat{\phi}}_{i}^{M_{i}}{}\]-- апроксимація \emph{відкоректованого
% }\[M_{i}{}\] ітераціями розв'язку задачі при характеристиці середовища
% \[\Lambda_{i}{}\]; \[{\hat{\phi}}_{i + 1}^{0}{}\] -- апроксимація
% початкового наближення \emph{прогнозованого }розв'язку при
% \[\Lambda_{i + 1}{}\]; \[{\hat{\phi}}_{0}^{0}{}\] -- апроксимація
% розв'язку наступної задачі для однорідної області

% \[{\nabla^{2}{\phi_{0} = 0}}{}\] в \[\Omega{}\];

% \[{\phi_{0} = \phi_{s}}{}\] на \[\Gamma_{1}{}\];
% \[{\frac{\partial\phi_{0}}{\partial\overrightarrow{n}} = {- \psi_{s}}}{}\]
% на \[\Gamma_{2}{}\];

% Розв'язок кожної із задач (2.11) -- (2.13) зобразимо у вигляді

% \[{\phi_{i}^{{}^{m}}{\left( z \right) = \varphi_{i}^{m}}{\left( z \right) - \Phi_{i}^{m}}\left( z \right)}{}\],

% де

% \[{\Phi_{i}^{m}{\left( z \right) = \frac{1}{2\pi}}{\sum\limits_{k = 1}^{K}{a_{\mathit{\text{ik}}}{\int\limits_{\Omega_{k}}{\text{ln}\left( {\left( {x - x^{}} \right)^{2} + \left( {y - y^{}} \right)^{2}} \right)}}}^{\frac{- 1}{2}}}{\Lambda_{i}^{- 1}\left( z^{} \right)}{\nabla\lambda_{k}\left( z^{} \right)}{\nabla{\hat{\phi}}_{i}^{m - 1}\left( z^{} \right)}{d\Omega_{k}\left( z^{} \right)}}{}\]--
% частковий розв'язок рівняння (2.11);
% \[{\varphi_{i}^{m}\left( z \right)}{}\] -- розв'язок наступної змішаної
% задачі для рівняння Лапласа

% \[{\nabla^{2}{\varphi_{i}^{m} = 0}}{}\] в \[\Omega{}\];

% \[{\varphi_{i}^{m} = {\phi_{s} + \Phi_{i}^{m}}}{}\] на \[\Gamma_{1}{}\];
% \[{\frac{\partial\psi_{i}^{m}}{\partial\overrightarrow{s}} = {{- \psi_{s}} + \frac{\partial\Phi_{i}^{m}}{\partial\overrightarrow{n}}}}{}\]
% на \[\Gamma_{2}{}\],

% який легко можна знайти, використовуючи інтегральну формулу Коші для
% функції
% \[{\omega_{i}^{m}{\left( z \right) = \varphi_{i}^{m}}{\left( z \right) + {i \cdot \psi_{i}^{m}}}\left( z \right)}{}\]
% комплексної змінної \[z{}\]( \[{\varphi_{i}^{m}\left( z \right)}{}\] --
% потенціал; \[{\psi_{i}^{m}\left( z \right)}{}\] -- функція потоку;
% \[\overrightarrow{s}{}\] -- одиничний вектор, дотичний до
% \[\Gamma_{2}{}\] ) згідно з прямим чи непрямим формулюванням алгоритму
% КМГЕ.

% \hypertarget{ux447ux438ux441ux43bux43eux432ux456-ux440ux435ux437ux443ux43bux44cux442ux430ux442ux438}{%
% \subsubsection[2.2.3. Числові
% результати]{\texorpdfstring{\protect\hypertarget{anchor-34}{}{}2.2.3.
% Числові
% результати}{2.2.3. Числові результати}}\label{ux447ux438ux441ux43bux43eux432ux456-ux440ux435ux437ux443ux43bux44cux442ux430ux442ux438}}

% Об'єктом числових досліджень було вибрано послідовність змішаних
% крайових задач теплопровідності типу (2.6) -- (2.9) в області
% \[{\Omega = {\left( {x^{-};x^{+}} \right) \times \left( {y^{-};y^{+}} \right)}}{}\]
% з краєм
% \includegraphics[width=0.7646in,height=0.2217in]{./ObjectReplacements/Obj217},
% де\[\]\textbf{; }\[\]\textbf{;}

% \[\]\textbf{; }\[\]\textbf{ , }яка містить область локальної
% неоднорідності \[\].

% Граничні умови, схарактеризовані так: \[\]\textbf{; }\[\]\textbf{.}

% Якісну характеристику теплопровідності середовища було задано у вигляді

% \[\]\textbf{;
% }\includegraphics[width=0.5429in,height=0.2217in]{./ObjectReplacements/Obj226}\textbf{.}

% Усі графіки, показані на рис. 2.1--2.3, стосуються таких конкретних
% значень параметрів: \[\]; \[\]; \[\]; \[\]; \[\]

% \includegraphics[width=5.111in,height=3.528in]{Pictures/100000000000020200000163358C6F063C52CE42.png}\includegraphics[width=5.111in,height=3.25in]{Pictures/10000000000001EB0000013832389E2C121C29E6.png}

% Рис. 2.1

% На рис. 2.1 наочно простежується залежність температурного поля (рис.
% 2.1,а) від властивостей середовища (рис. 2.1,б). Видно, що зі
% збільшенням відстані до області локальної неоднорідності
% (\includegraphics[width=0.5283in,height=0.2217in]{./ObjectReplacements/Obj232})
% зменшується її вплив на температурне поле, тобто розв'язок задачі щораз
% менше відрізняється від розв'язку аналогічної задачі для випадку
% однорідного середовища.

% \includegraphics[width=5.111in,height=2.889in]{Pictures/100000000000031C000001C112ECC8EDD043C886.png}

% Рис. 2.2

% \includegraphics[width=5.111in,height=2.889in]{Pictures/100000000000031C000001C1A825D2F7BF570EC8.png}

% Рис. 2.3

% Зображені на рис. 2.2, рис. 2.3 графіки демонструють зміну розв'язків
% крайових задач (2.6)--(2.9) уздовж осі
% \includegraphics[width=0.4in,height=0.2217in]{./ObjectReplacements/Obj233},
% що проходить через центр області локальної неоднорідності, де
% найяскравіше виявляються закономірності впливу теплофізичних
% властивостей середовища на розв'язок.

% Крива 1 на рис. 2.2 -- розв'язок задачі без неоднорідності; криві 2--5
% -- розв'язки задач за наявної неоднорідності радіуса
% \includegraphics[width=0.5283in,height=0.2217in]{./ObjectReplacements/Obj234}
% і максимальних відхиленнях у ній значень коефіцієнта теплопровідності
% від значення у решті області, відповідно, у 3, 5, 7, 9 разів. Криві 1--5
% добре демонструють збіжність послідовності розв'язків крайових задач
% (2.6)--(2.9)
% (\includegraphics[width=0.4854in,height=0.2354in]{./ObjectReplacements/Obj235}),
% а також щоразу яскравіший прояв із зростанням провідності ефекту
% ``плато''.

% Графіки на рис. 2.3 одержані при
% \includegraphics[width=0.3646in,height=0.1783in]{./ObjectReplacements/Obj236}
% і значеннях
% \includegraphics[width=1.0429in,height=0.2217in]{./ObjectReplacements/Obj237}
% (відповідно, криві 2--4). Крива 1 відповідає розв'язку задачі для
% однорідного середовища. Можемо спостерігати розширення сфери впливу
% неоднорідності на розв'язок із збільшенням розмірів неоднорідності.

% \hypertarget{ux43cux435ux442ux43eux434-ux43fux43eux454ux434ux43dux430ux43dux43dux44f-ux433ux440ux430ux43dux438ux447ux43dux438ux445-ux435ux43bux435ux43cux435ux43dux442ux456ux432-ux456ux437-ux441ux43aux456ux43dux447ux435ux43dux438ux43cux438-ux435ux43bux435ux43cux435ux43dux442ux430ux43cux438}{%
% \subsection[2.3. Метод поєднання граничних елементів із скінченими
% елементами]{\texorpdfstring{\protect\hypertarget{anchor-35}{}{}2.3.
% Метод поєднання граничних елементів із скінченими
% елементами}{2.3. Метод поєднання граничних елементів із скінченими елементами}}\label{ux43cux435ux442ux43eux434-ux43fux43eux454ux434ux43dux430ux43dux43dux44f-ux433ux440ux430ux43dux438ux447ux43dux438ux445-ux435ux43bux435ux43cux435ux43dux442ux456ux432-ux456ux437-ux441ux43aux456ux43dux447ux435ux43dux438ux43cux438-ux435ux43bux435ux43cux435ux43dux442ux430ux43cux438}}

% \hypertarget{ux441ux443ux442ux44c-ux43cux435ux442ux43eux434ux443-1}{%
% \subsubsection[2.3.1. Суть
% методу]{\texorpdfstring{\protect\hypertarget{anchor-36}{}{}2.3.1. Суть
% методу}{2.3.1. Суть методу}}\label{ux441ux443ux442ux44c-ux43cux435ux442ux43eux434ux443-1}}

% Припустимо, що нам відома апроксимація
% \includegraphics[width=0.1283in,height=0.2071in]{./ObjectReplacements/Obj238}
% функції розв'язку
% \includegraphics[width=0.1283in,height=0.1646in]{./ObjectReplacements/Obj239}
% задачі (2.4),(2.5) в області кожної з локальних неоднорідностей
% \includegraphics[width=0.2071in,height=0.2071in]{./ObjectReplacements/Obj240}.
% Використаємо
% \includegraphics[width=0.1283in,height=0.2071in]{./ObjectReplacements/Obj241}
% у правій частині (2.4). Потрактуємо розв'язок
% \includegraphics[width=0.1354in,height=0.1646in]{./ObjectReplacements/Obj242}
% одержаної крайової задачі

% \includegraphics[width=1.2929in,height=0.4in]{./ObjectReplacements/Obj243}
% в
% \includegraphics[width=0.1646in,height=0.15in]{./ObjectReplacements/Obj244};(2.14)

% \includegraphics[width=0.3783in,height=0.1929in]{./ObjectReplacements/Obj245}
% на
% \includegraphics[width=0.1646in,height=0.2071in]{./ObjectReplacements/Obj246};
% \includegraphics[width=0.5429in,height=0.3854in]{./ObjectReplacements/Obj247}
% на
% \includegraphics[width=0.1783in,height=0.2071in]{./ObjectReplacements/Obj248},(2.15)

% як аналог розв'язку вихідної задачі.

% Далі розв'язок задачі (2.14),(2.15) зобразимо у вигляді

% \includegraphics[width=1.5in,height=0.4in]{./ObjectReplacements/Obj249},(2.16)

% де

% \includegraphics[width=2.1783in,height=0.4429in]{./ObjectReplacements/Obj250},(2.17)

% а
% \includegraphics[width=0.3217in,height=0.2354in]{./ObjectReplacements/Obj251}
% - розв'язок задачі для рівняння Лапласа з видозміненими крайовими
% умовами

% \includegraphics[width=0.4854in,height=0.1929in]{./ObjectReplacements/Obj252}
% в
% \includegraphics[width=0.1646in,height=0.15in]{./ObjectReplacements/Obj253};(2.18)

% \includegraphics[width=1.1146in,height=0.35in]{./ObjectReplacements/Obj254}
% на
% \includegraphics[width=0.1646in,height=0.2071in]{./ObjectReplacements/Obj255};
% \includegraphics[width=1.2929in,height=0.4in]{./ObjectReplacements/Obj256}
% на
% \includegraphics[width=0.1783in,height=0.2071in]{./ObjectReplacements/Obj257},(2.19)

% де
% \includegraphics[width=1.4146in,height=0.4283in]{./ObjectReplacements/Obj258}.

% Шуканий розв'язок задачі (2.18), (2.19) проінтерпретуємо наступним чином
% \includegraphics[width=0.5429in,height=0.1646in]{./ObjectReplacements/Obj259},
% де
% \includegraphics[width=1.4in,height=0.2354in]{./ObjectReplacements/Obj260}
% - аналітична в
% \includegraphics[width=0.3783in,height=0.1929in]{./ObjectReplacements/Obj261}
% функція комплексної змінної
% \includegraphics[width=0.5717in,height=0.1929in]{./ObjectReplacements/Obj262},
% причому потенціал
% \includegraphics[width=0.1283in,height=0.1354in]{./ObjectReplacements/Obj263}
% та функція потоку
% \includegraphics[width=0.1146in,height=0.1354in]{./ObjectReplacements/Obj264}
% є гармонійними в
% \includegraphics[width=0.3783in,height=0.1929in]{./ObjectReplacements/Obj265}.

% Беручи до уваги, що
% \includegraphics[width=0.5146in,height=0.3854in]{./ObjectReplacements/Obj266},
% де
% \includegraphics[width=0.1283in,height=0.1783in]{./ObjectReplacements/Obj267}
% - одиничний додатньоорієнтований (проти руху стрілок годинника) вектор,
% дотичний до
% \includegraphics[width=0.1354in,height=0.15in]{./ObjectReplacements/Obj268},
% перейдемо до розв'язування такої крайової задачі:

% \includegraphics[width=0.6783in,height=0.2354in]{./ObjectReplacements/Obj269}
% в
% \includegraphics[width=0.1646in,height=0.15in]{./ObjectReplacements/Obj270},(2.20)

% \includegraphics[width=1.5in,height=0.35in]{./ObjectReplacements/Obj271}
% на
% \includegraphics[width=0.1646in,height=0.2071in]{./ObjectReplacements/Obj272};(2.21)

% \includegraphics[width=1.6646in,height=0.4429in]{./ObjectReplacements/Obj273}
% на
% \includegraphics[width=0.1783in,height=0.2071in]{./ObjectReplacements/Obj274}.(2.22)

% Для довільної точки
% \includegraphics[width=0.1283in,height=0.1283in]{./ObjectReplacements/Obj275}
% заданої однозв'язної області
% \includegraphics[width=0.1646in,height=0.15in]{./ObjectReplacements/Obj276}
% з простим замкнутим додатньоорієнтованим краєм
% \includegraphics[width=0.1354in,height=0.15in]{./ObjectReplacements/Obj277}
% справедлива інтегральна формула Коші

% \includegraphics[width=1.3783in,height=0.4283in]{./ObjectReplacements/Obj278},(2.23)

% яка слугуватиме основою наступних викладок.

% \hypertarget{ux434ux438ux441ux43aux440ux435ux442ux438ux437ux430ux446ux456ux44f-ux433ux435ux43eux43cux435ux442ux440ux456ux457-ux442ux430-ux432ux438ux437ux43dux430ux447ux435ux43dux43dux44f-ux430ux43fux440ux43eux43aux441ux438ux43cux430ux446ux456ux439}{%
% \subsubsection[2.3.2. Дискретизація геометрії та визначення
% апроксимацій]{\texorpdfstring{\protect\hypertarget{anchor-37}{}{}\protect\hypertarget{anchor-38}{}{}2.3.2.
% Дискретизація геометрії та визначення
% апроксимацій}{2.3.2. Дискретизація геометрії та визначення апроксимацій}}\label{ux434ux438ux441ux43aux440ux435ux442ux438ux437ux430ux446ux456ux44f-ux433ux435ux43eux43cux435ux442ux440ux456ux457-ux442ux430-ux432ux438ux437ux43dux430ux447ux435ux43dux43dux44f-ux430ux43fux440ux43eux43aux441ux438ux43cux430ux446ux456ux439}}

% Дискретизуємо
% \includegraphics[width=0.1354in,height=0.15in]{./ObjectReplacements/Obj279}послідовністю
% \includegraphics[width=0.1646in,height=0.1646in]{./ObjectReplacements/Obj280}
% граничних елементів
% \includegraphics[width=0.1783in,height=0.2071in]{./ObjectReplacements/Obj281}
% (таких, що
% \includegraphics[width=0.35in,height=0.4in]{./ObjectReplacements/Obj282}
% апроксимує
% \includegraphics[width=0.1646in,height=0.2071in]{./ObjectReplacements/Obj283},
% а
% \includegraphics[width=0.4283in,height=0.4in]{./ObjectReplacements/Obj284}
% -
% \includegraphics[width=0.1783in,height=0.2071in]{./ObjectReplacements/Obj285}),
% моделюючи геометрію кожного з елементів за допомогою вектора
% \includegraphics[width=0.1354in,height=0.2071in]{./ObjectReplacements/Obj286}
% базових інтерполюючих функцій, пов'язаних із локальною нормалізованою
% координатою
% \includegraphics[width=0.1283in,height=0.1646in]{./ObjectReplacements/Obj287}.
% Вздовж кожного з елементів апроксимуємо
% \includegraphics[width=0.15in,height=0.1354in]{./ObjectReplacements/Obj288}
% інтерполяційним поліномом

% \includegraphics[width=1.0571in,height=0.2354in]{./ObjectReplacements/Obj289},(2.24)

% де
% \includegraphics[width=0.1783in,height=0.2071in]{./ObjectReplacements/Obj290}
% - вектор невідомих вузлових значень
% функції\includegraphics[width=0.15in,height=0.1354in]{./ObjectReplacements/Obj291}.

% Кожну з областей
% \includegraphics[width=0.2071in,height=0.2071in]{./ObjectReplacements/Obj292}
% дискретизуємо системою ермітових чотирикутних елементів
% \includegraphics[width=0.2646in,height=0.2071in]{./ObjectReplacements/Obj293}\includegraphics[width=0.6929in,height=0.2929in]{./ObjectReplacements/Obj294}
% і представимо
% \includegraphics[width=0.1283in,height=0.2071in]{./ObjectReplacements/Obj295}
% пробною функцією {[}\protect\hyperlink{anchor-39}{24}{]}

% \includegraphics[width=1.4854in,height=0.25in]{./ObjectReplacements/Obj296},(2.25)

% де
% \includegraphics[width=0.2217in,height=0.2354in]{./ObjectReplacements/Obj297}
% - вектор невідомих вузлових значень функції
% \includegraphics[width=0.1283in,height=0.2071in]{./ObjectReplacements/Obj298},
% значень її перших та змішаних похідних на елементі
% \includegraphics[width=0.2646in,height=0.2071in]{./ObjectReplacements/Obj299},
% \includegraphics[width=0.15in,height=0.2071in]{./ObjectReplacements/Obj300}
% - вектор базових функцій у локальній системі координат
% \includegraphics[width=0.65in,height=0.2354in]{./ObjectReplacements/Obj301}.

% Враховуючи (2.24), дискретний аналог (2.23) набуде вигляду

% \includegraphics[width=1.7354in,height=0.4571in]{./ObjectReplacements/Obj302}.(2.26)

% Зважаючи на (2.25),
% \includegraphics[width=0.1646in,height=0.2071in]{./ObjectReplacements/Obj303}
% (подібно й
% \includegraphics[width=0.1929in,height=0.2071in]{./ObjectReplacements/Obj304})
% зобразимо

% \includegraphics[width=1.7783in,height=0.2929in]{./ObjectReplacements/Obj305},

% де

% \includegraphics[width=3.5283in,height=0.4717in]{./ObjectReplacements/Obj306}.

% \hypertarget{ux43fux43eux431ux443ux434ux43eux432ux430-ux441ux438ux441ux442ux435ux43cux438-ux440ux456ux432ux43dux44fux43dux44c-ux434ux43bux44f-ux437ux43dux430ux445ux43eux434ux436ux435ux43dux43dux44f-ux432ux443ux437ux43bux43eux432ux438ux445-ux437ux43dux430ux447ux435ux43dux44c}{%
% \subsubsection[2.3.3. Побудова системи рівнянь для знаходження вузлових
% значень]{\texorpdfstring{\protect\hypertarget{anchor-40}{}{}\protect\hypertarget{anchor-41}{}{}2.3.3.
% Побудова системи рівнянь для знаходження вузлових
% значень}{2.3.3. Побудова системи рівнянь для знаходження вузлових значень}}\label{ux43fux43eux431ux443ux434ux43eux432ux430-ux441ux438ux441ux442ux435ux43cux438-ux440ux456ux432ux43dux44fux43dux44c-ux434ux43bux44f-ux437ux43dux430ux445ux43eux434ux436ux435ux43dux43dux44f-ux432ux443ux437ux43bux43eux432ux438ux445-ux437ux43dux430ux447ux435ux43dux44c}}

% У підрозділі 2.2 запропонована процедура типу прогноз-коректування до
% розв'язування зразу цілої серії задач класу (2.1), (2.2) для конкретно
% вибраної кількості, геометрії та конфігурації розташування неоднорідних
% включень, але різних функцій, що описують провідність середовища. Зараз
% же представимо принципово інший підхід, а саме: систему рівнянь для
% визначення вузлових значень побудуємо за методом зважених нев'язок,
% уведених як для крайових умов, так і для областей локальних включень. У
% матричній формі система матиме, зокрема, якщо скористатися непрямим
% формулюванням методу граничних елементів, такий загальний вигляд:

% \includegraphics[width=2.3217in,height=1.1in]{./ObjectReplacements/Obj307}\includegraphics[width=0.3354in,height=1.5429in]{./ObjectReplacements/Obj308}=\includegraphics[width=0.2646in,height=1.0717in]{./ObjectReplacements/Obj309},(2.27)

% де
% \includegraphics[width=0.1283in,height=0.1783in]{./ObjectReplacements/Obj310},
% \includegraphics[width=0.15in,height=0.2354in]{./ObjectReplacements/Obj311},
% ...,
% \includegraphics[width=0.1783in,height=0.2354in]{./ObjectReplacements/Obj312},
% \includegraphics[width=0.1146in,height=0.1783in]{./ObjectReplacements/Obj313}
% - вектори невідомих вузлових значень відповідно потенціалу на
% дискретному аналозі
% \includegraphics[width=0.1783in,height=0.2071in]{./ObjectReplacements/Obj314},
% дискретних аналогах областей
% \includegraphics[width=0.1929in,height=0.2071in]{./ObjectReplacements/Obj315},
% ...,
% \includegraphics[width=0.2354in,height=0.2071in]{./ObjectReplacements/Obj316},
% а також невідомі вузлові значення функції потоку на дискретному аналозі
% \includegraphics[width=0.1646in,height=0.2071in]{./ObjectReplacements/Obj317}
% краю області\[\Omega{}\].

% Блоки \[V_{j}^{i}{}\] \[\left( {{j = u},v} \right){}\], \[G_{k}{}\]
% \[\left( {k = \overline{1,K}} \right){}\] - суми вкладів відповідно
% дискретного аналога потенціалу і потоку на \[\]\[\Gamma_{i}{}\]
% \[\left( {i = 1,2} \right){}\], а також в \[\Omega_{k}{}\], для уявної
% частини \[\hat{w}{}\] на \[\Gamma_{1}{}\]. Блоки \[U_{j}^{i}{}\]
% \[\left( {{j = u},v} \right){}\], \[T_{k}{}\]
% \[\left( {k = \overline{1,K}} \right){}\] - суми вкладів відповідно
% аналога потенціалу і потоку на \[\]\[\Gamma_{i}{}\]
% \[\left( {i = 1,2} \right){}\], а також в \[\Omega_{k}{}\], для дійсної
% частини \[\hat{w}{}\] на \[\Gamma_{2}{}\]. Блоки
% \[{\overset{\sim}{U}}_{\mathit{\text{jl}}}^{i}{}\]
% \[\left( {{j = u},v} \right){}\],
% \[{\overset{\sim}{T}}_{\mathit{\text{lk}}}{}\]
% \[\left( {k,{l = \overline{1,K}}} \right){}\] - суми вкладів відповідно
% дискретного аналога потенціалу і потоку на \[\]\[\Gamma_{i}{}\]
% \[\left( {i = 1,2} \right){}\], а також в \[\Omega_{k}{}\], для уявної
% частини \[\hat{w}{}\] на \[\Omega_{k}{}\].

% У довільній точці
% \includegraphics[width=0.1283in,height=0.1283in]{./ObjectReplacements/Obj349}
% дискретного аналога області
% \includegraphics[width=0.1646in,height=0.15in]{./ObjectReplacements/Obj350}
% значення потенціалу та функції потоку знаходимо за допомогою (2.26),
% використовуючи розв'язок (2.27).

% \hypertarget{ux447ux438ux441ux43bux43eux432ux456-ux440ux435ux437ux443ux43bux44cux442ux430ux442ux438-1}{%
% \subsubsection[2.3.4. Числові
% результати]{\texorpdfstring{\protect\hypertarget{anchor-42}{}{}2.3.4.
% Числові
% результати}{2.3.4. Числові результати}}\label{ux447ux438ux441ux43bux43eux432ux456-ux440ux435ux437ux443ux43bux44cux442ux430ux442ux438-1}}

% Числові дослідження викладеного підходу проводились для ряду змішаних
% крайових задач теорії потенціалу типу (2.1)-(2.3). Наведемо результати,
% одержані, зокрема, для задачі теплопровідності в області
% \includegraphics[width=0.1646in,height=0.15in]{./ObjectReplacements/Obj351}
% з краєм
% \includegraphics[width=0.6929in,height=0.2071in]{./ObjectReplacements/Obj352},
% де\textbf{:
% }\includegraphics[width=4.0283in,height=0.2646in]{./ObjectReplacements/Obj353}\textbf{;
% }\[\]\textbf{; }\[\]\textbf{, }\[\].

% Граничні умови схарактеризуємо так \[\]\textbf{; }\[\]\textbf{. }

% Усі графіки, наведені на рис. 2.4-2.7, стосуються таких конкретних
% значень параметрів: \[\]; \[\]; \[\]; \[\].

% \includegraphics[width=5.111in,height=3.278in]{Pictures/100000000000026F0000018FE2545062E6ED6A43.png}

% Рис. 2.4

% \includegraphics[width=3.6457in,height=2.3646in]{Pictures/100000000000032200000209705079E12375BE8C.jpg}Рисунок
% 2.4 наочно демонструє збурення температурного поля наявністю системи з
% п'яти локальних включень, максимум коефіцієнта теплопровідності
% матеріалу яких на порядок перевищує значення в однорідній частині.

% Наведені на рис. 2.5 криві 1-3, побудовані для значень
% \[\]\includegraphics[width=0.8854in,height=0.2071in]{./ObjectReplacements/Obj364}
% відповідно, якщо в області -- єдине локальне включення з геометричними
% характеристиками
% \includegraphics[width=0.4146in,height=0.2217in]{./ObjectReplacements/Obj365};
% Рис. 2.5

% \includegraphics[width=0.5283in,height=0.2217in]{./ObjectReplacements/Obj366};
% \includegraphics[width=0.4571in,height=0.2217in]{./ObjectReplacements/Obj367};
% \includegraphics[width=0.5283in,height=0.2217in]{./ObjectReplacements/Obj368}.
% Графіки демонструють зміну температурного поля вздовж осі
% \includegraphics[width=0.3646in,height=0.2071in]{./ObjectReplacements/Obj369},
% що проходить через центр області неоднорідності, де найяскравіше
% проявляються закономірності впливу теплофізичних властивостей середовища
% на розв'язок. Існує точка в зоні включення, розв'язок у якій не залежить
% від коефіцієнта провідності в цій області

% \includegraphics[width=5.111in,height=3.3055in]{Pictures/10000000000004000000029795911D6C1FD05AFF.jpg}

% Рис. 2.6

% Ізотерми 1-9, зображені на рис. 2.6, 2.7, відповідають значенням
% температури 0,1; ...; 0,9. На рисунках вказано область неоднорідного
% включення. Результати одержані при дискретизації кожного ребра
% шестикутника за допомогою п'яти елементів, а області неоднорідного
% включення -- шістнадцяти.

% \includegraphics[width=5.139in,height=3.3335in]{Pictures/100000000000040000000297BBC8A69C9C99964F.jpg}

% Рис. 2.7

% Легко зауважити, що при \[{\lambda_{\text{max}}\rightarrow\infty}{}\]
% розв'язок прямує до розв'язку задачі для однорідної області з отвором в
% області включення, на краю якого задана стала температура; а при
% \[{\lambda_{\text{max}}{\rightarrow + 0}}{}\] - з отвором, край якого
% тепло ізольований.

% \hypertarget{ux43fux456ux434ux441ux443ux43cux43eux43a}{%
% \subsection[2.4.
% Підсумок]{\texorpdfstring{\protect\hypertarget{anchor-43}{}{}2.4.
% Підсумок}{2.4. Підсумок}}\label{ux43fux456ux434ux441ux443ux43cux43eux43a}}

% До розв'язування послідовності задач для основного рівняння
% стаціонарного поля в локально-неоднорідній області застосовано
% комбіновану методику поєднання процедури типу прогноз--коректування з
% комплексним методом граничних елементів. Виконані числові дослідження
% температурного поля для широкого діапазону зміни значень коефіцієнта
% теплопровідності в області з локальними неоднорідними включеннями різних
% розмірів. З'ясовано, що використання методики ефективне як при
% невеликих, так і при значних відхиленнях значень коефіцієнта в області
% неоднорідності від значень у решті тіла.

% Поширено область застосування КМГЕ на змішані плоскі задачі теорії
% потенціалу для однозв'язних областей із сукупністю локальних
% неоднорідних включень. Система рівнянь побудована за методом зважених
% залишків, уведених як для крайових умов, так і для областей локальних
% включень. Проведено широкий спектр числових досліджень стосовно
% співвідношень значень коефіцієнта теплопровідності в областях
% неоднорідних включень і поза ними. Запропоновано прийом моделювання з
% використанням КМГЕ розв'язків задачі для багатозв'язної однорідної\emph{
% }області з теплоізольованими та ізотермічними отворами за допомогою
% розв'язків для локально-неоднорідної\emph{ }області.
