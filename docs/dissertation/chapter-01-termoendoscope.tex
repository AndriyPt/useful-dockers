\chapter{ЗАДАЧА ТЕПЛОПРОВІДНІСТЬ У М'ЯКИХ ТКАНИНАХ ТІЛА ЛЮДИНИ}

%Hidden cite for correct bibliography ordering
\nocite{bahvalov-et-al,tumor-thermal-model,lung-tumor-thermal-conductivity,database-tissue-properties} 

\section{Огляд проблеми}

Одним із пріоритних напрямків покращення малоінвазивної операції є оснащення хірурга додатковою інформацією, яку
він\textbackslashвона не змогли б отримати проводячи відкриту операцію. У перспективі ця інформація може бути
використана для алгоритмів машинного навчання щоб перетворити її кількісні показники у якісні. У цьому розділі
розглянемо потенціал використання теплового ендоскопа для ідентифікації потенційних особливосте у м'яких тканинах на які
хірургу варто звернути увагу. Один зі таких приладів описаний у статті \cite{thermal-endoscope-concept}.

\noindentНаступні речі хіруг може виявити використовуючи тепловий ендоскоп:
\begin{itemize}
    \item Злоякісних пухлин.
    \item Запальні процеси у м'яких тканинах чи органах.
    \item Теплові ушкодження після проведення електро-коагуляції.
\end{itemize}

Зокрема такі підходи цікавлять лідерів індустрії, такиї як компанія Intuitive Machines, виробника робота да Вінчі. І 
відповідно науковці, як це зазначено у статті \cite{thermal-endoscope-concept}, прагнуть розробити відносно дешеві 
прилади щоб здешевити малоінвазивну роботохірургію. Це дозволить ширше застосування роботохірургів. Це одним фактором 
здешевлення операції є зменшення часу виконання процедури. Наявність теплового ендоскопа дозволить швидше виявляти зони 
інтересу для хірургів і оперувати ефективніше.

Для отримання відповідних навиків хірурги зазвичай тренуються у симуляторах. У даному розділі будуть описані задачі, які
потребують роз'язання для ефективного відтворення фізичних процесів програмним забезпеченням. Завдяки отриманим 
роз'язкам можна буде провести дослідження серед контрольної групи хірургів. Зокрема оцінити наскільки тепловий ендоскоп 
спросить задачу виявлення і видалення злоякісних утворень, ідентифікації ушкоджень внаслідок електро-коагуляції і т.д. 
Ці результати у подальшому зможуть призвести до виготовлення приладу і програмного забезпечення, яке здешевить процедуру  
малоінвазивної хірургії.

Математична модель, яка буде використовується у цьому розділі базується на статті Пеннеса
\cite{thermal-process-tissue-pennes}, яка наводить вже класичне рівняння біотеплового процесу у людському організмі. Ця
стаття, була опублікована у 1948 році, але до тепер залишається актуальною для більшості досліджень. Зокрема це рівняння
використовується для оцінки ракових утворень у легенях \cite{lung-tumor-thermal-conductivity}.

У цьому розділі будуть розглядатися дві задачі. Перша задача описує стаціонарний процес у черевній порожнині із одним 
неоднорідним ключенням. Друга задача описує стаціонарний просторовий випадок із одним ключенням. Подаються потенційні 
зображення, які буде бачити хірург у симульований тепловий ендоскоп.

\section{Плоска стаціонарна задача}

\subsection{Формулюваня задачі}

Спочатку розв'яжемо плоский випадок із стаціонарним розподілом температури.
Візьмемо перетин області проведеннея малоінвазивного втручання

На (рис. \ref{fig:thermo_2d_domain}) зображено область перерізу, яка складається із трьох областей $\Omega_1$, 
$\Omega_2$ і $\Omega_3$. Зовнішня границя $\Gamma$ складається із двох частин $\Gamma_1$ і $\Gamma_2$, тобто 
$\Gamma=\Gamma_1\cap\Gamma_2$. 

\begin{figure}[ht!]
    \centering
    \begin{tikzpicture}
        \draw (1,3) rectangle (11,7);
        \draw (1,1) rectangle (11,3);
        \draw (6,2.5) circle (0.5);

        \draw[thick,->] (0,0) -- (12,0) node[anchor=north west] {x};
        \draw[thick,->] (0,0) -- (0,9) node[anchor=south east] {y};        
    
        \foreach \x in {0,...,4}
            \pgfmathtruncatemacro{\label}{\x * 10}
            \draw ( \x * 2.5 cm,1pt) -- ( \x * 2.5 cm,-1pt) node[anchor=north] {\label};
        \foreach \y in {0,...,3}
            \pgfmathtruncatemacro{\label}{\y * 10}
            \draw (1pt,\y * 2.5 cm) -- (-1pt,\y * 2.5 cm) node[anchor=east] {\label};
        
        \node at (11.5,2) {a};
        \node at (11.5,5) {b};
        \node at (6,7.5) {c};

        \node at (8,2) {$\Omega_1$};
        \node at (8,5) {$\Omega_2$};
        \node at (6,2.5) {$\Omega_3$};
        \node at (6,0.5) {$\Gamma_1$};

    \end{tikzpicture}
    
    \caption{Переріз тканин}
    \label{fig:thermo_2d_domain}
\end{figure}

\noindent Параметр a рівний 8 см, b - 16 см, а с дорівнює 40 см. Радіус кола $\Omega_3$ r дорівнює 5 мм. Область 
$\Omega_1$ - моделює черевнину порожнину у яку закачаний газ, для збільшення операційного простору. $\Omega_2$ - це 
м'які тканини черевної порожнини, над якими буде проводитися операційне втручанння, а область $\Omega_3$ - це злоякісне
утворення в м'яких тканинах.

\noindent Задача формулюється так:

\begin{equation}
    \label{eqn:thermo_2d_eqn_co2}
    \nabla \cdot (k_1 \nabla u(z)) = 0, z \in \Omega_1
\end{equation}

\begin{equation}
    \label{eqn:thermo_2d_eqn_tissue}
    \nabla \cdot (k_2(z) \nabla u(z)) + \omega \rho_b C_b (T_c - u(z)) + q_m  = 0, z \in \Omega_2
\end{equation}

\noindent На частині границі $\Gamma_2$ задані умови Діріхле, а на $\Gamma_1$ - Неймана.

\begin{equation}
    \label{eqn:thermo_2d_cond_1}
    u(z) = 36, z \in \Gamma_2
\end{equation}

\begin{equation}
    \label{eqn:thermo_2d_cond_2}
    \dfrac{\partial{u(z)}}{\partial{n}} = 0, z \in \Gamma_1
\end{equation}

\noindentУ рівняннях \ref{eqn:thermo_2d_eqn_co2} і \ref{eqn:thermo_2d_eqn_tissue} використовуються наступні позначення:

\begin{center}
    \begin{tabular}{|c|c|} 
        \hline
            \(k_1 і k_2\)  & коефіцієнти тепловідності \\
        \hline
            \(\omega\) & перфузія крові через орган \\
        \hline
            \(\rho_b\) & густина крові \\
        \hline
            \(C_b\) & питома теплоємність крові \\
        \hline
            \(T_c\) & внутрішня температура тіла \\
        \hline
            \(q_m\) & метаболічне тепло у тканині \\
        \hline
    \end{tabular}
\end{center}

\textbf{Коефіцієнт тепропровідності у м'яких тканинах} Коефіцієнт тепропровідності у ракових утвореннях взято із статті 
\cite{lung-tumor-thermal-conductivity}. Модель злоякісного утворення взята із статті \cite{tumor-thermal-model}.
Коефіцієнт тепловідності у пухлині рівний 0.495 Вт/м/К, у прилеглих тканинах 0.445 Вт/м/К і у здорових тканинах 
0.19 Вт/м/К. Злоякісна пухлини предаставлена областю $\Omega_3$. Відповідно коефіцієнт теплопровідності k буде залежати
від координат і його можна представити у вигляді наступої функції:

\begin{equation}
    \label{eqn:thermo_2d_thermal_conductivity}
    k_1(x, y) = (0.495 - 0.19)cos(\frac{\pi}{2r^2}((x - x_3)^2 + (y - y_3)^2))\chi_3 + 0.19
\end{equation}

\noindent де $\chi_3$ - характеристична функція області $\Omega_3$, a ($x_3$, $y_3$) - центр області.

\noindent На (рис. \ref{fig:thermo_2d_thermal_conductivity}) зображено графік коефіцієнта теплопровідності. 
Припускається, що центр області $\Omega_3$ знаходиться у точці (0, 0) і $y_3$ зафіксована на 0, міняється тільки 
координата x. Для наочності функцію помножили на 5. 

\begin{figure}[ht!]
    \centering
    \begin{tikzpicture}[domain=-5:5]
        \draw[->] (-7,0) -- (7,0) node[right] {$x$};
        \draw[->] (0,-1) -- (0,5) node[above] {$5k(x, y_3)$};
    
        \draw[color=blue] plot (\x, {5 * (0.305 * cos(1.57 / 25 * (\x * \x r)) + 0.19)});
        \draw[color=blue] (-6,0.95) -- (-5,0.95);
        \draw[color=blue] (5,0.95) -- (6,0.95);
    \end{tikzpicture}
    \caption{Коефіцієнт тепловідності}
    \label{fig:thermo_2d_thermal_conductivity}
\end{figure}

\textbf{Iнші параметри} \cite[Пеннес змоделював руку як довгий цилінд і обчислив стаціонарний розроділ температури. 
У його моделі не можна було просто виміряти перфузію крові через м'які тканини $\omega$, тому Пеннес узгодив цей 
параметр із експерементальними даними для фіксованої {\hl???ambient???} температури і метаболічного виділення тепла. Він 
отримав значення перфузії крові $\omega$ між 1.2 і 1.8 мл/хв/100г м'яких тканин, що є типовими діапазоном значень для
м'язів людської руки у стані спокою]{kutz-zhu-heat-transfer-biological-systems}. Усерднимо це значення і візьмемо 
1.5 мл/хв/100г м'яких тканин, як значення $\omega$.

\noindent Головним джерелом даних для фізичних параметрів органів тіла була онлайн база даних 
\cite{database-tissue-properties}. Значення параметрів:

\begin{center}
    \begin{tabular}{|c|c|c|} 
        \hline
            \(k_2\) & 1.5 ??????? & Дж/кг/С \\
        \hline
            \(\omega\) & 1.5 & мл/хв/100г \\
        \hline
            \(\rho_b\) & 1050 & \(\text{кг}/\text{м}^3\) \\
        \hline
            \(C_b\) & 3617 & Дж/кг/С  \\
        \hline
            \(T_c\) & 37 &  \textcelsius \\
        \hline
            \(q_m\) & 4.2 & ккал/кг/\textcelsius \\
        \hline
    \end{tabular}
\end{center}

\subsection{Дискретизація рівнянь}

\subsection{Розв'язок}

\section{Просторова стаціонарна задача}

\subsection{Формулюваня задачі}

\subsection{Дискретизація рівнянь}

\subsection{Розв'язок}

\section{Висновок}
