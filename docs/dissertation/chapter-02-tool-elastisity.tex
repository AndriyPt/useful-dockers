\chapter{ЗАДАЧА ПРУЖНОСТІ У М'ЯКИХ ТКАНИНАХ ТІЛА ЛЮДИНИ}

%Hidden cite for correct bibliography ordering
\nocite{bahvalov-et-al,benerdge-et-al} 

\section{Огляд проблеми}

Одним із нововведень у малоінвазивній робототехніці був тактильний відгук від інструментів. Це була суттєва різниця між
відкритою операцією і малоінвазивною хірургією. Під час вікритої операції хірург міг\textbackslashмогла відчути певні
характеристики руками і оцінити властивосте м'яких тканин. Тільки остання версія робота хірурга від Intuitive Machines
да Вінчі отримав таку функціональність (???джерело???). Першим роботом був Senhance із такою функціональністю
(джурело??? Стання на NHI). Тим не менше, наразі обмежене коло роботів хірургів надає таку функціональність.

Для того щоб можна було розробляти такі системи управління роботами із тактильним відгуком потрібно створити симульоване 
середовище, яке буде надавати достовірні дані для калібрування інструментів. Його також можна використати для навчання
хірургів і виробляння звички ефективного управління такою системою. Однією із переваг такого керування роботом була б 
здатність відчувати неоднорідні включення у м'яких тканинах, які по аналогії із відкритою операцією можна використати 
для виявлення пухлин чи інших злоякісних утворень.

У цьому розділі розглянуто дві задачі. Перша плоска, а друга - просторова стаціонарна задача контакту кінця хірургічного 
інстременту із неоднорідним включенням у м'яких тканинах. 

\section{Формулюваня плоскої задачі}

\section{Дискретизація рівнянь}

\section{Розв'язок}

\section{Формулюваня просторової задачі}

\section{Дискретизація рівнянь}

\section{Розв'язок}

\section{Висновок}
